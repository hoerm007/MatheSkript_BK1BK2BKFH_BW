\section{Funktionen}
\subsection{Grundlagen}
Wir werden im folgenden die drei Funktionstypen der ganzrationalen Funktionen, der Exponentialfunktionen und der trigonometrischen Funktionen betrachten. Doch zuerst wollen wir uns �berlegen was wir unter einer Funktion verstehen wollen und wie wir formal korrekt mit Funktionen umgehen k�nnen.
\begin{tcolorbox}
\begin{defi}
Unter einer Funktion verstehen wir eine Abbildung, die von einer Menge in eine andere Menge abbildet. Dabei wird \textbf{jedem} Element bzw. jeder Zahl der Ausgangsmenge (Definitionsmenge genannt) \textbf{genau ein} Element bzw. \textbf{genau eine} Zahl in der Zielmenge (Wertemenge genannt) zugeordnet.
\end{defi}
\end{tcolorbox}
Um verschiedene Funktionen unterscheiden zu k�nnen, benennen wir Funktionen. �blicherweise werden Kleinbuchstaben, z.B. $f$, $g$, $h$, usw. verwendet. Oft werden statt verschiedenen Buchstaben die Funktionen auch einfach durchgez�hlt mit einem tiefgestellten Index, z.B. $f_1$, $f_2$, $f_3$, usw. Um anzuzeigen welche Menge die Definitionsmenge und welche Menge die Wertemenge ist, verwendet man folgende Schreibweise:
\begin{align}
f:\text{Definitionsmenge}\rightarrow\text{Wertemenge}
\end{align}
Soweit nicht anders angegeben werden wir Funktionen betrachten, die aus den reellen Zahlen in die reellen Zahlen abbilden, also ${f:\R\rightarrow\R}$. Die Zuordnungsvorschrift kann auf verschiedenen Wegen angegeben werden:
\begin{itemize}
	\item Verbal: Jeder Zahl der Definitionsmenge wird das Quadrat dieser Zahl zugeordnet.
	\item $f(x)=x^2$, dabei ist $f$ der Name der Funktion, $x$ ein beliebiges Element aus der Definitionsmenge und $x^2$ ist das Element der Wertemenge, das $x$ zugeordnet wird.
	\item $f:x\mapsto x^2$, eine weniger bekannte Schreibweise bei der klarer wird, dass $x$ auf $x^2$ abgebildet wird. Man beachte den Pfeil mit dem kleinen vertikalen Strich.
\end{itemize}
Ein Teil der Zuordnungen kann auch als Wertetabelle dargestellt werden:
\begin{align*}
\renewcommand\arraystretch{1.5}
\begin{array}{|c||c|c|c|c|c|}
	\hline
	x		&0&1&2&3&4\\
	\hline
	f(x)=x^2	&0&1&4&9&16\\
	\hline
\end{array}
\end{align*}

\subsection{Wiederholung: lineare und quadratische Funktionen}
\begin{tcolorbox}
Eine Funktion $f(x)$ wird als lineare Funktion bezeichnet, wenn man sie in der folgenden Form darstellen kann:
\begin{align}
f(x)=m\cdot x+b \quad m, b \in \R
\end{align}
$m$: Steigung 
$c$: y-Achsenabschnitt
\end{tcolorbox}
Die Schaubilder linearer Funktionen sind Geraden. Der y-Achsenabschnitt ist der Funktionswert, der von $f$ an der Stelle $x=0$ angenommen wird. Dieser Wert entspricht dem y-Wert bei dem das Schaubild die y-Achse schneidet.\\
Die Steigung $m$ kann im Schaubild mit Hilfe des Steigungsdreiecks bestimmt werden:
\begin{align*}
m=\frac{\Delta y}{\Delta x}=\frac{\text{Unterschied der }y\text{-Werte}}{\text{Unterschied der }x\text{-Werte}}
\end{align*}
Als Beispiele betrachten wir die beiden Funktionen ${f_1(x)=\frac{1}{2}x+1}$ und ${f_2(x)=-2x+3}$:
\begin{tikzpicture}[scale=1]
\draw [step=1cm,lightgray,very thin] (-6.9,-6.9) grid (6.9,6.9);
\foreach \x in {-6,...,-1}
\draw [very thick] (\x cm,-3pt) -- (\x cm,3pt) node [below=5pt,fill=white] {$\x$};
\foreach \x in {1,...,6}
\draw [very thick] (\x cm,-3pt) -- (\x cm,3pt) node [below=5pt,fill=white] {$\x$};
\foreach \y in {-6,...,-1}
\draw [very thick] (-3pt,\y cm) -- (3pt,\y cm) node [anchor=west,fill=white] {$\y$};
\foreach \y in {1,...,6}
\draw [very thick] (-3pt,\y cm) -- (3pt,\y cm) node [anchor=west,fill=white] {$\y$};
\draw [->,very thick] (-7,0) -- (7,0);
\draw [->,very thick] (0,-7) -- (0,7);
\draw (7,0) node[below=5pt, fill=white] {\textbf{x}};
\draw (0,7) node[right=5pt, fill=white] {\textbf{y}};
\draw[name path= f1, very thick,color=red,domain=-7:7] plot (\x, 1/2*\x+1) node[right] {$f_1(x)=\frac{1}{2}x+1$};
\begin{scope}
\clip (-7,-7) rectangle (7,7);
\draw[very thick,color=blue,domain=-7:7] plot (\x, -2*\x+3);
\end{scope}
\draw (4,-5) node[left=5pt, color=blue, fill=white] {$f_2(x)=-2x+3$};
% Steigungsdreieck f�r f_1
\path [name path=f1x] (0,2) -- (4,2);
\draw [name intersections={of=f1 and f1x, by=s}] [very thick, orange] (s) -- (4,2) node[midway, below=4pt, fill=white] {$\Delta x=2$};
\draw [very thick, orange] (4,2)--(4,3) node [midway, right, fill=white] {$\Delta y=1$};
\draw [very thick, orange] (-6,-2) -- (-2,-2) node[midway, below=4pt, fill=white] {$\Delta x=4$};
\draw [very thick, orange] (-2,-2)--(-2,0) node [midway, right, fill=white] {$\Delta y=2$};
%Steigungsdreieck f�r f_2
\draw [very thick, orange] (2,-1) -- (4.5,-1) node[midway, below=0pt, fill=white] {$\Delta x=2,5$};
\draw [very thick, orange] (4.5,-1)--(4.5,-6) node [midway, right, fill=white] {$\Delta y=-5$};
\end{tikzpicture}
Das Steigungsdreieck kann an einer beliebigen Stelle und beliebig gro� eingezeichnet werden. Das Verh�ltnis von $\Delta y$ zu $\Delta x$ und damit die Steigung bleibt immer gleich.