\documentclass[a4paper,12pt, headsepline, ngerman]{scrartcl}

\usepackage{scrlayer-scrpage}
\usepackage[nodisplayskipstretch]{setspace} %vspace before/after math mode
%\setstretch{5}
\raggedbottom
\onehalfspacing
%\renewcommand*{\familydefault}{\sfdefault}



\pagestyle{scrheadings} %KOMA-Script mit Kopf-Fuß-Zeilen
\usepackage{hyperref}			%Hyperlinks setzen
\usepackage{babel}				%Silbentrennung mit ngerman

\usepackage{booktabs} 			% For prettier tables

\usepackage{mathtools}  		%Mathe-Paket
\usepackage{amssymb}			%Symbole
\usepackage{bbm}				%\mathbbm{N} für natürliche Zahlen o.ä.
%\usepackage{graphicx}			%Optionen für \includegraphics{imagefile}
%Solte in mathtools beinhaltet sein
\usepackage{color}				%\textcolor{blue}{text...}
\usepackage[dvipsnames]{xcolor}
%Häufig verwendetet Farben
%ForestGreen 		RGB(  0,155, 85)
%YellowOrange		RGB(250,162, 26)

\usepackage[T1]{fontenc}		%Umlaute




\usepackage{ulem}
%\uline{important} underlined text
%\uuline{urgent} double-underlined text
%\uwave{boat} wavy underline
%\sout{wrong} line struck through word
%\xout{removed} marked over like //////////
%\dashuline{dashing} dashed underline
%\dotuline{dotty} dotted underline

%\usepackage{}
\usepackage{cancel}					%Durchstreichen von Dingen in Formeln
\usepackage{enumitem}				%Aufzählungen [label=\alph*)]
\setlist[enumerate]{nosep, topsep=0pt}	%Kleinere Abstände bei Aufzählungen
\setlist[itemize]{noitemsep, topsep=0pt}

\usepackage{framed}                 %Rahmen machen \begin{framed} ... \end{framed}

\usepackage{tcolorbox} 				%Für Boxen um Text
\usepackage{array}
%Plots
\usepackage{tikz}
\usepackage{pgf}
\usepackage{pgfmath}
\usepackage{systeme}				% Für LGS
\sysdelim..							% Keine geschweifte Klammer neben LGS

\usepackage{bm}						%\bm{xxx} for bold in math mode

\usepackage{adjustbox}
\usepackage{geometry}
\geometry{a4paper, portrait, left=1.5cm, right=2cm, top=1cm, bottom=2cm, headsep=0.2cm, includehead, head=27.30193pt}

\usepackage{marvosym} %\Lightning
\usepackage{multirow}
\renewcommand{\mvchr}[1]{\Large{\mbox{\mvs\symbol{#1}}}} %\Lightning in math mode

\usetikzlibrary{intersections}

\setkomafont{headsepline}{\color{black}}

\usepackage{amsthm}					%Definitionsumgebung für \newtheorem{defi}{Definition}[section] usw.
\theoremstyle{definition}
\newtheorem{defi}{Definition}[subsection]
\newtheorem*{bsp}{Beispiel}
\newtheorem{kon}[defi]{Konstruktion}
\newtheorem{nota}[defi]{Notation}
\newtheorem{cha}[defi]{Charakterisierung}
\newtheorem{norm}[defi]{Normierung}
\newtheorem{bem}[defi]{Bemerkung}
\newtheorem{folg}[defi]{Folgerung}
\newtheorem{beob}[defi]{Beobachtung}
\newtheorem{erin}[defi]{Erinnerung}
\newtheorem{sit}[defi]{Situation}
%Einheiten
\newcommand{\ms}{\frac{m}{s}}
\newcommand{\kmh}{\frac{km}{h}}
%Mathebefehle
\newcommand{\beq}{\begin{align}}
\newcommand{\eeq}{\end{align}}
\newcommand{\beqn}{\begin{align*}}
\newcommand{\eeqn}{\end{align*}}
\newcommand{\td}{\ \text{d}}
\newcommand{\ul}{\underline}
\newcommand{\tTr}{\text{Tr}}
\newcommand{\bra}[1]{\langle #1\vert}
\newcommand{\braa}{\langle}
\newcommand{\ket}[1]{\vert#1\rangle}
\newcommand{\kett}{\rangle}
\newcommand{\braket}[2]{\langle #1\vert#2\rangle}
\newcommand{\mH}{\mathcal{H}}
\newcommand{\R}{\mathbb{R}}
\newcommand{\N}{\mathbb{N}}
\newcommand{\Z}{\mathbb{Z}}
\newcommand{\Q}{\mathbb{Q}}

%Exercise-Paket Umbenennungen
\usepackage[]{exercise}			%Nach hyperref einbinden! answerdelayed
\renewcommand{\listexercisename}{Liste der Aufgaben}%
\renewcommand{\ExerciseName}{Aufgabe}%
\renewcommand{\AnswerName}{L{\"o}sung zu Aufgabe}%
\renewcommand{\ExerciseListName}{Aufg.}%
\renewcommand{\AnswerListName}{L{\"o}sung}%
\renewcommand{\ExePartName}{Teil}%
\renewcommand{\ArticleOf}{von\ }%
%\renewcommand{\ExerciseHeaderTitle}{\ExerciseTitle}
\renewcommand{\ExerciseHeader}{%
	\textbf{\large\ExerciseHeaderDifficulty\ExerciseName\ %
	\ExerciseHeaderNB\normalsize\ExerciseHeaderTitle\ExerciseHeaderOrigin}\medskip}
\renewcommand{\AnswerHeader}{
	\newpage\textbf{L{\"o}sung zu \ExerciseName\ \ExerciseHeaderNB}\smallskip}


\definecolor{tcback}{rgb}{.95,.95,.95}
%Farbe für die Lösungen, die die Schüler selbst ausfüllen sollen
\definecolor{loes}{rgb}{1,1,1}
\definecolor{loestc}{rgb}{.95,.95,.95}
%Arbeitsblatt-Modus
\definecolor{loes}{rgb}{.0,.13,.8}
\definecolor{loestc}{rgb}{.6,.0,.1}
\tcbset{
%	frame code={}
%	center title,
%	left=0pt,
%	right=0pt,
%	top=0pt,
%	bottom=0pt,
	colback=tcback,
%	colframe=white,
%	width=\dimexpr\textwidth\relax,
%	enlarge left by=0mm,
%	boxsep=5pt,
%	arc=0pt,outer arc=0pt,
}
%\sqrt[\leftroot{0}\uproot{2}n]{x}
\newcommand{\fakesection}[1]{%
	\par\refstepcounter{section}% Increase section counter
	\sectionmark{#1}% Add section mark (header)
	\addcontentsline{toc}{section}{\protect\numberline{\thesection}#1}% Add section to ToC
	% Add more content here, if needed.
}
\newcommand{\fakesubsection}[1]{%
	\par\refstepcounter{subsection}% Increase section counter
	\subsectionmark{#1}% Add section mark (header)
	\addcontentsline{toc}{subsection}{\protect\numberline{\thesubsection}#1}% Add section to ToC
	% Add more content here, if needed.
}

%\sqrt[\leftroot{0}\uproot{1}n]{y} für bessere Wurzeln

\begin{document}
	\setlength\parindent{0pt} %keine Einrückungen beim Start eines Paragraphen

	%Header
	\lohead{Mathematik}
	%\cohead{} %im Arbeitsblatt
	\rohead{Datum \phantom{00.00.0000}}
	\lehead{lehead}
	\cehead{cehead}
	\rehead{rehead}
%	\cofoot[plain content]{} %keine Seitenzahlen
	\cofoot[\pagemark]{\pagemark}
	\title{Mathematik\\
		Ein Skript für das Berufskolleg 1}
	\author{Hermann Maier}
%	\maketitle
%	\newpage
%	\tableofcontents
%	\newpage

	%\setlength\extrarowheight{10pt} %Horizontales padding für Tabellen
	\def\grundlagen{./Arbeitsblaetter/Grundlagen}
	\def\linFkt{./Arbeitsblaetter/LineareFunktionen}
	\def\quadFkt{./Arbeitsblaetter/QuadratischeFunktionen}
	\def\ganzFkt{./Arbeitsblaetter/GanzrationaleFunktionen}
	\def\eFkt{./Arbeitsblaetter/EFunktionen}
	\def\ableitung{./Arbeitsblaetter/Ableitung}
	\def\lgs{./Arbeitsblaetter/LGS}
	\def\optimierung{./Arbeitsblaetter/Optimierung}
	\def\integration{./Arbeitsblaetter/Integration}
	\def\trigonometrie{./Arbeitsblaetter/Trigonometrie}
	%%%%%%%%%%%%%%%%%%%%%%%%%%%%%%%%%%%%%%%%%%%%%%%%%%%%%%%%%%%%%%%%%%%%%%%%%
%	\fakesection{Grundlagen}
%	\input{\grundlagen/Mengen.tex}
%	\newpage
%	\input{\grundlagen/EinfachesRechnen.tex}
%	\newpage
%	\input{\grundlagen/Mitternachtsformel.tex}
%	\newpage
%	\cohead{\Large\textbf{Lösungen}}
%	\fakesubsection{Lösungen}
%	\shipoutAnswer
%	\newpage
%	%%%%%%%%%%%%%%%%%%%%%%%%%%%%%%%%%%%%%%%%%%%%%%%%%%%%%%%%%%%%%%%%%%%%%%%%%
%	\fakesection{Lineare Funktionen}
%	\input{\linFkt/Begriffe.tex}
%	\newpage
%	\input{\linFkt/Einfuehrung.tex}
%	\newpage
%	\input{\linFkt/Punktprobe.tex}
%	\newpage
%	\input{\linFkt/Nullstellen.tex}
%	\newpage
%	\cohead{\Large\textbf{Gegenseitige Lage von Geraden}}
\fakesubsection{Gegenseitige Lage von Geraden}

\begin{minipage}{\textwidth}
	\centering{\Large\textcolor{loes}{Parallele Geraden}}
	\adjustbox{valign=t}{\begin{minipage}{0.5\linewidth}\centering
		\includegraphics[width=0.95\textwidth]{\linFkt/pics/lage1.png}
		
		\(f_1(x)=\tfrac{1}{2}x-2\qquad g_1(x)=0,5x+1\)
	\end{minipage}}%
	\adjustbox{valign=t}{\begin{minipage}{0.5\linewidth}\centering
		\begin{tcolorbox}[width=0.95\textwidth, height=1.2cm, valign=center]\centering
			\textcolor{loestc}{\(m_f=m_g\)}
		\end{tcolorbox}
		\raggedright
		\textcolor{loes}{Die beiden Geraden haben die gleiche Steigung. Solche Paare von Geraden nennt man parallele Geraden. Sie haben keinen Schnittpunkt, d.h. die Gleichung \(f(x)=g(x)\) hat keine Lösungen. Paralle Geraden, die auch den gleichen y-Achsenabschnitt haben, nennt man identische Geraden. In diesem Fall ist jedes \(x\) eine Lösung der Gleichung \(f(x)=g(x)\)}.
	\end{minipage}}%
\end{minipage}

\bigskip

\begin{minipage}{\textwidth}
	\adjustbox{valign=t}{\begin{minipage}{0.5\linewidth}
		\centering{\Large\textcolor{loes}{Senkrechte Geraden}}
	\end{minipage}}%
	\adjustbox{valign=t}{\begin{minipage}{0.5\linewidth}
		\centering{\Large\textcolor{loes}{keine besondere Lage}}
	\end{minipage}}%
\end{minipage}%

\begin{minipage}{\textwidth}
	\adjustbox{valign=t}{\begin{minipage}{0.5\linewidth}
			\centering{\includegraphics[width=0.95\textwidth]{\linFkt/pics/lage2.png}}	
	\end{minipage}}%
	\adjustbox{valign=t}{\begin{minipage}{0.5\linewidth}
			\centering{\includegraphics[width=0.95\textwidth]{\linFkt/pics/lage3.png}}
	\end{minipage}}%
\end{minipage}%

\bigskip

\begin{minipage}{\textwidth}
	\adjustbox{valign=t}{\begin{minipage}{0.5\linewidth}
			\centering\(f_2(x)=-\tfrac{1}{2}x+2\qquad g_2(x)=2x-1\)
	\end{minipage}}%
	\adjustbox{valign=t}{\begin{minipage}{0.5\linewidth}
			\centering\(f_3(x)=\tfrac{1}{3}x+2\qquad g_3(x)=3x-3\)
	\end{minipage}}%
\end{minipage}%

\bigskip

\begin{minipage}{\textwidth}
	\adjustbox{valign=t}{\begin{minipage}{0.5\linewidth}\centering
			\begin{tcolorbox}[width=0.95\textwidth, height=1.2cm, valign=center]\centering
				\textcolor{loestc}{\(m_f\cdot m_g=-1\)}
			\end{tcolorbox}
	\end{minipage}}%
	\adjustbox{valign=t}{\begin{minipage}{0.5\linewidth}\centering
			\begin{tcolorbox}[width=0.95\textwidth, height=1.2cm, valign=center]\centering
				\textcolor{loestc}{\(m_f\neq m_g\textbf{ und }m_f\cdot m_g\neq-1\)}
			\end{tcolorbox}
	\end{minipage}}%
\end{minipage}%

\bigskip

\begin{minipage}{\textwidth}
	\adjustbox{valign=t}{\begin{minipage}{0.5\linewidth}
			\raggedright
			\textcolor{loes}{Die beiden Geraden schneiden sich in einem rechten Winkel. Solche Paare von Geraden stehen orthogonal bzw. normal zueinander.}	
	\end{minipage}}%
	\adjustbox{valign=t}{\begin{minipage}{0.5\linewidth}
			\raggedright
			\textcolor{loes}{Die beiden Geraden sind weder parallel noch orthogonal, d.h. sie haben keine besondere Lage zueinander.}
	\end{minipage}}%
\end{minipage}%
%	\newpage
%	\cohead{\Large\textbf{Schnittstellen und Schnittpunkte}}
\fakesubsection{Schnittstellen und Schnittpunkte}
Erinnerung: In der Mathematik unterscheidet man grundsätzlich zwischen Stellen und Punkten. Stellen sind x-Werte während Punkte einen x-Wert und einen y-Wert haben. Die Schnittpunkte zweier Funktionen $f(x)$ und $g(x)$ sind alle Punkte, in denen sich die Schaubilder schneiden. Um die Schnittstellen zu erhalten, muss man die Funktionen gleichsetzen:
\begin{tcolorbox}\centering
	$\textcolor{loestc}{f(x)=g(x)}$
\end{tcolorbox}
\begin{minipage}{0.49\textwidth}
	\includegraphics[width=.95\textwidth]{\linFkt/pics/schnittpunkt.png}
\end{minipage}
\begin{minipage}{0.49\textwidth}
	Im nebenstehenden Beispiel sind die Schaubilder der Funktionen $\textcolor{red}{f(x)=x+1}$ und $\textcolor{blue}{g(x)=3x-1}$ gezeichnet. Der Schnittpunkt lässt sich wie folgt berechnen:
	\begin{align*}
		\textcolor{red}{f(x)}&=\textcolor{blue}{g(x)}\\
		\textcolor{red}{x+1}&=\textcolor{blue}{3x-1}\ \rvert -3x-1\\
		-2x&=-2\ \rvert \cdot\left(-\tfrac{1}{2}\right)\\
		x&=1
	\end{align*}
\end{minipage}\smallskip\\
Die Schnittstelle ist also $\textcolor{ForestGreen}{x=1}$. Um die y-Koordinate zu erhalten, setzt man $\textcolor{ForestGreen}{x=1}$ entweder in $\textcolor{red}{f(x)}$ oder $\textcolor{blue}{g(x)}$ ein. Zur Demonstration setzen wir die Schnittstelle in beide Funktionen ein:
\begin{align*}
	\textcolor{red}{f(}\textcolor{ForestGreen}{1}\textcolor{red}{)}&=\textcolor{red}{\textcolor{ForestGreen}{1}+1}=\textcolor{YellowOrange}{2}\\
	\textcolor{blue}{g(}\textcolor{ForestGreen}{1}\textcolor{blue}{)}&=\textcolor{blue}{3\cdot \textcolor{ForestGreen}{1}-1}=\textcolor{YellowOrange}{2}
\end{align*}
Der Schnittpunkt liegt also bei $P\left(\textcolor{ForestGreen}{1}\lvert\textcolor{YellowOrange}{2}\right)$.
\begin{Exercise}[title={Bestimme jeweils den Schnittpunkt}, label=schnittpunktA1]\\
	\begin{minipage}{0.5\textwidth}
		\begin{enumerate}[label=\alph*)]
			\item $f_1(x)=x-1$ und $g_1(x)=-x+3$
			\item $f_2(x)=-2x+4$ und $g_2(x)=0,5x-1$
			\item $f_3(x)=\frac{3}{2}x+\frac{1}{2}$ und $g_3(x)=4x$
		\end{enumerate}
	\end{minipage}
	\begin{minipage}{0.5\textwidth}
		\begin{enumerate}[label=\alph*)]
			\setcounter{enumi}{3}
			\item $f_4(x)=\frac{4}{5}x+\frac{2}{5}$ und $g_4(x)=-\frac{2}{5}x$
			\item $f_5(x)=-\frac{2}{3}x-15$ und $g_5(x)=3x-\frac{5}{4}$
			\item $f_6(x)=-\frac{5}{8}x$ und $g_6(x)=-\frac{3}{2}x+\frac{1}{2}$
		\end{enumerate}
	\end{minipage}
\end{Exercise}\vspace{.5cm}
\begin{Answer}[ref=schnittpunktA1]\\
	\begin{minipage}{0.5\textwidth}
		\begin{enumerate}[label=\alph*)]
			\item $P_1\left(2\vert 1\right)$
			\item $P_2\left(2\vert 0\right)$
			\item $P_3\left(\frac{1}{5}\vert \frac{4}{5}\right)$
		\end{enumerate}
	\end{minipage}
	\begin{minipage}{0.5\textwidth}
		\begin{enumerate}[label=\alph*)]
			\setcounter{enumi}{3}
			\item $P_4\left(-\frac{1}{3}\vert -\frac{2}{15}\right)$
			\item $P_5\left(-\frac{15}{4}\vert -\frac{25}{2}\right)$
			\item $P_6\left(\frac{4}{7}\vert -\frac{5}{14}\right)$
		\end{enumerate}
	\end{minipage}
\end{Answer}
%	\newpage
%	\cohead{\Large\textbf{Lösungen}}
%	\fakesubsection{Lösungen}
%	\shipoutAnswer
%	\newpage
%	%%%%%%%%%%%%%%%%%%%%%%%%%%%%%%%%%%%%%%%%%%%%%%%%%%%%%%%%%%%%%%%%%%%%%%%%%
%	\fakesection{Quadratische Funktionen}
%	\input{\quadFkt/Scheitelform.tex}
%	\newpage
%	\input{\quadFkt/Hauptform.tex}
%	\newpage
%	\input{\quadFkt/Produktform.tex}
%	\newpage
%	\cohead{\Large\textbf{Lösungen}}
%	\fakesubsection{Lösungen}
%	\shipoutAnswer
%	\newpage
%	%%%%%%%%%%%%%%%%%%%%%%%%%%%%%%%%%%%%%%%%%%%%%%%%%%%%%%%%%%%%%%%%%%%%%%%%%
%	\fakesection{Ganzrationale Funktionen}
%	\input{\ganzFkt/Einfuehrung.tex}
%	\newpage
%	\cohead{\Large\textbf{Potenzfunktionen}}
\fakesubsection{Potenzfunktionen}
Funktionen vom Typ
\[f(x)=a\cdot x^n, \quad a\neq 0, \quad n\in\N\]
bezeichnen wir als Potenzfunktionen.\\
Der Koeffizient \(a\) ist der Streckfaktor, wie wir ihn bereits von quadratischen Funktionen kennen.\\
Die Hochzahl bzw. der Exponent \(n\) ist eine natürliche Zahl: \(\N=\{1,\,2,\,3,\,4,\,\dots\}\)\\
Die Schaubilder der Potenzfunktionen teilen sich in drei verschiedene Formen auf:\vspace{0.3cm}\\
\begin{tabular}{cc}
	\begin{minipage}{0.6\textwidth}
		\centering\Large\textcolor{loes}{Für \(n=1\) ergibt sich eine Gerade.}
	\end{minipage}
	&
	\begin{minipage}{0.39\textwidth}
		\includegraphics[width=.95\linewidth]{\ganzFkt/pics/potenzGerade.png}
	\end{minipage} \\
	\midrule
	\begin{minipage}{0.6\textwidth}
		\centering\Large\textcolor{loes}{Gerade Hochzahlen: \(x^2,\ x^4,\ x^6,\ \dots\)\\
			Parabelförmig\\
			Achsensymmetrie zur y-Achse\\
			\(f(x)\xrightarrow{\hphantom{\ }x\to-\infty\hphantom{\ }}\infty\)\\
			\(f(x)\xrightarrow{\hphantom{\ }x\to\infty\hphantom{\ }}\infty\)
		}
	\end{minipage}
	&
	\begin{minipage}{0.39\textwidth}
		\includegraphics[width=.95\linewidth]{\ganzFkt/pics/potenzGeradeHZ.png}
	\end{minipage} \\
	\midrule
	\begin{minipage}{0.6\textwidth}
		\centering\Large\textcolor{loes}{Ungerade Hochzahlen (größer 1): \(x^3,\ x^5,\ x^7,\ \dots\)\\
			S-förmig\\
			Punktsymmetrie zum Ursprung\\
			\(f(x)\xrightarrow{\hphantom{\ }x\to-\infty\hphantom{\ }}-\infty\)\\
			\(f(x)\xrightarrow{\hphantom{\ }x\to\infty\hphantom{\ }}\infty\)
		}
	\end{minipage}
	&
	\begin{minipage}{0.39\textwidth}
		\includegraphics[width=.95\linewidth]{\ganzFkt/pics/potenzUngeradeHZ.png}
	\end{minipage} \\
\end{tabular}
\newpage
%%%%%%%%%%%%%%%%%%%%%%%%%%%%%%%%%%%%%%%%%%%%%%%%%%%%%%%%%%%%%%%%%%%%%%%%%%%%%%%%%%%%%%%%%%%%%%%%%%%%%%%%%%%%%%%%%%%%%
\begin{Exercise}[title={Skizziere das Schaubild, gib die Symmetrie sowie das Verhalten für sehr große/kleine \(x\) an.}, label=potenzA1]\\
	\begin{minipage}{\textwidth}
		\begin{minipage}{0.49\textwidth}
			\begin{enumerate}[label=\alph*)]
				\item \(f(x)=-x^2\)
				\item \(g(x)=0,5x^3\)
				\item \(h(x)=2x^6\)
				\item \(i(x)=-\frac{3}{2}x^5\)
			\end{enumerate}
		\end{minipage}
		\begin{minipage}{0.49\textwidth}
			\begin{enumerate}[label=\alph*)]
				\setcounter{enumi}{4}
				\item \(j(x)=0,1x^4\)
				\item \(k(x)=-\frac{3}{5}x^7\)
				\item \(l(x)=-\sqrt{2}x^4\)
				\item \(m(x)=3x^5\)
			\end{enumerate}
		\end{minipage}
	\end{minipage}
\end{Exercise}
\newpage
%%%%%%%%%%%%%%%%%%%%%%%%%%%%%%%%%%%%%%%%%
\begin{Answer}[ref=potenzA1]\\
	Man muss nur das Vorzeichen des Streckfaktors \(a\) beachten sowie ob die Hochzahl gerade oder ungerade ist:\vspace{0.5cm}\\
	\begin{tabular}{c|c}
		\begin{minipage}{0.49\textwidth}\centering
			\(a\) positiv und \(n\) gerade wie \(h(x)\) und \(j(x)\)\\
			Parabelförmig\\
			Achsensymmetrie zur y-Achse\\
			\(f(x)\xrightarrow{\hphantom{\ }x\to-\infty\hphantom{\ }}\infty\)\\
			\(f(x)\xrightarrow{\hphantom{\ }x\to\infty\hphantom{\ }}\infty\)\\
			\includegraphics[width=.95\linewidth]{\ganzFkt/pics/potenzPosGerA1.png}\vspace{0.5cm}
		\end{minipage}&
		\begin{minipage}{0.49\textwidth}\centering
			\(a\) negativ und \(n\) gerade wie \(f(x)\) und \(l(x)\)\\
			Parabelförmig\\
			Achsensymmetrie zur y-Achse\\
			\(f(x)\xrightarrow{\hphantom{\ }x\to-\infty\hphantom{\ }}-\infty\)\\
			\(f(x)\xrightarrow{\hphantom{\ }x\to\infty\hphantom{\ }}-\infty\)\\
			\includegraphics[width=.95\linewidth]{\ganzFkt/pics/potenzNegGerA1.png}\vspace{0.5cm}
		\end{minipage}\\ \hline
		\begin{minipage}{0.49\textwidth}\centering\vspace{0.5cm}
			\(a\) positiv und \(n\) ungerade wie \(g(x)\) und \(m(x)\)\\
			S-förmig\\
			Punktsymmetrie zum Ursprung\\
			\(f(x)\xrightarrow{\hphantom{\ }x\to-\infty\hphantom{\ }}-\infty\)\\
			\(f(x)\xrightarrow{\hphantom{\ }x\to\infty\hphantom{\ }}\infty\)\\
			\includegraphics[width=.95\linewidth]{\ganzFkt/pics/potenzPosUngerA1.png}
		\end{minipage}&
		\begin{minipage}{0.49\textwidth}\centering\vspace{0.5cm}
			\(a\) negativ und \(n\) ungerade wie \(i(x)\) und \(k(x)\)\\
			S-förmig\\
			Punktsymmetrie zum Ursprung\\
			\(f(x)\xrightarrow{\hphantom{\ }x\to-\infty\hphantom{\ }}\infty\)\\
			\(f(x)\xrightarrow{\hphantom{\ }x\to\infty\hphantom{\ }}-\infty\)\\
			\includegraphics[width=.95\linewidth]{\ganzFkt/pics/potenzNegUngerA1.png}
		\end{minipage}
	\end{tabular}
\end{Answer}
%	\newpage
%	\input{\ganzFkt/Hauptform.tex}
%	\newpage
%	\input{\ganzFkt/Symmetrie.tex}
%	\newpage
%	\input{\ganzFkt/Verhalten.tex}
%	\newpage
%	\input{\ganzFkt/Nullstellen.tex}
%	\newpage
%	\input{\ganzFkt/Wurzeln.tex}
%	\newpage
%	\input{\ganzFkt/Produktform.tex}
%	\newpage
%	\cohead{\Large\textbf{Lösungen}}
%	\fakesubsection{Lösungen}
%	\shipoutAnswer
%	\newpage
%	%%%%%%%%%%%%%%%%%%%%%%%%%%%%%%%%%%%%%%%%%%%%%%%%%%%%%%%%%%%%%%%%%%%%%%%%%
%	\fakesection{Exponentialfunktionen}
%	\input{\eFkt/Einfuehrung.tex}
%	\newpage
% 	\input{\eFkt/Begriffe.tex}
%	\newpage
% 	\input{\eFkt/WaagrechteAsymptoten.tex}
% 	\newpage
% 	\input{\eFkt/Ln.tex}
% 	\newpage
% 	\input{\eFkt/FktAuf.tex}
% 	\newpage
% 	\input{\eFkt/SchiefeAsymptoten.tex}
% 	\newpage
% 	\input{\eFkt/NaehrungsweiseNST.tex}
% 	\newpage
% 	\cohead{\Large\textbf{Lösungen}}
% 	\fakesubsection{Lösungen}
% 	\shipoutAnswer
% 	\newpage
 	%%%%%%%%%%%%%%%%%%%%%%%%%%%%%%%%%%%%%%%%%%%%%%%%%%%%%%%%%%%%%%%%%%%%%%%%%
% 	\input{\ableitung/MittlereAenderungsrate.tex}
% 	\newpage
% 	\input{\ableitung/GrafischesAbleitung.tex}
% 	\newpage
% 	\input{\ableitung/MomentanPunktweise.tex}
% 	\newpage
% 	\input{\ableitung/MomentanAllg.tex}
% 	\newpage
% 	\input{\ableitung/Ableitungsregeln.tex}
% 	\newpage
% 	\cohead{\Large\textbf{Faktorregel}}
\fakesubsection{Faktorregel}
Wir kennen nun die Ableitung der Normalparabel. Es scheint naheliegend, dass die Ableitungen von \(2x^2,\ 3x^2,\ -x^2\) und ähnlichen Funktionen eine zu \(x^2\) ähnliche Ableitung haben. Wir berechnen die Ableitung von \(f_a(x)=ax^2\):\\
\begin{minipage}[t]{\textwidth}
	\begin{minipage}{0.4\textwidth}
		\begin{align*}
			f_a(x)&=ax^2\\
			f_a'(x)&=\textcolor{loes}{\lim\limits_{h\to 0}\frac{f_a(x+h)-f_a(x)}{h}}\\
			&\textcolor{loes}{=\lim\limits_{h\to 0}\frac{a(x+h)^2-ax^2}{h}}\\
			&\textcolor{loes}{=\lim\limits_{h\to 0}\frac{ax^2+a2hx+ah^2-ax^2}{h}}\\
			&\textcolor{loes}{=\lim\limits_{h\to 0}\frac{a2hx+ah^2}{h}}\\
			&\textcolor{loes}{=\lim\limits_{h\to 0}a2x+ah}\\
			&\textcolor{loes}{=a2x}\\
		\end{align*}
	\end{minipage}
	\begin{minipage}{0.6\textwidth}
		\textcolor{loes}{\(f(x)=x^2\) und \(f_a(x)=ax^2\) unterscheiden sich nur durch den Faktor \(a\). Das gleiche Verhalten zeigen die Ableitungen. \(f'(x)=2x\) und \(f_a'(x)=a2x\) unterscheiden sich ebenfalls nur durch den Faktor \(a\).}
	\end{minipage}
\end{minipage}\\
Tatsächlich lässt sich ganz allgemein zeigen, dass die Ableitung einer Funktion und einem Faktor \(af(x)\) gleich dem gleichen Faktor mal der Ableitung ist: \((af(x))'=af'(x)\):\\
\begin{minipage}[t]{\textwidth}
	\begin{minipage}{0.5\textwidth}
		\begin{align*}
			g(x)&=af(x)\\
			g'(x)&=\textcolor{loes}{\lim\limits_{h\to 0}\frac{af(x+h)-af(x)}{h}}\\
			&\textcolor{loes}{=a\lim\limits_{h\to 0}\frac{f(x+h)-f(x)}{h}}\\
			&\textcolor{loes}{=af'(x)}\\
		\end{align*}
	\end{minipage}
	\begin{minipage}{0.5\textwidth}
		\begin{tcolorbox}
			\phantom{text}\\
			\textcolor{loestc}{Faktorregel:\\
				Die Ableitung von \(af(x)\) ist \(af'(x)\).}\\
		\end{tcolorbox}
	\end{minipage}
\end{minipage}\\
% 	\newpage
% 	\input{\ableitung/Potenzregel.tex}
% 	\newpage
% 	\input{\ableitung/Summenregel.tex}
% 	\newpage
% 	\input{\ableitung/eFktAbleitung.tex}
% 	\newpage
% 	\input{\ableitung/SchaubilderDerAbleitung.tex}
% 	\newpage
% 	\input{\ableitung/Tangenten.tex}
% 	\newpage
% 	\input{\ableitung/Normalen.tex}
% 	\newpage
% 	\input{\ableitung/Beruehren.tex}
% 	\newpage
% 	\input{\ableitung/Monotonie.tex}
% 	\newpage
% 	\input{\ableitung/HoehereAbleitungen.tex}
% 	\newpage
% 	\input{\ableitung/Extrempunkte.tex}
% 	\newpage
% 	\input{\ableitung/Kruemmung.tex}
% 	\newpage
% 	\cohead{\Large\textbf{Wendepunkte}}
\fakesubsection{Wendepunkte}
Als Wendepunkte bezeichnet man die Punkte einer Funktion an denen die Krümmung wechselt, also die Funktion von einer Linkskurve in eine Rechtskurve übergeht oder umgekehrt. Oder anders ausgedrückt, eine Funktion hat genau dann einen Wendepunkt, wenn die zweite Ableitung eine Nullstelle mit Vorzeichenwechsel hat.\\
Die folgenden drei Aussagen sind äquivalent:

\bigskip

\begin{minipage}{\textwidth}
	\adjustbox{valign=t}{\begin{minipage}{\textwidth/\real{3}-1ex}
		\begin{tcolorbox}[width=\textwidth, height=4cm, valign=center]
			\textcolor{loestc}{Das Schaubild von \(f(x)\) hat bei \(x_0\) einen Wendepunkt.}
		\end{tcolorbox}\end{minipage}}%
	\adjustbox{valign=t, padding=1.5ex 0ex 0ex 0ex}{\begin{minipage}{\textwidth/\real{3}-1ex}
		\begin{tcolorbox}[width=\textwidth, height=4cm, valign=center]
			\textcolor{loestc}{Das Schaubild von \(f'(x)\) hat bei \(x_0\) einen Extrempunkt.}
		\end{tcolorbox}\end{minipage}}%
	\adjustbox{valign=t, padding=1.5ex 0ex 0ex 0ex}{\begin{minipage}{\textwidth/\real{3}-1ex}
		\begin{tcolorbox}[width=\textwidth, height=4cm, valign=center]
			\textcolor{loestc}{Das Schaubild von \(f''(x)\) hat bei \(x_0\) eine Nullstelle mit Vorzeichenwechsel.}
		\end{tcolorbox}\end{minipage}}%
\end{minipage}

\bigskip

Beispiel:

\bigskip

\begin{minipage}{\textwidth}
	\adjustbox{valign=t}{\begin{minipage}{\textwidth/\real{3}-1ex}
			\centering\includegraphics[width=\textwidth]{\ableitung/pics/WP_funktion.png}

			\(f(x)=\frac{1}{6}x^3-\frac{1}{2}x^2-\frac{3}{2}x+1\)

			\textcolor{loes}{Das Schaubild von \(f(x\) hat bei \(x_0\)=1 einen Wendepunkt.}
	\end{minipage}}%
	\adjustbox{valign=t, padding=1.5ex 0ex 0ex 0ex}{\begin{minipage}{\textwidth/\real{3}-1ex}
			\centering\includegraphics[width=\textwidth]{\ableitung/pics/WP_ableitung.png}

			\(f'(x)=\frac{1}{2}x^2-x-\frac{3}{2}\)

			\textcolor{loes}{Das Schaubild von \(f'(x\) hat bei \(x_0\)=1 einen Extrempunkt.}
	\end{minipage}}%
	\adjustbox{valign=t, padding=1.5ex 0ex 0ex 0ex}{\begin{minipage}{\textwidth/\real{3}-1ex}
			\centering\includegraphics[width=\textwidth]{\ableitung/pics/WP_zwableitung.png}

			\(f''(x)=x-1\)

			\textcolor{loes}{Das Schaubild von \(f''(x\) hat bei \(x_0\)=1 eine Nullstelle mit Vorzeichenwechsel.}
	\end{minipage}}%
\end{minipage}
\newpage
%%%%%%%%%%%%%%%%%%%%%%%%%%%%%%%%%%%%%%%%%%%%%%%%%%%%%%%%%%%%%%%%%%%%%%%%%%%%%%%%%%%%%%%%%%%%%%%%%%%%%%%
\begin{Exercise}[title={\raggedright Bestimme die Wendepunkte.}, label=wendepunkteA1]
	\begin{enumerate}[label=\alph*)]
		\item \(f_1(x)=x^3+3x^2+3x-9\)
		\item \(f_2(x)=0,5x^3-1,5x^2+5\)
		\item \(f_3(x)=-\frac{2}{3}x^3+4x^2-4x-3\)
		\item \(f_4(x)=-\frac{1}{36}x^3+\frac{7}{24}x^2-8x-\frac{55}{144}\)
		\item \(f_5(x)=\frac{5}{12}x^4+\frac{5}{6}x^3-5x^2+x-3\)
		\item \(f_6(x)=-\frac{1}{2}x^4-x^3+\frac{45}{4}x^2+2x-\frac{13}{32}\)
		\item \(f_7(x)=2x^4+16x^3-128x\)
	\end{enumerate}
\end{Exercise}
%%%%%%%%%%%%%%%%%%%%%%%%%%%%%%%%%%%%%%%%%
\begin{Answer}[ref=wendepunkteA1]
	\begin{enumerate}[label=\alph*)]
		\item \(f_1(x):\ W\left(-1\vert -10\right)\)
		\item \(f_2(x):\ W\left(1\vert 4\right)\)
		\item \(f_3(x):\ W\left(2\vert -\frac{1}{3}\right)\)
		\item \(f_4(x):\ W\left(\frac{7}{2}\vert -26\right)\)
		\item \(f_5(x):\ W_1\left(-2\vert -25\right),\ W_2\left(1\vert -\frac{23}{4}\right)\)
		\item \(f_6(x):\ W_1\left(-\frac{5}{2}\vert 61\right),\ W_2\left(\frac{3}{2}\vert 22\right)\)
		\item \(f_7(x):\ W_1\left(0\vert 0\right),\ W_2\left(-4\vert 0\right)\)
	\end{enumerate}
\end{Answer}
% 	\newpage
% 	\cohead{\Large\textbf{Aufstellen von Funktionsgleichungen}}
\fakesubsection{Aufstellen von Funktionsgleichungen}
\begin{tabular}{p{.3\textwidth}|p{.65\textwidth}}
	\textbf{Angabe}&\textbf{Gleichung}\\
	\hline
	\textbf{Punkt}\newline (\(x\)-Wert und \(y\)-Wert)\newline\newline&\textcolor{loes}{Punkte werden in die normale Funktionsgleichung eingesetzt (Punktprobe):  \(f(x_P)=y_P\)}\\
	\hline
	\textbf{Extrempunkt oder\newline Sattelpunkt} bei \(x_0\)\newline\newline&\textcolor{loes}{Die erste Ableitung muss Null ergeben: \(f'(x_0)=0\)}\\
	\hline
	\textbf{Wendestelle} an der Stelle \(x_0\)\newline\newline\newline&\textcolor{loes}{An einer Wendestelle muss die zweite Ableitung Null sein: \(f''(x_0)=0\)\newline Hinweis: Jeder Sattelpunkt ist auch ein Wendepunkt.}\\
	\hline
	\textbf{Steigung \(m\)} an der Stelle \(x_0\)\newline\newline\newline&\textcolor{loes}{Die Steigung entspricht der Ableitung: \(f'(x_0)=m\)}\\
	\hline
	\textbf{Nullstellen}&\textcolor{loes}{Punktprobe: \(f(x_0)=0\)\newline Sind bei einer ganzrationalen Funktionen so viele Nullstellen wie der Grad gegeben, so empfiehlt sich die Produktform als Ansatz.}\newline\newline\\
	\hline
	\textbf{Symmetrieeigenschaften:}\newline 1. Achsensymmetrie zur\newline y-Achse\newline\newline 2. Punktsymmetrie zum\newline Ursprung& \phantom{x}\newline\textcolor{loes}{1. Bei einer ganzrationalen Funktion müssen alle Hochzahlen gerade oder Null sein.\newline\newline 2. Bei einer ganzrationalen Funktion müssen alle Hochzahlen ungerade sein.}\newline\newline\\
	\hline
	\textbf{Asymptote} \(b\) bzw.\newline\(mx+b\) für eine waagrechte oder schiefe Asymptote&\textcolor{loes}{Die Asymptote kann bei Exponentialfunktionen direkt in die Funktionsgleichung eingesetzt werden: \(f(x)=ae^{kx}+mx+b\)}\newline\newline\newline\newline
\end{tabular}
\newpage
%%%%%%%%%%%%%%%%%%%%%%%%%%%%%%%%%%%%%%%%%%%%%%%%%%%%%%%%%%%%%%%%%%%%%%%%%%%%%%%%%%%%%%%%%%%%%%%%%%%%%%%
\begin{Exercise}[title={\raggedright Bestimme jeweils die Funktionsgleichung.}, label=fktbestimmenA1]
	\begin{enumerate}[label=\alph*)]
		\item Das Schaubild der ganzrationalen Funktion \(f_1(x)\) vierten Grades hat den y-Achsenabschnitt 3, ist achsensymmetrisch zur y-Achse hat bei \(H(2|4)\) einen Hochpunkt.
		\item Das Schaubild der ganzrationalen Funktion \(f_2(x)\) dritten Grades berührt die x-Achse bei \(x=3\), schneidet die x-Achse bei \(x=-1\) und verläuft durch den Punkt \(P(1|4)\)
		\item Das Schaubild der ganzrationalen Funktion \(f_3(x)\) dritten Grades ist punktsymmetrisch zum Ursprung und hat den Tiefpunkt \(T(2|-8)\).
		\item Das Schaubild der ganzrationalen Funktion \(f_4(x)\) dritten Grades hat im Wendepunkt \newline\(W(0|-1)\) die Steigung -2 und eine Nullstelle bei \(x_0=3\).
%		\item \(f_5(x)\)
%		\item \(f_6(x)\)
%		\item \(f_7(x)\)
	\end{enumerate}
\end{Exercise}
%%%%%%%%%%%%%%%%%%%%%%%%%%%%%%%%%%%%%%%%%
\begin{Answer}[ref=fktbestimmenA1]
	\begin{enumerate}[label=\alph*)]
		\item \(f_1(x)=-\frac{1}{16}x^4+\frac{1}{2}x^2+3\)
		\item \(f_2(x)=0,5(x+1)(x-3)^2\)
		\item \(f_3(x)=0,5x^3-6x\)
		\item \(f_4(x)=\frac{7}{27}x^3-2x-1\)
%		\item \(f_5(x)=\)
%		\item \(f_6(x)=\)
%		\item \(f_7(x)=\)
	\end{enumerate}
\end{Answer}

 %Aufgaben erstellen
% 	\newpage
%%%%%%%%%%%%%%%%%%%%%%%%%%%%%%%%%%%%%%%%%%%%%%%%%%%%%%%%%%%%%%%%%%%%%%%%%
% 	\input{\lgs/xxxxxxxxxxx.tex}
% 	\newpage
%%%%%%%%%%%%%%%%%%%%%%%%%%%%%%%%%%%%%%%%%%%%%%%%%%%%%%%%%%%%%%%%%%%%%%%%%
% 	\input{\optimierung/Optimierung.tex}
% 	\newpage
%%%%%%%%%%%%%%%%%%%%%%%%%%%%%%%%%%%%%%%%%%%%%%%%%%%%%%%%%%%%%%%%%%%%%%%%%
% 	\input{\integration/Stammfunktionen.tex}
% 	\newpage
% 	\input{\integration/Integral_grafisch.tex}
% 	\newpage
% 	\input{\integration/Integral_berechnen.tex}
% 	\newpage
% 	\input{\integration/Flaechenberechnung.tex}
% 	\newpage
% 	\cohead{\Large\textbf{Integrationsgrenze bestimmen}}
\fakesubsection{Bestimmen von Integrationsgrenzen}
Eine des Öfteren vorkommende Aufgabenstellung beinhaltet das grafische Abschätzen oder das Berechnen einer Integrationsgrenze. Bei diesem Aufgabentyp ist in der Regel die obere Integrationsgrenze gesucht.\\
Beispiel: Gegeben ist das Schaubild der Funktion \(f(x)\). Gesucht sind zwei Lösungen \(u>0\), so dass \(\int_0^u f(x) \td x=0\) gilt.\\ \\
\begin{minipage}{\textwidth}
	\adjustbox{valign=t}{\begin{minipage}{.5\textwidth}\raggedright
		Grafische Abschätzung:\\
		\textcolor{loes}{Damit der Wert des Integrals Null ergibt, müssen die Teilflächen oberhalb der \(x\)-Achse und unterhalb der \(x\)-Achse gleich groß sein. Dies ist für \(x_1\approx 1\) und \(x_2\approx 4\) der Fall, wie man sehen kann, wenn man die entsprechenden Flächen im Schaubild markiert.}\\
	\end{minipage}}
	\adjustbox{valign=t}{\begin{minipage}{.5\textwidth}
		\includegraphics[width=.95\linewidth]{\integration/pics/grenzeBsp1.png}
	\end{minipage}}
\end{minipage}\vspace{\baselineskip}\\
Ist die Funktionsgleichung bekannt, so kann man die Lösung auch berechnen. Im Beispiel ist \(f(x)=x^2-\frac{10}{3}x+\frac{4}{3}\):\\
\textcolor{loes}{Wir lösen die oben gegebene Gleichung:}
\begin{align*}
	\textcolor{loes}{\int_0^u f(x) \td x}&\textcolor{loes}{=0}\\
	\textcolor{loes}{\int_0^u x^2-\frac{10}{3}x+\frac{4}{3} \td x}&\textcolor{loes}{=0}\\
	\textcolor{loes}{\left[\frac{1}{3}x^3-\frac{5}{3}x^2+\frac{4}{3}x \right]_0^u}&\textcolor{loes}{=0}\\
	\textcolor{loes}{\frac{1}{3}u^3-\frac{5}{3}u^2+\frac{4}{3}u}&\textcolor{loes}{=0}\\
	\textcolor{loes}{u\left(\frac{1}{3}u^2-\frac{5}{3}u+\frac{4}{3}\right)}&\textcolor{loes}{=0}\\	
	\textcolor{loes}{\text{S.v.N.: }u_0=0\text{ oder }\frac{1}{3}u^2-\frac{5}{3}u+\frac{4}{3}}&\textcolor{loes}{=0}
\end{align*}
\textcolor{loes}{Die Mitternachtsformel liefert \(u_1=1\) und \(u_2=4\). Da laut Aufgabenstellung \(u>0\) sein soll, sind nur \(u_1\) und \(u_2\) zulässige Lösungen.}
\newpage

\begin{Exercise}[title={\raggedright\normalfont Schätze jeweils ein \(u\) graphisch ab und berechne dann die Lösung. Das gesuchte \(u\) soll immer größer als die untere Grenze sein:}, label=intGrenzeA1]\\
	\begin{minipage}{\textwidth}
		\begin{minipage}{.5\textwidth}
			\includegraphics[width=.8\linewidth]{\integration/pics/intGrenze1.png}
			\[\int_0^u-\tfrac{1}{2}x+\tfrac{1}{2} \td x=-1\]
		\end{minipage}
		\begin{minipage}{.5\textwidth}
			\includegraphics[width=.8\linewidth]{\integration/pics/intGrenze2.png}
			\[\int_{-1}^u-0,5x^2 \td x=-1,5\]
		\end{minipage}\vspace{\baselineskip}\\
		\begin{minipage}{.5\textwidth}
			\includegraphics[width=.8\linewidth]{\integration/pics/intGrenze3.png}
			\[\int_0^u-\tfrac{1}{20}x^3+\tfrac{6}{5}x \td x=2,2\]
		\end{minipage}
		\begin{minipage}{.5\textwidth}
			\includegraphics[width=.8\linewidth]{\integration/pics/intGrenze4.png}
			\[\int_0^u e^{0,5x} \td x=4\]
		\end{minipage}\vspace{\baselineskip}\\
		\begin{minipage}{.5\textwidth}
			\includegraphics[width=.8\linewidth]{\integration/pics/intGrenze5.png}
			\[\int_{-1}^u 2e^{-\tfrac{1}{3}x} \td x=5\]
			\phantom{Finde 2 verschiedene \(u>0\).}
		\end{minipage}
		\begin{minipage}{.5\textwidth}
			\includegraphics[width=.8\linewidth]{\integration/pics/intGrenze6.png}
			\[\int_0^u \tfrac{1}{2}x^3-\tfrac{3}{2}x^2+\tfrac{3}{4}x \td x=0\]
			\centering{Finde 2 verschiedene \(u>0\).}
		\end{minipage}
	\end{minipage}
\end{Exercise}

%%%%%%%%%%%%%%%%%%%%%%%%%%%%%%%%%%%%%%%%%
\begin{Answer}[ref=intGrenzeA1]\\
	Die grafisch abgeschätzten Lösungen sollten jeweils bis auf \(\pm0,3\) mit den berechneten Lösungen übereinstimmen. Lösungen in Klammer lösen zwar die Gleichung, sind aber nicht größer als die untere Grenze.\\
	\begin{minipage}{\textwidth}
		\begin{minipage}{.5\textwidth}\raggedright
			\[\int_0^u-\frac{1}{2}x+\frac{1}{2} \td x=-1\]
			\[u_1=1+\sqrt{5}\approx3,24\ \left(\text{und }u_2=1-\sqrt{5}\right)\] 
		\end{minipage}
		\begin{minipage}{.5\textwidth}
			\[\int_{-1}^u-0,5x^2 \td x=-1,5\]
			\[u_1=2\]
		\end{minipage}\vspace{\baselineskip}\\\vspace{\baselineskip}\\
		\begin{minipage}{.5\textwidth}\raggedright
			\[\int_0^u-\tfrac{1}{20}x^3+\tfrac{6}{5}x \td x=2,2\]
			\[u_1=2, u_2=\sqrt{44}\ \left(\text{und }u_3=-2,\ u_4=-\sqrt{44}\right)\]
		\end{minipage}
		\begin{minipage}{.5\textwidth}
			\[\int_0^u e^{0,5x} \td x=4\]
			\[u_1=2\ln(3)\]
		\end{minipage}\vspace{\baselineskip}\\\vspace{\baselineskip}\\
		\begin{minipage}{.5\textwidth}\raggedright
			\[\int_{-1}^u 2e^{-\tfrac{1}{3}x} \td x=5\]
			\[u_1=-3\ln\left(e^{\tfrac{1}{3}}-\frac{5}{6}\right)\]
		\end{minipage}
		\begin{minipage}{.5\textwidth}
			\[\int_0^u \frac{1}{2}x^3-\frac{3}{2}x^2+\tfrac{3}{4}x \td x=0\]
			\[u_1=1,\ u_2=3\ \left(\text{und }u_3=0\right)\] 
		\end{minipage}
	\end{minipage}
\end{Answer}
% 	\newpage
% 	\input{\integration/FlaechZwischenFunktionen.tex}
% 	\newpage
% 	\input{\integration/VerhaeltnisVonFlaechen.tex}
% 	\newpage
%%%%%%%%%%%%%%%%%%%%%%%%%%%%%%%%%%%%%%%%%%%%%%%%%%%%%%%%%%%%%%%%%%%%%%%%%
% 	\input{\trigonometrie/Bogenmass.tex}
% 	\newpage
% 	\cohead{\Large\textbf{Sinus-/Cosinusfunktion}}
\fakesubsection{Definition der Sinus- und Cosinusfunktion}
Wie beim Bogenmaß beginnen wir beim Punkt \(P(1\vert 0)\) mit dem Bogenmaß \(b=0\) bzw. \(\alpha=0^\circ\) im Gradmaß.\\
\begin{minipage}{\textwidth}
	\begin{minipage}{.26\textwidth}
		\includegraphics[width=.95\linewidth]{\trigonometrie/pics/Einheitskreis.png}
	\end{minipage}
	\begin{minipage}{.74\textwidth}
		\includegraphics[width=.95\linewidth]{\trigonometrie/pics/leeresKoordinatensystem.png}
	\end{minipage}
\end{minipage}\\
\begin{tcolorbox}
	\textbf{Definition der Sinusfunktion}\\
	\textcolor{loestc}{Zu jedem Winkel \(b\) kann man genau einen Punkt \(P(x\vert y)\) auf dem Einheitskreis zuordnen. Der Sinus von \(b\) ist dann die \(y\)-Koordinate dieses Punkts:
		\[\sin\left(b\right)=y\]		
	}
\end{tcolorbox}
Analog kann man die Cosinusfunktion definieren:\\
\begin{minipage}{\textwidth}
	\begin{minipage}{.26\textwidth}
		\includegraphics[width=.95\linewidth]{\trigonometrie/pics/Einheitskreis.png}
	\end{minipage}
	\begin{minipage}{.74\textwidth}
		\includegraphics[width=.95\linewidth]{\trigonometrie/pics/leeresKoordinatensystem.png}
	\end{minipage}
\end{minipage}\\
\begin{tcolorbox}
	\textbf{Definition der Cosinusfunktion}\\
	\textcolor{loestc}{Zu jedem Winkel \(b\) kann man genau einen Punkt \(P(x\vert y)\) auf dem Einheitskreis zuordnen. Der Cosinus von \(b\) ist dann die \(x\)-Koordinate dieses Punkts:
		\[\cos\left(b\right)=x\]		
	}
\end{tcolorbox}
\textbf{Periode:}\\
\textcolor{loes}{Die Sinus- und Cosinusfunktion sind beide periodische Funktionen, d.h. sie wiederholen sich nach einer bestimmten Länge unendlich oft. Nach einer Umdrehung auf dem Einheitskreis beginnen beide Funktionen wieder von vorne. Durch Rückwärtsdrehen kann man die beiden Funktionen auch für negative Windel definieren. Die kleinstmögliche Länge auf der \(x\)-Achse, nach der sich eine periodische Funktion wiederholt, wird als Periode \(p\) der Funktion bezeichnet. Für \(\sin x\) und \(\cos x\) ist die Periode jeweils \(p=2\pi\).}
% 	\newpage
% 	\input{\trigonometrie/EigenschaftenSinCos.tex}
% 	\newpage
% 	\cohead{\Large\textbf{Allgemeine Sinus-/Cosinusfunktion}}
\fakesubsection{Allgemeine Sinus- und Cosinusfunktion}
Die allgemeine Sinusfunktion bzw. Cosinusfunktion sind gegeben durch:
\[f(x)=a\sin\left(bx\right)+d\qquad g(x)=a\cos\left(bx\right)+d\]
\begin{itemize}
	\item \(a\): \textcolor{loes}{Streckfaktor relativ zum Mittelwert in \(y\)-Richtung. Der Betrag von \(a\) gibt die Amplitude an.}\\
	\item \(d\): \textcolor{loes}{Verschiebung in \(y\)-Richtung. Da der Mittelwert zuvor bei \(0\) lag, entspricht \(d\) dem Mittelwert.}\\
	\item \(b\): \textcolor{loes}{Streckfaktor in \(x\)-Richtung. Zwischen dem Streckfaktor und der Periode \(p\) besteht folgender Zusammenhang: \(b\cdot p=2\pi\).}\\
\end{itemize}
Beispiel: \(\displaystyle f(x)=2\sin\left(0,5x\right)-1\)\\
\begin{minipage}{\textwidth}
	\includegraphics[width=.95\linewidth]{\trigonometrie/pics/AllgSin.png}\\
\end{minipage}\\
Beispiel: \(\displaystyle g(x)=-0,5\cos\left(\pi x\right)+1\)\\
\begin{minipage}{\textwidth}
	\includegraphics[width=.95\linewidth]{\trigonometrie/pics/AllgCos.png}\\
\end{minipage}\\
\newpage
\textbf{Bestimmen der Nullstellen:}\\
Die Nullstellen sind an sich nicht schwer zu bestimmen. Das Problem ist, dass es unendlich viele Nullstellen gibt. Will man alle Nullstellen aufschreiben, so braucht man eine neue Notation. Das gleiche Problem ergibt sich bei den Maxima, Minima und Wendestellen.
\begin{tcolorbox}
	\textbf{Notation für unendlich viele Stellen:}\\
	Erinnerung: Die ganzen Zahlen sind wie folgt definiert: \(\Z=\{0,\ 1,\ -1,\ 2,\ -2,\ 3,\ -3,\dots\}\)
	\textcolor{loestc}{Zur Notation nutzen wir aus, dass sich die Nullstellen/Maxima/Minima/Wendestellen in festen Abständen wiederholen, z.B. wieder beträgt der Abstand von einer Nullstelle zur nächsten bei sowohl \(\sin x\) als auch \(\cos x\)	jeweils \(\pi\):
		\[\text{alle Stellen }=\text{ eine Stelle }+\ k\cdot\text{Abstand der Stellen, }k\in\Z\]
		Bsp.: Nullstellen von \(\sin x\): \(x_k=0+k\cdot \pi,\ k\in\Z\)
	}
\end{tcolorbox}

\begin{Exercise}[title={\raggedright\normalfont Gib jeweils die Amplitude, die Periode und den Mittelwert an. Skizziere dann das Schaubild so, dass mindestens eine Periode zu sehen ist.}, label=allgSinCosA1]
	\begin{enumerate}[label=\alph*)]
		\item \(f_1(x)=-2\sin\left(3x\right)-4\)
		\item \(f_2(x)=1,5\sin\left(4x\right)-2\)
		\item \(f_3(x)=-3\cos\left(0,5x\right)+1\)
		\item \(f_4(x)=\cos\left(\frac{1}{3}x\right)\)
		\item \(f_5(x)=-\sin\left(\frac{1}{2}x\right)-1\)
		\item \(f_6(x)=3\sin\left(\frac{2}{3}x\right)+3\)
		\item \(f_7(x)=4\cos\left(\frac{3}{4}x\right)+2\)
		\item \(f_8(x)=2,5\sin\left(\pi x\right)-1\)
		\item \(f_9(x)=-\cos\left(\frac{\pi}{2}x\right)+2\)
		\item \(f_{10}(x)=-4\sin\left(2\pi x\right)-1\)
		\item \(f_{11}(x)=-2,5\cos\left(4\pi x\right)+3,5\)
		\item \(f_{12}(x)=3\cos\left(1,5x\right)+2\)
		\item \(f_{13}(x)=0,5\sin\left(0,5\pi x\right)+1,5\)
		\item \(f_{14}(x)=-5\sin\left(4\pi x\right)-3\)
		\item \(f_{15}(x)=3\cos\left(x\right)+2\)
		\item \(f_{16}(x)=-2\sin\left(2x\right)\)
		\item \(f_{17}(x)=2,5\sin\left(2x\right)-3,5\)
		\item \(f_{18}(x)=3,5\cos\left(\frac{2}{3}\pi x\right)+2\)
		\item \(f_{19}(x)=5\cos\left(x\right)-4\)
		\item \(f_{20}(x)=-6\cos\left(2\pi x\right)+10\)
	\end{enumerate}
\end{Exercise}
\newpage
%%%%%%%%%%%%%%%%%%%%%%%%%%%%%%%%%%%%%%%%%%%%%%%%%%%%%%%%%%%%%%%%%%%%%%%%%%%%%%%%%%%%%%%%%%%%%%%%%%%%%%%%%%%%%%%%%%%%%%%%
\begin{Exercise}[title={Stelle jeweils die Funktionsgleichung vom Typ \(a\cdot \sin\left(bx\right)+d\)\\oder \(a\cdot \cos\left(bx\right)+d\) auf.}, label=allgSinCosA2]\\
	\begin{minipage}{\textwidth}
		\begin{minipage}{0.49\textwidth}
			\begin{enumerate}[label=\alph*)]
				\item \begin{minipage}{.9\textwidth}
					\includegraphics[width=.75\linewidth]{\trigonometrie/pics/AllgSinA2_1.png}\\
				\end{minipage}	
				\item \begin{minipage}{.9\textwidth}
					\includegraphics[width=.75\linewidth]{\trigonometrie/pics/AllgSinA2_2.png}\\
				\end{minipage}
				\item \begin{minipage}{.9\textwidth}
					\includegraphics[width=.75\linewidth]{\trigonometrie/pics/AllgSinA2_3.png}\\
				\end{minipage}
				\item \begin{minipage}{.9\textwidth}
					\includegraphics[width=.75\linewidth]{\trigonometrie/pics/AllgSinA2_4.png}\\
				\end{minipage}
				\item \begin{minipage}{.9\textwidth}
					\includegraphics[width=.75\linewidth]{\trigonometrie/pics/AllgSinA2_5.png}\\
				\end{minipage}
			\end{enumerate}
		\end{minipage}
		\begin{minipage}{0.49\textwidth}
			\begin{enumerate}[label=\alph*)]
				\setcounter{enumi}{5}
				\item \begin{minipage}{.9\textwidth}
					\includegraphics[width=.75\linewidth]{\trigonometrie/pics/AllgSinA2_6.png}\\
				\end{minipage}	
				\item \begin{minipage}{.9\textwidth}
					\includegraphics[width=.75\linewidth]{\trigonometrie/pics/AllgSinA2_7.png}\\
				\end{minipage}
				\item \begin{minipage}{.9\textwidth}
					\includegraphics[width=.75\linewidth]{\trigonometrie/pics/AllgSinA2_8.png}\\
				\end{minipage}
				\item \begin{minipage}{.9\textwidth}
					\includegraphics[width=.75\linewidth]{\trigonometrie/pics/AllgSinA2_9.png}\\
				\end{minipage}
				\item \begin{minipage}{.9\textwidth}
					\includegraphics[width=.75\linewidth]{\trigonometrie/pics/AllgSinA2_10.png}\\
				\end{minipage}
			\end{enumerate}
		\end{minipage}
	\end{minipage}
\end{Exercise}


%%%%%%%%%%%%%%%%%%%%%%%%%%%%%%%%%%%%%%%%%
\begin{Answer}[ref=allgSinCosA1]
	\begin{enumerate}[label=\alph*)]
		\item Amplitude \(a_{1}=2\), Periode \(p_{1}=\frac{2\pi}{3}\), Mittelwert \(d_{1}=-4\)\\
		\begin{minipage}{\textwidth}
			\includegraphics[height=5cm]{\trigonometrie/pics/AllgSinA1_1.png}\\
		\end{minipage}	 
		\item Amplitude \(a_{2}=1,5\), Periode \(p_{2}=\frac{\pi}{2}\), Mittelwert \(d_{2}=-2\)\\
		\begin{minipage}{\textwidth}
			\includegraphics[height=5cm]{\trigonometrie/pics/AllgSinA1_2.png}\\
		\end{minipage}	 
		\item Amplitude \(a_{3}=3\), Periode \(p_{3}=4\pi\), Mittelwert \(d_{3}=1\)\\
		\begin{minipage}{\textwidth}
			\includegraphics[height=5cm]{\trigonometrie/pics/AllgSinA1_3.png}\\
		\end{minipage}	 
		\item Amplitude \(a_{4}=1\), Periode \(p_{4}=6\pi\), Mittelwert \(d_{4}=0\)\\
		\begin{minipage}{\textwidth}
			\includegraphics[height=5cm]{\trigonometrie/pics/AllgSinA1_4.png}\\
		\end{minipage}	 
		\item Amplitude \(a_{5}=1\), Periode \(p_{5}=4\pi\), Mittelwert \(d_{5}=-1\)\\
		\begin{minipage}{\textwidth}
			\includegraphics[height=5cm]{\trigonometrie/pics/AllgSinA1_5.png}\\
		\end{minipage}	 
		\item Amplitude \(a_{6}=3\), Periode \(p_{6}=3\pi\), Mittelwert \(d_{6}=3\)\\
		\begin{minipage}{\textwidth}
			\includegraphics[height=5cm]{\trigonometrie/pics/AllgSinA1_6.png}\\
		\end{minipage}	 
		\item Amplitude \(a_{7}=4\), Periode \(p_{7}=\frac{8\pi}{3}\), Mittelwert \(d_{7}=2\)\\
		\begin{minipage}{\textwidth}
			\includegraphics[height=5cm]{\trigonometrie/pics/AllgSinA1_7.png}\\
		\end{minipage}	 
		\item Amplitude \(a_{8}=2,5\), Periode \(p_{8}=2\), Mittelwert \(d_{8}=-1\)\\
		\begin{minipage}{\textwidth}
			\includegraphics[height=5cm]{\trigonometrie/pics/AllgSinA1_8.png}\\
		\end{minipage}	\newpage
		\item Amplitude \(a_{9}=1\), Periode \(p_{9}=4\), Mittelwert \(d_{9}=2\)\\
		\begin{minipage}{\textwidth}
			\includegraphics[height=5cm]{\trigonometrie/pics/AllgSinA1_9.png}\\
		\end{minipage}	 
		\item Amplitude \(a_{10}=4\), Periode \(p_{10}=1\), Mittelwert \(d_{10}=-1\)\\
		\begin{minipage}{\textwidth}
			\includegraphics[height=5cm]{\trigonometrie/pics/AllgSinA1_10.png}\\
		\end{minipage}	 
		\item Amplitude \(a_{11}=2,5\), Periode \(p_{11}=\frac{1}{2}\), Mittelwert \(d_{11}=3,5\)\\
		\begin{minipage}{\textwidth}
			\includegraphics[height=5cm]{\trigonometrie/pics/AllgSinA1_11.png}\\
		\end{minipage}	 
		\item Amplitude \(a_{12}=3\), Periode \(p_{12}=\frac{4\pi}{3}\), Mittelwert \(d_{12}=2\)\\
		\begin{minipage}{\textwidth}
			\includegraphics[height=5cm]{\trigonometrie/pics/AllgSinA1_12.png}\\
		\end{minipage}	 \newpage
		\item Amplitude \(a_{13}=0,5\), Periode \(p_{13}=4\), Mittelwert \(d_{13}=1,5\)\\
		\begin{minipage}{\textwidth}
			\includegraphics[height=5cm]{\trigonometrie/pics/AllgSinA1_13.png}\\
		\end{minipage}	 
		\item Amplitude \(a_{14}=5\), Periode \(p_{14}=0,5\), Mittelwert \(d_{14}=-3\)\\
		\begin{minipage}{\textwidth}
			\includegraphics[height=5cm]{\trigonometrie/pics/AllgSinA1_14.png}\\
		\end{minipage}	 
		\item Amplitude \(a_{15}=3\), Periode \(p_{15}=2\pi\), Mittelwert \(d_{15}=2\)\\
		\begin{minipage}{\textwidth}
			\includegraphics[height=5cm]{\trigonometrie/pics/AllgSinA1_15.png}\\
		\end{minipage}	 
		\item Amplitude \(a_{16}=2\), Periode \(p_{16}=\pi\), Mittelwert \(d_{16}=0\)\\
		\begin{minipage}{\textwidth}
			\includegraphics[height=5cm]{\trigonometrie/pics/AllgSinA1_16.png}\\
		\end{minipage}	 \newpage
		\item Amplitude \(a_{17}=2,5\), Periode \(p_{17}=\pi\), Mittelwert \(d_{17}=-3,5\)\\
		\begin{minipage}{\textwidth}
			\includegraphics[height=5cm]{\trigonometrie/pics/AllgSinA1_17.png}\\
		\end{minipage}	 
		\item Amplitude \(a_{18}=3,5\), Periode \(p_{18}=3\), Mittelwert \(d_{18}=2\)\\
		\begin{minipage}{\textwidth}
			\includegraphics[height=5cm]{\trigonometrie/pics/AllgSinA1_18.png}\\
		\end{minipage}	 
		\item Amplitude \(a_{19}=5\), Periode \(p_{19}=2\pi\), Mittelwert \(d_{19}=-4\)\\
		\begin{minipage}{\textwidth}
			\includegraphics[height=5cm]{\trigonometrie/pics/AllgSinA1_19.png}\\
		\end{minipage}	 
		\item Amplitude \(a_{20}=6\), Periode \(p_{20}=1\), Mittelwert \(d_{20}=10\)\\
		\begin{minipage}{\textwidth}
			\includegraphics[height=5cm]{\trigonometrie/pics/AllgSinA1_20.png}\\
		\end{minipage}	 
	\end{enumerate}
\end{Answer}
\begin{Answer}[ref=allgSinCosA2]
	\begin{enumerate}[label=\alph*)]
		\item \(f_1(x)=2\sin\left(x\right)+1\)
		\item \(f_2(x)=-\sin\left(\frac{1}{2}x\right)-1\)
		\item \(f_3(x)=1,5\cos\left(\pi x\right)+0,5\)
		\item \(f_4(x)=-3\cos\left(2\pi x\right)+3\)
		\item \(f_5(x)=-\cos\left(4x\right)+2\)
		\item \(f_6(x)=-0,5\sin\left(\frac{3}{2}x\right)+1\)
		\item \(f_7(x)=\sin\left(\frac{3\pi}{2}x\right)-0,5\)
		\item \(f_8(x)=-3\sin\left(4\pi x\right)-1\)
		\item \(f_9(x)=-2,5\cos\left(2\pi x\right)-5\)
		\item \(f_{10}(x)=\sin\left(\frac{1}{4} x\right)-0,75\)
	\end{enumerate}
\end{Answer}
% 	\newpage


\cohead{\Large\textbf{Trigonometrische Gleichungen}}
\fakesubsection{Trigonometrische Gleichungen ohne Streckung in \(x\)-Richtung}
Die allgemeine Sinusfunktion bzw. Cosinusfunktion sind gegeben durch:
\[f(x)=a\sin\left(bx\right)+d\qquad g(x)=a\cos\left(bx\right)+d\]
\begin{itemize}
	\item \(a\): \textcolor{loes}{Streckfaktor relativ zum Mittelwert in \(y\)-Richtung. Der Betrag von \(a\) gibt die Amplitude an.}\\
	\item \(d\): \textcolor{loes}{Verschiebung in \(y\)-Richtung. Da der Mittelwert zuvor bei \(0\) lag, entspricht \(d\) dem Mittelwert.}\\
	\item \(b\): \textcolor{loes}{Streckfaktor in \(x\)-Richtung. Zwischen dem Streckfaktor und der Periode \(p\) besteht folgender Zusammenhang: \(b\cdot p=2\pi\).}\\
\end{itemize}
Beispiel: \(\displaystyle f(x)=2\sin\left(0,5x\right)-1\)\\
\begin{minipage}{\textwidth}
	\includegraphics[width=.95\linewidth]{\trigonometrie/pics/AllgSin.png}\\
\end{minipage}\\
Beispiel: \(\displaystyle g(x)=-0,5\cos\left(\pi x\right)+1\)\\
\begin{minipage}{\textwidth}
	\includegraphics[width=.95\linewidth]{\trigonometrie/pics/AllgCos.png}\\
\end{minipage}\\
\newpage
\textbf{Bestimmen der Nullstellen:}\\
Die Nullstellen sind an sich nicht schwer zu bestimmen. Das Problem ist, dass es unendlich viele Nullstellen gibt. Will man alle Nullstellen aufschreiben, so braucht man eine neue Notation. Das gleiche Problem ergibt sich bei den Maxima, Minima und Wendestellen.
\begin{tcolorbox}
	\textbf{Notation für unendlich viele Stellen:}\\
	Erinnerung: Die ganzen Zahlen sind wie folgt definiert: \(\Z=\{0,\ 1,\ -1,\ 2,\ -2,\ 3,\ -3,\dots\}\)
	\textcolor{loestc}{Zur Notation nutzen wir aus, dass sich die Nullstellen/Maxima/Minima/Wendestellen in festen Abständen wiederholen, z.B. wieder beträgt der Abstand von einer Nullstelle zur nächsten bei sowohl \(\sin x\) als auch \(\cos x\)	jeweils \(\pi\):
		\[\text{alle Stellen }=\text{ eine Stelle }+\ k\cdot\text{Abstand der Stellen, }k\in\Z\]
	Bsp.: Nullstellen von \(\sin x\): \(x_k=0+k\cdot \pi,\ k\in\Z\)
	}
\end{tcolorbox}

\begin{Exercise}[title={\raggedright\normalfont Gib jeweils die Amplitude, die Periode und den Mittelwert an. Skizziere dann das Schaubild so, dass mindestens eine Periode zu sehen ist.}, label=allgSinCosA1]
\begin{enumerate}[label=\alph*)]
	\item \(f_1(x)=-2\sin\left(3x\right)-4\)
	\item \(f_2(x)=1,5\sin\left(4x\right)-2\)
	\item \(f_3(x)=-3\cos\left(0,5x\right)+1\)
	\item \(f_4(x)=\cos\left(\frac{1}{3}x\right)\)
	\item \(f_5(x)=-\sin\left(\frac{1}{2}x\right)-1\)
	\item \(f_6(x)=3\sin\left(\frac{2}{3}x\right)+3\)
	\item \(f_7(x)=4\cos\left(\frac{3}{4}x\right)+2\)
	\item \(f_8(x)=2,5\sin\left(\pi x\right)-1\)
	\item \(f_9(x)=-\cos\left(\frac{\pi}{2}x\right)+2\)
	\item \(f_{10}(x)=-4\sin\left(2\pi x\right)-1\)
	\item \(f_{11}(x)=-2,5\cos\left(4\pi x\right)+3,5\)
	\item \(f_{12}(x)=3\cos\left(1,5x\right)+2\)
	\item \(f_{13}(x)=0,5\sin\left(0,5\pi x\right)+1,5\)
	\item \(f_{14}(x)=-5\sin\left(4\pi x\right)-3\)
	\item \(f_{15}(x)=3\cos\left(x\right)+2\)
	\item \(f_{16}(x)=-2\sin\left(2x\right)\)
	\item \(f_{17}(x)=2,5\sin\left(2x\right)-3,5\)
	\item \(f_{18}(x)=3,5\cos\left(\frac{2}{3}\pi x\right)+2\)
	\item \(f_{19}(x)=5\cos\left(x\right)-4\)
	\item \(f_{20}(x)=-6\cos\left(2\pi x\right)+10\)
\end{enumerate}
\end{Exercise}
\newpage
%%%%%%%%%%%%%%%%%%%%%%%%%%%%%%%%%%%%%%%%%%%%%%%%%%%%%%%%%%%%%%%%%%%%%%%%%%%%%%%%%%%%%%%%%%%%%%%%%%%%%%%%%%%%%%%%%%%%%%%%
\begin{Exercise}[title={Stelle jeweils die Funktionsgleichung vom Typ \(a\cdot \sin\left(bx\right)+d\)\\oder \(a\cdot \cos\left(bx\right)+d\) auf.}, label=allgSinCosA2]\\
	\begin{minipage}{\textwidth}
		\begin{minipage}{0.49\textwidth}
			\begin{enumerate}[label=\alph*)]
				\item \begin{minipage}{.9\textwidth}
					\includegraphics[width=.75\linewidth]{\trigonometrie/pics/AllgSinA2_1.png}\\
				\end{minipage}	
				\item \begin{minipage}{.9\textwidth}
					\includegraphics[width=.75\linewidth]{\trigonometrie/pics/AllgSinA2_2.png}\\
				\end{minipage}
				\item \begin{minipage}{.9\textwidth}
					\includegraphics[width=.75\linewidth]{\trigonometrie/pics/AllgSinA2_3.png}\\
				\end{minipage}
				\item \begin{minipage}{.9\textwidth}
					\includegraphics[width=.75\linewidth]{\trigonometrie/pics/AllgSinA2_4.png}\\
				\end{minipage}
				\item \begin{minipage}{.9\textwidth}
					\includegraphics[width=.75\linewidth]{\trigonometrie/pics/AllgSinA2_5.png}\\
				\end{minipage}
			\end{enumerate}
		\end{minipage}
		\begin{minipage}{0.49\textwidth}
			\begin{enumerate}[label=\alph*)]
				\setcounter{enumi}{5}
				\item \begin{minipage}{.9\textwidth}
					\includegraphics[width=.75\linewidth]{\trigonometrie/pics/AllgSinA2_6.png}\\
				\end{minipage}	
				\item \begin{minipage}{.9\textwidth}
					\includegraphics[width=.75\linewidth]{\trigonometrie/pics/AllgSinA2_7.png}\\
				\end{minipage}
				\item \begin{minipage}{.9\textwidth}
					\includegraphics[width=.75\linewidth]{\trigonometrie/pics/AllgSinA2_8.png}\\
				\end{minipage}
				\item \begin{minipage}{.9\textwidth}
					\includegraphics[width=.75\linewidth]{\trigonometrie/pics/AllgSinA2_9.png}\\
				\end{minipage}
				\item \begin{minipage}{.9\textwidth}
				\includegraphics[width=.75\linewidth]{\trigonometrie/pics/AllgSinA2_10.png}\\
				\end{minipage}
			\end{enumerate}
		\end{minipage}
	\end{minipage}
\end{Exercise}


%%%%%%%%%%%%%%%%%%%%%%%%%%%%%%%%%%%%%%%%%
\begin{Answer}[ref=allgSinCosA1]
\begin{enumerate}[label=\alph*)]
	\item Amplitude \(a_{1}=2\), Periode \(p_{1}=\frac{2\pi}{3}\), Mittelwert \(d_{1}=-4\)\\
	 \begin{minipage}{\textwidth}
	 \includegraphics[height=5cm]{\trigonometrie/pics/AllgSinA1_1.png}\\
	 \end{minipage}	 
	 \item Amplitude \(a_{2}=1,5\), Periode \(p_{2}=\frac{\pi}{2}\), Mittelwert \(d_{2}=-2\)\\
	 \begin{minipage}{\textwidth}
	 \includegraphics[height=5cm]{\trigonometrie/pics/AllgSinA1_2.png}\\
	 \end{minipage}	 
	 \item Amplitude \(a_{3}=3\), Periode \(p_{3}=4\pi\), Mittelwert \(d_{3}=1\)\\
	 \begin{minipage}{\textwidth}
	 \includegraphics[height=5cm]{\trigonometrie/pics/AllgSinA1_3.png}\\
	 \end{minipage}	 
	 \item Amplitude \(a_{4}=1\), Periode \(p_{4}=6\pi\), Mittelwert \(d_{4}=0\)\\
	 \begin{minipage}{\textwidth}
	 \includegraphics[height=5cm]{\trigonometrie/pics/AllgSinA1_4.png}\\
	 \end{minipage}	 
	 \item Amplitude \(a_{5}=1\), Periode \(p_{5}=4\pi\), Mittelwert \(d_{5}=-1\)\\
	 \begin{minipage}{\textwidth}
	 \includegraphics[height=5cm]{\trigonometrie/pics/AllgSinA1_5.png}\\
	 \end{minipage}	 
	 \item Amplitude \(a_{6}=3\), Periode \(p_{6}=3\pi\), Mittelwert \(d_{6}=3\)\\
	 \begin{minipage}{\textwidth}
	 \includegraphics[height=5cm]{\trigonometrie/pics/AllgSinA1_6.png}\\
	 \end{minipage}	 
	 \item Amplitude \(a_{7}=4\), Periode \(p_{7}=\frac{8\pi}{3}\), Mittelwert \(d_{7}=2\)\\
	 \begin{minipage}{\textwidth}
	 \includegraphics[height=5cm]{\trigonometrie/pics/AllgSinA1_7.png}\\
	 \end{minipage}	 
	 \item Amplitude \(a_{8}=2,5\), Periode \(p_{8}=2\), Mittelwert \(d_{8}=-1\)\\
	 \begin{minipage}{\textwidth}
	 \includegraphics[height=5cm]{\trigonometrie/pics/AllgSinA1_8.png}\\
	 \end{minipage}	\newpage
	 \item Amplitude \(a_{9}=1\), Periode \(p_{9}=4\), Mittelwert \(d_{9}=2\)\\
	 \begin{minipage}{\textwidth}
	 \includegraphics[height=5cm]{\trigonometrie/pics/AllgSinA1_9.png}\\
	 \end{minipage}	 
	 \item Amplitude \(a_{10}=4\), Periode \(p_{10}=1\), Mittelwert \(d_{10}=-1\)\\
	 \begin{minipage}{\textwidth}
	 \includegraphics[height=5cm]{\trigonometrie/pics/AllgSinA1_10.png}\\
	 \end{minipage}	 
	 \item Amplitude \(a_{11}=2,5\), Periode \(p_{11}=\frac{1}{2}\), Mittelwert \(d_{11}=3,5\)\\
	 \begin{minipage}{\textwidth}
	 \includegraphics[height=5cm]{\trigonometrie/pics/AllgSinA1_11.png}\\
	 \end{minipage}	 
	 \item Amplitude \(a_{12}=3\), Periode \(p_{12}=\frac{4\pi}{3}\), Mittelwert \(d_{12}=2\)\\
	 \begin{minipage}{\textwidth}
	 \includegraphics[height=5cm]{\trigonometrie/pics/AllgSinA1_12.png}\\
	 \end{minipage}	 \newpage
	 \item Amplitude \(a_{13}=0,5\), Periode \(p_{13}=4\), Mittelwert \(d_{13}=1,5\)\\
	 \begin{minipage}{\textwidth}
	 \includegraphics[height=5cm]{\trigonometrie/pics/AllgSinA1_13.png}\\
	 \end{minipage}	 
	 \item Amplitude \(a_{14}=5\), Periode \(p_{14}=0,5\), Mittelwert \(d_{14}=-3\)\\
	 \begin{minipage}{\textwidth}
	 \includegraphics[height=5cm]{\trigonometrie/pics/AllgSinA1_14.png}\\
	 \end{minipage}	 
	 \item Amplitude \(a_{15}=3\), Periode \(p_{15}=2\pi\), Mittelwert \(d_{15}=2\)\\
	 \begin{minipage}{\textwidth}
	 \includegraphics[height=5cm]{\trigonometrie/pics/AllgSinA1_15.png}\\
	 \end{minipage}	 
	 \item Amplitude \(a_{16}=2\), Periode \(p_{16}=\pi\), Mittelwert \(d_{16}=0\)\\
	 \begin{minipage}{\textwidth}
	 \includegraphics[height=5cm]{\trigonometrie/pics/AllgSinA1_16.png}\\
	 \end{minipage}	 \newpage
	 \item Amplitude \(a_{17}=2,5\), Periode \(p_{17}=\pi\), Mittelwert \(d_{17}=-3,5\)\\
	 \begin{minipage}{\textwidth}
	 \includegraphics[height=5cm]{\trigonometrie/pics/AllgSinA1_17.png}\\
	 \end{minipage}	 
	 \item Amplitude \(a_{18}=3,5\), Periode \(p_{18}=3\), Mittelwert \(d_{18}=2\)\\
	 \begin{minipage}{\textwidth}
	 \includegraphics[height=5cm]{\trigonometrie/pics/AllgSinA1_18.png}\\
	 \end{minipage}	 
	 \item Amplitude \(a_{19}=5\), Periode \(p_{19}=2\pi\), Mittelwert \(d_{19}=-4\)\\
	 \begin{minipage}{\textwidth}
	 \includegraphics[height=5cm]{\trigonometrie/pics/AllgSinA1_19.png}\\
	 \end{minipage}	 
	 \item Amplitude \(a_{20}=6\), Periode \(p_{20}=1\), Mittelwert \(d_{20}=10\)\\
	 \begin{minipage}{\textwidth}
	 \includegraphics[height=5cm]{\trigonometrie/pics/AllgSinA1_20.png}\\
	 \end{minipage}	 
\end{enumerate}
\end{Answer}
\begin{Answer}[ref=allgSinCosA2]
\begin{enumerate}[label=\alph*)]
	\item \(f_1(x)=2\sin\left(x\right)+1\)
	\item \(f_2(x)=-\sin\left(\frac{1}{2}x\right)-1\)
	\item \(f_3(x)=1,5\cos\left(\pi x\right)+0,5\)
	\item \(f_4(x)=-3\cos\left(2\pi x\right)+3\)
	\item \(f_5(x)=-\cos\left(4x\right)+2\)
	\item \(f_6(x)=-0,5\sin\left(\frac{3}{2}x\right)+1\)
	\item \(f_7(x)=\sin\left(\frac{3\pi}{2}x\right)-0,5\)
	\item \(f_8(x)=-3\sin\left(4\pi x\right)-1\)
	\item \(f_9(x)=-2,5\cos\left(2\pi x\right)-5\)
	\item \(f_{10}(x)=\sin\left(\frac{1}{4} x\right)-0,75\)
\end{enumerate}
\end{Answer}

%%%%%%%%%%%%%%%%%%%%%%%%%%%%%%%%%%%%%%%%%%
%\newpage
%\cohead{\Large\textbf{Lösungen}}
%\fakesubsection{Lösungen}
%\shipoutAnswer
\end{document}

\begin{enumerate}[label=\alph*)]
	\setcounter{enumi}{3}
	\item \(a=-1\) und \(b=2\)
	\item \(a=-1\) und \(b=1\)
	\item \(a=0\) und \(b=2\)
\end{enumerate}


\begin{Exercise}[title={\raggedright xxxxxxxxxxxxxxxxxxxxxxxxx}, label=xxxxxxxxxxxxxxxxxxxx]\\
	\begin{minipage}{\textwidth}
		\adjustbox{valign=t}{\begin{minipage}{0.5\linewidth}
				\begin{enumerate}[label=\alph*)]
					\item
				\end{enumerate}
		\end{minipage}}%
		\adjustbox{valign=t}{\begin{minipage}{0.5\linewidth}
				\begin{enumerate}[label=\alph*)]
					\setcounter{enumi}{13}
					\item
				\end{enumerate}
		\end{minipage}}
	\end{minipage}
\end{Exercise}
\begin{Answer}[ref=xxxxxxxxxxxxxxxxxxx]\\
		\begin{minipage}{\textwidth}
		\adjustbox{valign=t}{\begin{minipage}{0.5\linewidth}
				\begin{enumerate}[label=\alph*)]
					\item
				\end{enumerate}
		\end{minipage}}%
		\adjustbox{valign=t}{\begin{minipage}{0.5\linewidth}
				\begin{enumerate}[label=\alph*)]
					\setcounter{enumi}{13}
					\item
				\end{enumerate}
		\end{minipage}}
	\end{minipage}
\end{Answer}
%\begin{Exercise}[title={xxxx}, label=xxxx]\\
%xxxx
%\end{Exercise}

%\begin{tcolorbox}\centering
%	\textcolor{loestc}{Das Schaubild einer ganzrationalen Funktion ist\dots\\
	%		\dots achsensymmetrisch zur y-Achse, wenn alle Hochzahlen gerade oder Null sind.\\
	%		\dots punksymmetrisch zum Ursprung, wenn alle Hochzahlen ungerade sind.\\
	%		\dots weder achsensymmetrisch zur y-Achse noch punktsymmetrisch zum Ursprung, wenn die Hochzahlen eine Mischung aus geraden Hochzahlen oder Null und ungeraden Hochzahlen sind.}
%\end{tcolorbox}
\begin{minipage}{\textwidth}
	\adjustbox{valign=t}{\begin{minipage}{0.24\textwidth}
			\centering\includegraphics[width=\textwidth]{\ableitung/pics/LK1.png}
	\end{minipage}}
	\adjustbox{valign=t}{\begin{minipage}{0.24\textwidth}
			\centering\includegraphics[width=\textwidth]{\ableitung/pics/RK1.png}
	\end{minipage}}
	\adjustbox{valign=t}{\begin{minipage}{0.24\textwidth}
			\centering\includegraphics[width=\textwidth]{\ableitung/pics/RK2.png}
	\end{minipage}}
	\adjustbox{valign=t}{\begin{minipage}{0.24\textwidth}
			\centering\includegraphics[width=\textwidth]{\ableitung/pics/LK2.png}
	\end{minipage}}
\end{minipage}

\begin{tabular}{p{.3\textwidth}|p{.3\textwidth}|p{.3\textwidth}}
	\hline
	Monoton fallend auf\newline \(]-\infty;2]\)\newline
	Monoton steigend auf\newline \([2;\infty[\)
	&
	Monoton fallend auf\newline \(]-\infty;-2]\) und \([0;\infty[\)\newline
	Monoton steigend auf\newline \([-2;0]\)
	&
	Monoton fallend auf\newline \(]-\infty;1]\) und \([3;\infty[\)\newline
	Monoton steigend auf\newline \([1;3]\)
	\\
	\hline
	Monoton fallend auf\newline \([-2;2]\)\newline
	Monoton steigend auf\newline \(]-\infty;-2]\) und \([2;\infty[\)
	&
	Monoton fallend auf\newline \([3;\infty[\)\newline
	Monoton steigend auf\newline \(]-\infty;3]\)
	&
	Monoton fallend auf\newline \(]-\infty;-2]\) und \([1;2]\)\newline
	Monoton steigend auf\newline \([-2;0]\) und \([2;\infty[\)
	\\
	\hline
	Monoton fallend auf\newline \([-4;-2]\) und \([0;\infty[\)\newline
	Monoton steigend auf\newline \(]-\infty;-4]\) und \([-2;0]\)
	&
	Monoton fallend auf\newline \([-1;0]\) und \([1;2]\)\newline
	Monoton steigend auf\newline \(]-\infty;-1]\), \([0;1]\) und \([1;\infty[\)
	&
	Monoton fallend auf\newline \(]-\infty;-1]\) und \([0;1]\)\newline
	Monoton steigend auf\newline \([-1;0]\) und \([1;\infty[\)
	\\
	\hline
\end{tabular}
