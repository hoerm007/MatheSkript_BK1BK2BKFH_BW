% !TeX root = ../../Skript.tex
\cohead{\Large\textbf{Trigonometrische Gleichungen}}
\fakesubsection{Trigonometrische Gleichungen mit Streckung in \textit{x}-Richtung}
Wir können Gleichungen der Form \(\cos(bx)=r\) bzw. \(\sin(bx)=r\) lösen, indem wir die gleichen Schritte durchführen wie beim Lösen von Gleichungen der Form \(\cos(x)=r\) bzw. \(\sin(x)=r\). Es sind lediglich zwei Extraschritte notwendig:

\bigskip

\begin{minipage}{\textwidth}
	\adjustbox{valign=t}{\begin{minipage}{0.5\linewidth}
			Beispiel: \(2\cos(\pi x)-\sqrt{2}=0\)
    \end{minipage}}%
    \adjustbox{valign=t}{\begin{minipage}{0.5\linewidth}
            Beispiel: \(4\sin(0,25x)-1=1\phantom{\sqrt{2}}\)
    \end{minipage}}%
\end{minipage}

\bigskip

\begin{minipage}{\textwidth}
    \adjustbox{valign=t}{\begin{minipage}{0.5\linewidth}
        \begin{enumerate}
            \item Gleichung zu \(\cos(bx)=r\) umformen
            \begin{align*}
                \textcolor{loes}{2\cos(\pi x)-\sqrt{2}}&\textcolor{loes}{\;=0\ \vert +\sqrt{2}}\\
                \textcolor{loes}{2\cos(\pi x)}&\textcolor{loes}{\;=\sqrt{2}\ \vert :2}\\
                \textcolor{loes}{\cos(\pi x)}&\textcolor{loes}{\;=\frac{\sqrt{2}}{2}}
            \end{align*}
        \end{enumerate}
    \end{minipage}}%
    \adjustbox{valign=t}{\begin{minipage}{0.5\linewidth}
        \begin{enumerate}
            \item Gleichung zu \(\sin(bx)=r\) umformen
            \begin{align*}
                \textcolor{loes}{4\sin(0,25 x)-1}&\textcolor{loes}{\;=1\ \vert +1}\phantom{\sqrt{2}}\\
                \textcolor{loes}{4\sin(0,25 x)}&\textcolor{loes}{\;=2\ \vert :4}\phantom{\sqrt{2}}\\
                \textcolor{loes}{\sin(0,25 x)}&\textcolor{loes}{\;=0,5}\phantom{\frac{\sqrt{2}}{2}}
            \end{align*}
        \end{enumerate}
    \end{minipage}}%

    \vspace{1cm}

\end{minipage}

\bigskip

\begin{minipage}{\textwidth}
    \adjustbox{valign=t}{\begin{minipage}{0.5\linewidth}
            \begin{enumerate}
                \setcounter{enumi}{1}
                \item Substitution \(bx=z\)
                \begin{align*}
                    \textcolor{loes}{\cos(\pi x)}&\textcolor{loes}{\;=\frac{\sqrt{2}}{2}\ \bigg\vert\text{ Sub. }z=\pi x}\\
                    \textcolor{loes}{\cos(z)}&\textcolor{loes}{\;=\frac{\sqrt{2}}{2}}
                \end{align*}
            \end{enumerate}
    \end{minipage}}%
    \adjustbox{valign=t}{\begin{minipage}{0.5\linewidth}
            \begin{enumerate}
            \setcounter{enumi}{1}
                \item Substitution \(bx=z\)
                \begin{align*}
                    \textcolor{loes}{\sin(0,25 x)}&\textcolor{loes}{\;=0,5\ \big\vert\text{ Sub. }z=0,25 x}\phantom{\bigg\vert\frac{\sqrt{2}}{2}}\\
                    \textcolor{loes}{\sin(z)}&\textcolor{loes}{\;=0,5}\phantom{\frac{\sqrt{2}}{2}}
                \end{align*}
            \end{enumerate}
    \end{minipage}}%

\vspace{1cm}

\end{minipage}

\bigskip

\begin{minipage}{\textwidth}
    \adjustbox{valign=t}{\begin{minipage}{0.5\linewidth}
            \begin{enumerate}
                \setcounter{enumi}{2}
                \item Erste Lösung mit \(\cos^{-1}\) bestimmen
                \begin{align*}
                    \textcolor{loes}{\cos(z)}&\textcolor{loes}{\;=\frac{\sqrt{2}}{2}\ \vert \cos^{-1}}\\
                    \textcolor{loes}{z_1}&\textcolor{loes}{\;=\arccoss{\frac{\sqrt{2}}{2}}}\\
                    \textcolor{loes}{z_1}&\textcolor{loes}{\;=\frac{\pi}{4}}
                \end{align*}
            \end{enumerate}
    \end{minipage}}%
    \adjustbox{valign=t}{\begin{minipage}{0.5\linewidth}
            \begin{enumerate}
                \setcounter{enumi}{2}
                \item Erste Lösung mit \(\sin^{-1}\) bestimmen
                \begin{align*}
                    \textcolor{loes}{\sin(z)}&\textcolor{loes}{\;=0,5\ \vert \sin^{-1}}\phantom{\frac{\sqrt{2}}{2}}\\
                    \textcolor{loes}{z_1}&\textcolor{loes}{\;=\arcsinn{0,5}}\phantom{\arccoss{\frac{\sqrt{2}}{2}}}\\
                    \textcolor{loes}{z_1}&\textcolor{loes}{\;=\frac{\pi}{6}}
                \end{align*}
            \end{enumerate}
    \end{minipage}}%

\vspace{1cm}

\end{minipage}

\bigskip

\begin{minipage}{\textwidth}
    \adjustbox{valign=t}{\begin{minipage}{0.5\linewidth}
            \begin{enumerate}
                \setcounter{enumi}{3}
                \item Zweite Lösung für \(z\) aus Symmetrie
                \begin{align*}
                    \textcolor{loes}{z_2}&\textcolor{loes}{\;=-z_1=-\frac{\pi}{4}}
                \end{align*}
            \end{enumerate}
    \end{minipage}}%
    \adjustbox{valign=t}{\begin{minipage}{0.5\linewidth}
            \begin{enumerate}
                \setcounter{enumi}{3}
                \item Zweite Lösung für \(z\) aus Symmetrie
                \begin{align*}
                    \textcolor{loes}{z_2}&\textcolor{loes}{\;=\pi-z_1=\frac{5\pi}{6}}
                \end{align*}
            \end{enumerate}
    \end{minipage}}%
\end{minipage}

\bigskip

\begin{minipage}{\textwidth}
    \adjustbox{valign=t}{\begin{minipage}{0.5\linewidth}
            \begin{enumerate}
                \setcounter{enumi}{4}
                \item Alle Lösungen bestimmen
                \begin{align*}
                    \textcolor{loes}{z_k}&\textcolor{loes}{\;=\pm\frac{\pi}{4}+2\pi k,\ k\in\Z}
                \end{align*}
            \end{enumerate}
    \end{minipage}}%
    \adjustbox{valign=t}{\begin{minipage}{0.5\linewidth}
            \begin{enumerate}
                \setcounter{enumi}{4}
                \item Alle Lösungen bestimmen
                \begin{align*}
                    \textcolor{loes}{z_k}&\textcolor{loes}{\;=\frac{\pi}{6}+2\pi k\text{ oder }z_k=\frac{5\pi}{6}+2\pi k,\ k\in\Z}
                \end{align*}
            \end{enumerate}
    \end{minipage}}%

\vspace{1cm}

\end{minipage}

\bigskip

\begin{minipage}{\textwidth}
    \adjustbox{valign=t}{\begin{minipage}{0.5\linewidth}
            \begin{enumerate}
                \setcounter{enumi}{5}
                \item Rücksubstitution
                \begin{align*}
                    \textcolor{loes}{\pi x_k}&\textcolor{loes}{\;=\pm\frac{\pi}{4}+2\pi k\ \vert :\pi}\\
                    \textcolor{loes}{x_k}&\textcolor{loes}{\;=\pm\frac{1}{4}+2 k}
                \end{align*}
            \end{enumerate}
    \end{minipage}}%
    \adjustbox{valign=t}{\begin{minipage}{0.5\linewidth}
            \begin{enumerate}
                \setcounter{enumi}{5}
                \item Rücksubstitution
                \begin{align*}
                    \textcolor{loes}{0,25 x_k}&\textcolor{loes}{\;=\frac{\pi}{6}+2\pi k\ \vert\ \cdot 4}\\
                    \textcolor{loes}{x_k}&\textcolor{loes}{\;=\frac{2\pi}{3}+8\pi k\text{ oder }}\\
                    \textcolor{loes}{0,25 x_k}&\textcolor{loes}{\;=\frac{5\pi}{6}+2\pi k\ \vert\cdot 4}\\
                    \textcolor{loes}{x_k}&\textcolor{loes}{\;=\frac{10\pi}{3}+8\pi k}
                \end{align*}
            \end{enumerate}
    \end{minipage}}%
\end{minipage}

\newpage

\begin{Exercise}[title={\raggedright\normalfont Bestimme jeweils alle Lösungen:}, label=sincosGleichungenAllgA1]
	\begin{enumerate}[label=\alph*)]
		\item \(-3\sinn{2x}=\frac{3}{2}\)
		\item \(4\sinn{2\pi x}=1+\sqrt{5}\)
		\item \(\coss{0,5x}=\frac{1}{2}\)
		\item \(\coss{\frac{\pi}{2}x}=-1\)
		\item \(4\sinn{3\pi x}=-1\)
		\item \(0,5\coss{5x}+2=3\)
		\item \(-5\sinn{\frac{2}{3}x}=3\)
		\item \(\coss{\frac{5}{4}x}-3=-2,5\)
		\item \(5\sinn{3\pi x}=0\)
		\item \(4\sinn{\frac{3\pi}{2}x}=1\)
		\item \(\frac{1}{3}\coss{2x}=-\frac{1}{8}\)
		\item \(-\sinn{6\pi x}+1,6=1,3\)
		\item \(0,5\coss{\frac{\pi}{6}x}=0,6\)
		\item \(-3\coss{0,2x}+2=\frac{3}{2}\)
		\item \(-\frac{1}{7}\coss{\pi x}-\frac{1}{5}=-\frac{1}{10}\)
		\item \(-6\sinn{\frac{1}{\pi}x}-3=1\)
		\item \(2\sinn{2,5x}-4=-3,1\)
		\item \(4\coss{8x}=-2\sqrt{2}\)
		\item \(3\coss{\frac{5\pi}{8}x}-\frac{1}{4}=-1\)
		\item \(4\sinn{3x}+6=-10\)
		\item \(4\coss{\frac{3\pi}{4}x}-12=-10,8\)
		\item \(2\sinn{6x}+2=2\)
		\item \(-\frac{3}{4}\coss{\frac{3}{8}x}+\frac{1}{8}=\frac{1}{2}\)
		\item \(-\frac{25}{13}\sinn{5\pi x}-\frac{5}{2}=-\frac{5}{7}\)
		\item \(\frac{5}{3}\sinn{\frac{5\pi}{3}x}+\frac{8}{3}=\frac{5}{3}\)
		\item \(-\frac{9}{4}\coss{\frac{1}{3}x}-1=0\)
	\end{enumerate}
\end{Exercise}

%%%%%%%%%%%%%%%%%%%%%%%%%%%%%%%%%%%%%%%%%
\begin{Answer}[ref=sincosGleichungenAllgA1]
	\begin{enumerate}[label=\alph*)]
		\item \(x_k=-\frac{\pi}{12} +\pi k\text{ oder }x_k=\frac{7\pi}{12} +\pi k,\ k\in\Z\)
		\item \(x_k=-\frac{1}{20} + k\text{ oder }x_k=\frac{7}{20} + k,\ k\in\Z\)
		\item \(x_k=\pm \frac{2\pi}{3} +4\pi k,\ k\in\Z\)
		\item \(x_k=\pm 2 +4 k,\ k\in\Z\)
		\item \(x_k=\frac{1}{3\pi}\arcsinn{-\frac{1}{4}} +\frac{2}{3} k\approx -0,03+\frac{2}{3} k\text{ oder }x_k=\frac{1}{3}-\frac{1}{3\pi}\arcsinn{-\frac{1}{4}} +\frac{2}{3} k\approx 0,36+\frac{2}{3} k,\ k\in\Z\)
		\item keine Lösungen
		\item \(x_k=\frac{3}{2}\arcsinn{-\frac{3}{5}} +3\pi k\approx -0,97+3\pi k\text{ oder }x_k=\frac{3\pi}{2}-\frac{3}{2}\arcsinn{-\frac{3}{5}} +3\pi k\approx 5,68+3\pi k,\ k\in\Z\)
		\item \(x_k=\pm \frac{4\pi}{15} +\frac{8}{5}\pi k,\ k\in\Z\)
		\item \(x_k=\frac{2}{3} k\text{ oder }x_k=\frac{1}{2} +\frac{2}{3} k,\ k\in\Z\)
		\item \(x_k=\frac{2}{3\pi}\arcsinn{\frac{1}{4}} +\frac{4}{3} k\approx 0,05+\frac{4}{3} k\text{ oder }x_k=\frac{2}{3}-\frac{2}{3\pi}\arcsinn{\frac{1}{4}} +\frac{4}{3} k\approx 0,61+\frac{4}{3} k,\ k\in\Z\)
		\item \(x_k=\pm \frac{1}{2}\arccoss{-\frac{3}{8}} +\pi k\approx\pm 0,98 +\pi k,\ k\in\Z\)
		\item \(x_k=\frac{1}{6\pi}\arcsinn{0,3} +\frac{1}{3} k\approx 0,02+\frac{1}{3} k\text{ oder }x_k=\frac{1}{6}-\frac{1}{6\pi}\arcsinn{0,3} +\frac{1}{3} k\approx 0,15+\frac{1}{3} k,\ k\in\Z\)
		\item keine Lösungen
		\item \(x_k=\pm 5\arccoss{\frac{1}{6}} +10\pi k\approx\pm 7,02 +10\pi k,\ k\in\Z\)
		\item \(x_k=\pm \frac{1}{\pi}\arccoss{-\frac{7}{10}} +2k\approx\pm 0,75 +2k,\ k\in\Z\)
		\item \(x_k=\pi\arcsinn{\frac{2}{3}} +2\pi^2 k\approx 2,29+2\pi^2 k\text{ oder }x_k=\pi^2-\pi\arcsinn{\frac{2}{3}} +2\pi^2 k\approx 7,58+2\pi^2 k,\ k\in\Z\)
		\item \(x_k=0,4\arcsinn{0,45} +0,8\pi k\approx 0,19+0,8\pi k\text{ oder }\)

        \(x_k=0,4\pi-0,4\arcsinn{0,45} +0,8\pi k\approx 1,07+0,8\pi k,\ k\in\Z\)
		\item \(x_k=\pm \frac{3\pi}{32} +\frac{\pi}{4} k,\ k\in\Z\)
		\item \(x_k=\pm \frac{8}{5\pi}\arccoss{-\frac{1}{4}} +\frac{16}{5} k\approx\pm 0,93 +\frac{16}{5} k,\ k\in\Z\)
		\item keine Lösungen
		\item \(x_k=\pm \frac{4}{3\pi}\arccoss{\frac{3}{10}} +\frac{8}{3} k\approx\pm 0,54 +\frac{8}{3} k,\ k\in\Z\)
		\item \(x_k=\frac{\pi}{3} k\text{ oder }x_k=\frac{\pi}{6} +\frac{\pi}{3} k,\ k\in\Z\)
		\item \(x_k=\pm \frac{16}{9}\pi +\frac{16}{3}\pi k,\ k\in\Z\)
		\item \(x_k=\frac{1}{5\pi}\arcsinn{-\frac{13}{14}} +\frac{2}{5} k\approx -0,08+\frac{2}{5} k\text{ oder }x_k=\frac{1}{5}-\frac{1}{5\pi}\arcsinn{-\frac{13}{14}} +\frac{2}{5} k\approx 0,28+\frac{2}{5} k,\ k\in\Z\)
		\item \(x_k=\frac{3}{5\pi}\arcsinn{-\frac{3}{5}} +\frac{6}{5} k\approx -0,12+\frac{6}{5} k\text{ oder }x_k=\frac{6}{5}-\frac{3}{5\pi}\arcsinn{-\frac{3}{5}} +\frac{6}{5} k\approx 1,32+\frac{6}{5} k,\ k\in\Z\)
		\item \(x_k=\pm 3\arccoss{-\frac{4}{9}} +6\pi k\approx \pm 6,09 +6\pi k,\ k\in\Z\)
	\end{enumerate}
\end{Answer}