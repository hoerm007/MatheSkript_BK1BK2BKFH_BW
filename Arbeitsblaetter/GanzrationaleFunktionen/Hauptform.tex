% !TeX root = ../../Skript.tex
\cohead{\Large\textbf{Hauptform ganzrat. Funktionen}}
\fakesubsection{Hauptform ganzrat. Funktionen}
Funktionen, deren Funktionsgleichung man wie folgt darstellen kann, bezeichnet man als ganzrationale Funktionen:

\(f(x)=a_nx^n+a_{n-1}x^{n-1}+a_{n-2}x^{n-2}+\dots +a_2x^2+a_1x+a_0, \quad a_n\neq 0, \quad n\in\N\)

Diese Darstellungsform (komplett ausmultipliziert und zusammengefasst) bezeichnet man als Hauptform oder Normalform.

Folgende Begriffe finden für die ganzrationalen Funktionen Verwendung:
\begin{itemize}\large
	\item[\textcolor{loes}{\textbullet}] \textcolor{loes}{\(n\): Grad der Funktion (größte Hochzahl)}
	\item[\textcolor{loes}{\textbullet}]  \textcolor{loes}{\(a_n,\ a_{n-1},\ a_{n-2}, \dots,\ a_2,\ a_1,\ a_0\): Koeffizienten}
	\item[\textcolor{loes}{\textbullet}]  \textcolor{loes}{\(a_n\): Leitkoeffizient (Koeffizient, der vor dem \(x\) mit der größten Hochzahl steht)}
	\item[\textcolor{loes}{\textbullet}]  \textcolor{loes}{\(a_0\): Absolutglied (immer der Koeffizient, der ohne \(x\) alleine steht)}
\end{itemize}\vspace{2cm}
Beispiele:

\medskip

\begin{minipage}{\textwidth}
	\adjustbox{valign=t}{\begin{minipage}{0.5\textwidth}\raggedright
		\(f(x)=-4x^5+3x^4-\frac{1}{2}x^3+x-3\)

		Grad: \(5\)

		Koeffizienten: \(a_5=-4\), \(a_4=3\), \(a_3=-\frac{1}{2}\),

		\(a_2=0\), \(a_1=1\), \(a_0=-3\)

		Leitkoeffizient \(a_5=-4\)

		Absolutglied \(a_0=-3\)
	\end{minipage}}%
	\adjustbox{valign=t}{\begin{minipage}{0.5\textwidth}\raggedright
		\(g(x)=x^4+2x^3-0,5x\)

		Grad: \textcolor{loes}{\(4\)}

		Koeffizienten: \textcolor{loes}{\(a_4=1\), \(a_3=2\), \(a_2=0\),}

		\textcolor{loes}{\(a_1=-0,5\), \(a_0=0\)}

		Leitkoeffizient \textcolor{loes}{\(a_4=1\)}

		Absolutglied \textcolor{loes}{\(a_0=0\)}
	\end{minipage}}%
\end{minipage}
%%%%%%%%%%%%%%%%%%%%%%%%%%%%%%%%%%%%%%%%%%%%%%%%%%%%%%%%%%%%%%%%%%%%%%%%%%%%%%%%%%%%%%%%%%%%%%%%%%%%%%%%%%%%%%%%%%%%%
\begin{Exercise}[title={Gib den Grad, die Koeffizienten, den Leitkoeffizienten sowie das Absolutglied an.}, label=ganzHauptA1]

	\begin{minipage}{\textwidth}
		\begin{minipage}{0.5\textwidth}
			\begin{enumerate}[label=\alph*)]
				\item \(f(x)=-6x^3+2x-3\)
				\item \(g(x)=0,5x^5-7x^4+2,5x\)
				\item \(h(x)=2x^6\)
				\item \(i(x)=-\frac{3}{2}x^5-8x^4+x^2-1\)
			\end{enumerate}
		\end{minipage}%
		\begin{minipage}{0.5\textwidth}
			\begin{enumerate}[label=\alph*)]
				\setcounter{enumi}{4}
				\item \(j(x)=0,1x^4-12x^3-x^2+8,6x-3,1\)
				\item \(k(x)=-\frac{3}{5}x^7+\frac{2}{7}x^6-\frac{11}{6}x^4-\frac{12}{5}x\)
				\item \(l(x)=2x\left(x^3-2x^2+5\right)\)
				\item \(m(x)=-3x^2\left(x+2\right)^2\)
			\end{enumerate}
		\end{minipage}%
	\end{minipage}%
\end{Exercise}
\newpage
%%%%%%%%%%%%%%%%%%%%%%%%%%%%%%%%%%%%%%%%%
\begin{Answer}[ref=ganzHauptA1]

	\begin{minipage}{\textwidth}
		\begin{minipage}[t]{0.49\textwidth}
			\begin{enumerate}[label=\alph*)]
				\item Grad: \(3\)

				Koeffizienten: \(a_3=-6\), \(a_2=0\), \(a_1=2\), \(a_0=-3\)

				Leitkoeffizient \(a_3=-6\)

				Absolutglied \(a_0=-3\)
				\item Grad: \(5\)

				Koeffizienten: \(a_5=0,5\), \(a_4=-7\), \(a_3=0\), \(a_2=0\), \(a_1=2,5\), \(a_0=0\)

				Leitkoeffizient \(a_5=0,5\)

				Absolutglied \(a_0=0\)
				\item
				Grad: \(6\)

				Koeffizienten: \(a_6=2\), \(a_5=0\), \(a_4=0\), \(a_3=0\), \(a_2=0\), \(a_1=0\), \(a_0=0\)

				Leitkoeffizient \(a_6=2\)

				Absolutglied \(a_0=0\)
				\item Grad: \(5\)

				Koeffizienten: \(a_5=-\frac{3}{2}\), \(a_4=-8\), \(a_3=0\), \(a_2=1\), \(a_1=0\), \(a_0=-1\)

				Leitkoeffizient \(a_5=-\frac{3}{2}\)

				Absolutglied \(a_0=-1\)
			\end{enumerate}
		\end{minipage}%
		\begin{minipage}[t]{0.5\textwidth}
			\begin{enumerate}[label=\alph*)]
				\setcounter{enumi}{4}
				\item Grad: \(4\)

				Koeffizienten: \(a_4=0,1\), \(a_3=-12\),

				\(a_2=-1\), \(a_1=8,6\), \(a_0=-3,1\)

				Leitkoeffizient \(a_4=0,1\)

				Absolutglied \(a_0=-3,1\)
				\item Grad: \(7\)

				Koeffizienten: \(a_7=-\frac{3}{5}\), \(a_6=\frac{2}{7}\), \(a_5=0\), \(a_4=-\frac{11}{6}\), \(a_3=0\), \(a_2=0\), \(a_1=-\frac{12}{5}\), \(a_0=0\)

				Leitkoeffizient \(a_7=-\frac{3}{5}\)

				Absolutglied \(a_0=0\)
				\item \(l(x)=2x\left(x^3-2x^2+5\right)=2x^4-4x^3+10x\)

				Grad: \(4\)

				Koeffizienten: \(a_4=2\), \(a_3=-4\), \(a_2=0\), \(a_1=10\), \(a_0=0\)

				Leitkoeffizient \(a_4=2\)

				Absolutglied \(a_0=0\)
				\item \(m(x)=-3x^2\left(x+2\right)^2\)

				\(\hphantom{m(x)}=-3x^4-12x^3-12x^2\)

				Grad: \(4\)\\
				Koeffizienten: \(a_4=-3\), \(a_3=-12\), \(a_2=-12\),

				\(a_1=0\), \(a_0=0\)

				Leitkoeffizient \(a_4=-3\)

				Absolutglied \(a_0=0\)
			\end{enumerate}
		\end{minipage}%
	\end{minipage}%
\end{Answer}