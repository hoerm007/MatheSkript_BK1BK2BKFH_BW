\cohead{\Large\textbf{Ableitungsregeln}}
\fakesubsection{Ableitungsregeln}
Es ist mühsam die Ableitung für einzelne Stellen mit Hilfe des Differenzialquotienten zu bestimmen. Statt erst einen Wert für \(x_0\) einzusetzen und dann die momentane Änderungsrate zu bestimmen, können wir das Vorgehen umdrehen und erst die momentane Änderungsrate bestimmen und dann Werte einsetzen:\\
\begin{minipage}[t]{\textwidth}
	\begin{minipage}{0.4\textwidth}
		\begin{align*}
			f(x)&=x^2\\
			f'(x)&=\textcolor{loes}{\lim\limits_{h\to 0}\frac{f(x+h)-f(x)}{h}}\\
			&\textcolor{loes}{=\lim\limits_{h\to 0}\frac{(x+h)^2-x^2}{h}}\\
			&\textcolor{loes}{=\lim\limits_{h\to 0}\frac{x^2+2hx+h^2-x^2}{h}}\\
			&\textcolor{loes}{=\lim\limits_{h\to 0}\frac{2hx+h^2}{h}}\\
			&\textcolor{loes}{=\lim\limits_{h\to 0}2x+h}\\
			&\textcolor{loes}{=2x}\\
		\end{align*}
	\end{minipage}
	\begin{minipage}{0.6\textwidth}
		\textcolor{loes}{Die Ableitung von \(f(x)=x^2\) ist also \(f'(x)=2x\). Wir haben bereits gezeigt, dass die Ableitung an der Stelle \(x_0=1\) den Wert 2 haben muss. Tatsächlich ergibt sich \(f'(1)=2\cdot 1=2\).}
	\end{minipage}
\end{minipage}\\
\vspace{5cm}\\
%%%%%%%%%%%%%%%%%%%%%%%%%%%%%%%%%%%%%%%%%%%%%%%%%%%%%%%%%%%%%%%%%%%%%%%%%%%%%%%%%%%%%%%%%%%%%%%%%%%%%%
\begin{minipage}{\textwidth}
	\begin{Exercise}[title={\raggedright Berechne jeweils allgemein die Ableitung \(f'(x)\)}, label=ablRegelnA1]
		\begin{minipage}{\textwidth}
			\begin{minipage}{0.49\textwidth}
				\begin{enumerate}[label=\alph*)]
					\item \(f_1(x)=2x^2\)
					\item \(f_2(x)=-x^2\)
					\item \(f_3(x)=x\)
					\item \(f_4(x)=3x-4\)
					\item \(f_5(x)=x^2+1\)
					\item \(f_6(x)=x^2+x\)
					\item \(f_7(x)=\frac{2}{3}x^2-\frac{1}{2}\)
					\item \(f_8(x)=5\)
				\end{enumerate}
			\end{minipage}
			\begin{minipage}{0.49\textwidth}
				\begin{enumerate}[label=\alph*)]
					\setcounter{enumi}{8}
					\item \(f_9(x)=-x+7\)
					\item \(f_{10}(x)=\frac{2}{3}-5x^2\)
					\item \(f_{11}(x)=3x^2-5x\)
					\item \(f_{12}(x)=-0,5x^2+0,1x-9\)
					\item \(f_{13}(x)=-4\)
					\item \(f_{14}(x)=9-4x\)
					\item \(f_{15}(x)=8x-4x^2\)
					\item \(f_{16}(x)=x^3\)
				\end{enumerate}
			\end{minipage}
		\end{minipage}
	\end{Exercise}
\end{minipage}
%%%%%%%%%%%%%%%%%%%%%%%%%%%%%%%%%%%%%%%%%
\begin{Answer}[ref=ablRegelnA1]\\
	\begin{minipage}{\textwidth}
		\begin{minipage}{0.49\textwidth}
			\begin{enumerate}[label=\alph*)]
				\item \(f_1(x)=4x\)
				\item \(f_2(x)=-2x\)
				\item \(f_3(x)=1\)
				\item \(f_4(x)=3\)
				\item \(f_5(x)=2x\)
				\item \(f_6(x)=2x+1\)
				\item \(f_7(x)=\frac{4}{3}x\)
				\item \(f_8(x)=0\)
			\end{enumerate}
		\end{minipage}
		\begin{minipage}{0.49\textwidth}
			\begin{enumerate}[label=\alph*)]
				\setcounter{enumi}{8}
				\item \(f_9(x)=-1\)
				\item \(f_{10}(x)=-10x\)
				\item \(f_{11}(x)=6x-5\)
				\item \(f_{12}(x)=-x+0,1\)
				\item \(f_{13}(x)=0\)
				\item \(f_{14}(x)=-4\)
				\item \(f_{15}(x)=-8x+8\)
				\item \(f_{16}(x)=3x^2\)
			\end{enumerate}
		\end{minipage}
	\end{minipage}
\end{Answer}
