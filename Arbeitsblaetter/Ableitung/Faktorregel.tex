\cohead{\Large\textbf{Faktorregel}}
\fakesubsection{Faktorregel}
Wir kennen nun die Ableitung der Normalparabel. Es scheint naheliegend, dass die Ableitungen von \(2x^2,\ 3x^2,\ -x^2\) und ähnlichen Funktionen eine zu \(x^2\) ähnliche Ableitung haben. Wir berechnen die Ableitung von \(f_a(x)=ax^2\):

\begin{minipage}{\textwidth}
	\begin{minipage}{0.4\textwidth}
		\begin{align*}
			f_a(x)&=ax^2\\
			f_a'(x)&=\textcolor{loes}{\lim\limits_{h\to 0}\frac{f_a(x+h)-f_a(x)}{h}}\\
			&\textcolor{loes}{\;=\lim\limits_{h\to 0}\frac{a(x+h)^2-ax^2}{h}}\\
			&\textcolor{loes}{\;=\lim\limits_{h\to 0}\frac{ax^2+a2hx+ah^2-ax^2}{h}}\\
			&\textcolor{loes}{\;=\lim\limits_{h\to 0}\frac{a2hx+ah^2}{h}}\\
			&\textcolor{loes}{\;=\lim\limits_{h\to 0}a2x+ah}\\
			&\textcolor{loes}{\;=a2x}
		\end{align*}
	\end{minipage}%
	\begin{minipage}{0.6\textwidth}
		\textcolor{loes}{\(f(x)=x^2\) und \(f_a(x)=ax^2\) unterscheiden sich nur durch den Faktor \(a\). Das gleiche Verhalten zeigen die Ableitungen. \(f'(x)=2x\) und \(f_a'(x)=a2x\) unterscheiden sich ebenfalls nur durch den Faktor \(a\).}
	\end{minipage}%
\end{minipage}

\vspace{2cm}

Tatsächlich lässt sich ganz allgemein zeigen, dass die Ableitung einer Funktion und einem Faktor \(af(x)\) gleich dem gleichen Faktor mal der Ableitung ist: \((af(x))'=af'(x)\):

\bigskip

\begin{minipage}{\textwidth}
	\adjustbox{valign=t, padding = 0ex 0ex 20ex 0ex}{\begin{minipage}{0.6\textwidth-20ex}
		\begin{align*}
			g(x)&=af(x)\\
			g'(x)&=\textcolor{loes}{\lim\limits_{h\to 0}\frac{af(x+h)-af(x)}{h}}\\
			&\textcolor{loes}{\;=a\lim\limits_{h\to 0}\frac{f(x+h)-f(x)}{h}}\\
			&\textcolor{loes}{\;=af'(x)}
		\end{align*}
	\end{minipage}}%
	\adjustbox{valign=t}{\begin{minipage}{0.4\textwidth}
		\begin{tcolorbox}\raggedright

            \textbf{Faktorregel:}

            \bigskip

			\textcolor{loestc}{Die Ableitung von \(af(x)\) ist \(af'(x)\).}

            \bigskip

		\end{tcolorbox}
	\end{minipage}}%
\end{minipage}