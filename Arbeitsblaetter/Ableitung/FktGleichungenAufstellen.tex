\cohead{\Large\textbf{Aufstellen von Funktionsgleichungen}}
\fakesubsection{Aufstellen von Funktionsgleichungen}
\begin{tabular}{p{.3\textwidth}|p{.7\textwidth}}
	\textbf{Angabe}&\textbf{Gleichung}\\
	\hline
	\textbf{Punkt}\newline (\(x\)-Wert und \(y\)-Wert)\newline\newline&\textcolor{loes}{Punkte werden in die normale Funktionsgleichung eingesetzt (Punktprobe):  \(f(x_P)=y_P\)}\\
	\hline
	\textbf{Extrempunkt oder\newline Sattelpunkt}bei \(x_0\)\newline\newline&\textcolor{loes}{Die erste Ableitung muss Null ergeben: \(f'(x_0)=0\)}\\
	\hline
	\textbf{Wendestelle} an der Stelle \(x_0\)\newline\newline\newline&\textcolor{loes}{An einer Wendestelle muss die zweite Ableitung Null sein: \(f''(x_0)=0\)\newline Hinweis: Jeder Sattelpunkt ist auch ein Wendepunkt.}\\
	\hline
	\textbf{Steigung \(m\)} an der Stelle \(x_0\)\newline\newline\newline&\textcolor{loes}{Die Steigung entspricht der Ableitung: \(f'(x_0)=m\)}\\
	\hline
	\textbf{Nullstellen}&\textcolor{loes}{Punktprobe: \(f(x_0)=0\)\newline Sind bei einer ganzrationalen Funktionen so viele Nullstellen wie der Grad gegeben, so empfiehlt sich die Produktform als Ansatz.}\newline\newline\\
	\hline
	\textbf{Symmetrieeigenschaften:}\newline 1. Achsensymmetrie zur\newline y-Achse\newline\newline 2. Punktsymmetrie zum\newline Ursprung& \phantom{x}\newline\textcolor{loes}{1. Bei einer ganzrationalen Funktion müssen alle Hochzahlen gerade oder Null sein.\newline\newline 2. Bei einer ganzrationalen Funktion müssen alle Hochzahlen ungerade sein.}\newline\newline\\
	\hline
	\textbf{Asymptote} \(b\) bzw.\newline\(mx+b\) für eine waagrechte oder schiefe Asymptote&\textcolor{loes}{Die Asymptote kann bei Exponentialfunktionen direkt in die Funktionsgleichung eingesetzt werden: \(f(x)=ae^{kx}+mx+b\)}\newline\newline\newline\newline
\end{tabular}
\newpage
%%%%%%%%%%%%%%%%%%%%%%%%%%%%%%%%%%%%%%%%%%%%%%%%%%%%%%%%%%%%%%%%%%%%%%%%%%%%%%%%%%%%%%%%%%%%%%%%%%%%%%%
\begin{Exercise}[title={\raggedright Bestimme jeweils die Funktionsgleichung.}, label=fktbestimmenA1]
	\begin{enumerate}[label=\alph*)]
		\item Das Schaubild der ganzrationalen Funktion \(f_1(x)\) vierten Grades hat den y-Achsenabschnitt 3, ist achsensymmetrisch zur y-Achse hat bei \(H(2|4)\) einen Hochpunkt.
		\item Das Schaubild der ganzrationalen Funktion \(f_2(x)\) dritten Grades berührt die x-Achse bei \(x=3\), schneidet die x-Achse bei \(x=-1\) und verläuft durch den Punkt \(P(1|4)\)
		\item Das Schaubild der ganzrationalen Funktion \(f_3(x)\) dritten Grades ist punktsymmetrisch zum Ursprung und hat den Tiefpunkt \(T(2|-8)\).
		\item Das Schaubild der ganzrationalen Funktion \(f_4(x)\) dritten Grades hat im Wendepunkt \newline\(W(0|-1)\) die Steigung -2 und eine Nullstelle bei \(x_0=3\).
%		\item \(f_5(x)\)
%		\item \(f_6(x)\)
%		\item \(f_7(x)\)
	\end{enumerate}
\end{Exercise}
%%%%%%%%%%%%%%%%%%%%%%%%%%%%%%%%%%%%%%%%%
\begin{Answer}[ref=fktbestimmenA1]
	\begin{enumerate}[label=\alph*)]
		\item \(f_1(x)=-\frac{1}{16}x^4+\frac{1}{2}x^2+3\)
		\item \(f_2(x)=0,5(x+1)(x-3)^2\)
		\item \(f_3(x)=0,5x^3-6x\)
		\item \(f_4(x)=\frac{7}{27}x^3-2x-1\)
%		\item \(f_5(x)=\)
%		\item \(f_6(x)=\)
%		\item \(f_7(x)=\)
	\end{enumerate}
\end{Answer}

