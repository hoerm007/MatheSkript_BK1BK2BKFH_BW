\cohead{\Large\textbf{Potenzregel}}
\fakesubsection{Potenzregel}
Auch mit Hilfe der Faktorregel müssen wir die Ableitung von Funktionen wie \(x^2,\ x^3,\ x^4,\dots\) mit Hilfe des Differentialquotienten bestimmen. Glücklicherweise können wir die Ableitung dieser Funktionen sehr einfach mit Hilfe der Potenzregel berechnen. Dazu bestimmen wir die Ableitung einer Funktion \(f(x)=x^n\):\\
\begin{minipage}[t]{\textwidth}
	\begin{minipage}{0.5\textwidth}
		\begin{align*}
			f(x)&=x^n\\
			f'(x)&=\textcolor{loes}{\lim\limits_{h\to 0}\frac{f(x+h)-f(x)}{h}}\\
			&\textcolor{loes}{=\lim\limits_{h\to 0}\frac{(x+h)^n-x^n}{h}}\\
			&\textcolor{loes}{=\lim\limits_{h\to 0}\frac{x^n+nx^{n-1}h+\mathcal{O}(h^2)-x^n}{h}}\\
			&\textcolor{loes}{=\lim\limits_{h\to 0}\frac{nx^{n-1}h+\mathcal{O}(h^2)}{h}}\\
			&\textcolor{loes}{=\lim\limits_{h\to 0}nx^{n-1}+\mathcal{O}(h)}\\
			&\textcolor{loes}{=nx^{n-1}}\\
		\end{align*}
	\end{minipage}
	\begin{minipage}{0.5\textwidth}
		\textcolor{loes}{Der Ausdruck \(\mathcal{O}(h^2)\) steht für eine Summe, bei der jeder Summand mindestens ein \(h^2\) beinhaltet. Der interessierte Leser sei an die Binomialkoeffizienten verwiesen. Kürzt man diese Summe mit \(h\), so bleibt eine Summe übrig, die in jedem Summanden mindestens ein \(h\) hat: \(\mathcal{O}(h)\). Diese Summe wird 0, wenn man \(h\) gegen 0 gehen lässt.}
	\end{minipage}
\end{minipage}\\
Wir haben also eine Regel gefunden um alle Funktionen vom Typ \(f(x)=x^n\) abzuleiten. Verbunden mit der Faktorregel ergibt sich:\\
\begin{minipage}[t]{\textwidth}
	\begin{minipage}{0.3\textwidth}
		\begin{align*}
			f(x)&=ax^n\\
			f'(x)&=\textcolor{loes}{anx^{n-1}}\\
		\end{align*}
	\end{minipage}
	\begin{minipage}{0.7\textwidth}
		\begin{tcolorbox}
			\phantom{text}\\
			\textcolor{loestc}{Potenzregel:\\
				Eine Potenzfunktion vom Typ \(f(x)=ax^n\) leitet man ab, indem man die Funktion mit der Hochzahl \(n\) multipliziert und dann die Hochzahl um 1 verringert: \(f'(x)=anx^{n-1}\)}\\
		\end{tcolorbox}
	\end{minipage}
\end{minipage}\\
%%%%%%%%%%%%%%%%%%%%%%%%%%%%%%%%%%%%%%%%%%%%%%%%%%%%%%%%%%%%%%%%%%%%%%%%%%%%%%%%%%%%%%%%%%%%%%%%%%%%%%
\begin{minipage}{\textwidth}
	\begin{Exercise}[title={\raggedright Berechne jeweils allgemein die Ableitung \(f'(x)\)}, label=potenzregelA1]
		\begin{minipage}{\textwidth}
			\begin{minipage}{0.49\textwidth}
				\begin{enumerate}[label=\alph*)]
					\item \(f_1(x)=x^3\)
					\item \(f_2(x)=x^4\)
					\item \(f_3(x)=x^5\)
					\item \(f_4(x)=-2x^3\)
					\item \(f_5(x)=5x^4\)
					\item \(f_6(x)=\frac{2}{3}x^6\)
					\item \(f_7(x)=\frac{1}{2}x^4\)
					\item \(f_8(x)=4x\)
				\end{enumerate}
			\end{minipage}
			\begin{minipage}{0.49\textwidth}
				\begin{enumerate}[label=\alph*)]
					\setcounter{enumi}{8}
					\item \(f_9(x)=0.5x^6\)
					\item \(f_{10}(x)=\frac{2}{3}x^9\)
					\item \(f_{11}(x)=\frac{3}{8}x^4\)
					\item \(f_{12}(x)=-3x^{11}\)
					\item \(f_{13}(x)=\frac{x^3}{6}\)
					\item \(f_{14}(x)=x^5\cdot 7\)
					\item \(f_{15}(x)=-0,2x^8\)
					\item \(f_{16}(x)=\frac{5}{99}x^{99}\)
				\end{enumerate}
			\end{minipage}
		\end{minipage}
	\end{Exercise}
\end{minipage}
%%%%%%%%%%%%%%%%%%%%%%%%%%%%%%%%%%%%%%%%%
\begin{Answer}[ref=potenzregelA1]\\
	\begin{minipage}{\textwidth}
		\begin{minipage}{0.49\textwidth}
			\begin{enumerate}[label=\alph*)]
				\item \(f_1'(x)=3x^2\)
				\item \(f_2'(x)=4x^3\)
				\item \(f_3'(x)=5x^4\)
				\item \(f_4'(x)=-6x^2\)
				\item \(f_5'(x)=20x^3\)
				\item \(f_6'(x)=4x^5\)
				\item \(f_7'(x)=2x^3\)
				\item \(f_8'(x)=4\)
			\end{enumerate}
		\end{minipage}
		\begin{minipage}{0.49\textwidth}
			\begin{enumerate}[label=\alph*)]
				\setcounter{enumi}{8}
				\item \(f_9'(x)=3x^5\)
				\item \(f_{10}'(x)=6x^8\)
				\item \(f_{11}'(x)=\frac{3}{2}x^3\)
				\item \(f_{12}'(x)=-33x^{10}\)
				\item \(f_{13}'(x)=\frac{1}{2}x^2\)
				\item \(f_{14}'(x)=35x^4\)
				\item \(f_{15}'(x)=-1,6x^7\)
				\item \(f_{16}'(x)=5x^{98}\)
			\end{enumerate}
		\end{minipage}
	\end{minipage}
\end{Answer}