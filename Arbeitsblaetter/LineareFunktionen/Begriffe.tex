\cohead{\Large\textbf{Begriffe}}
\fakesubsection{Begriffe}
\begin{tcolorbox}
	Funktion:\\ \\
	\textcolor{loestc}{Eine Funktion ist eine Zuordnung, die \textit{jedem} x-Wert \textit{genau einen} y-Wert zuordnet.\\
	}
\end{tcolorbox}
Beispiel: \(y=2x\)\\
\textcolor{loes}{Diese Funktion ordnet jedem x-Wert seinen doppelten Wert als y-Wert zu, d.h. der 1 wird die 2 zugeordnet, der 2 die 4, usw.}\\
\begin{tcolorbox}
	Schreibweise für Funktionen:\\ \\
	\textcolor{loestc}{Bisher wurden Funktionen z.B. als \(y=2x+1\) geschrieben.\\ \\
		Neue Schreibweise der Funktionsgleichung: \(f(x)=2x+1\)\\ \\
		\(f\): Name der Funktion, in den meisten Fällen \(f,\ g,\ h\) oder \(f_1,\ ,f_2,\ f_3\) usw.\\ \\
		\(x\): Laufvariable, wird bei uns so gut wie immer mit \(x\) bezeichnet.\\ \\
		\(2x+1\): Funktionsterm\\ \\
		Gesprochen als \(f\) von \(x\) gleich Zwei \(x\) Plus Eins.\\
	}
\end{tcolorbox}
Der Vorteil der neuen Schreibweise ist, dass man Funktionen an Hand des Namens unterscheiden kann. Zudem erlaubt sie uns für \(x\) Werte einzusetzen und die zugeordneten y-Werte allgemeiner aufzuschreiben:
\begin{tcolorbox}
	Einsetzen von Werten:\\ \\
	\textcolor{loestc}{Für \(x\) können nun verschiedene Werte eingesetzt werden:, z.B.:\\ \\
		\(x=5\): \(f(5)=2\cdot 5+1=11\)\\ \\
		Gesprochen: \(f\) von 5 ist gleich 11\\ \\
		\(f(5)\) bezeichnet man auch als Funktionswert von \(f\) an der Stelle 5.\\
		Grafisch bedeutet dies, dass das Schaubild von \(f(x)\) durch den Punkt \(P(5|11)\) verläuft.
	}
\end{tcolorbox}
\newpage
%%%%%%%%%%%%%%%%%%%%%%%%%%%%%%%%%%%%%%%%%%%%%%%%%%%%%%%%%%%%%%%%%%%%%%%%%%%%%%%%%%%%%%%%%%%%%%%%%%%%%%%
\begin{Exercise}[title={\raggedright Finde die passenden Paare gleichwertiger Aussagen.}, label=begriffeA1]\\
	\begin{minipage}{\textwidth}
		\adjustbox{valign=t}{\begin{minipage}{0.5\linewidth}
				\begin{enumerate}[label=\alph*)]
					\item Der Funktionswert an der Stelle 3 ist 4.
					\item \(f\) von 8 ist 0.
					\item \(g(x)=0,5x\)
					\item \(P(-5|1)\) liegt auf dem Schaubild der Funktion.
					\item Die Funktion ordnet jedem \(x\) das Dreifache des Wertes von \(x\) zu.
					\item Der Punkt \(P(-1|4)\) liegt auf dem Schaubild der Funktion.
					\item Die Funktion ordnet der 0 die 5 zu.
					\item \(f(4)=3\)
					\item Der Punkt \(Q(0|0)\) liegt auf dem Schaubild der Funktion.
				\end{enumerate}
		\end{minipage}}%
		\adjustbox{valign=t}{\begin{minipage}{0.5\linewidth}
				\begin{enumerate}[label=\arabic*)]
					\item \(f(-5)=1\)
					\item Das Schaubild der Funktion verläuft durch den Ursprung.
					\item An der Stelle 4 ist der Funktionswert 3
					\item \(h(-1)=4\)
					\item \(f_2(0)=5\)
					\item \(f(8)=0\)
					\item Die Funktionsgleichung lautet \(g(x)=3x\)
					\item \(f(3)=4\)
					\item Die Funktion ordnet jedem  \(x\) den halben Wert als Funktionswert zu.
				\end{enumerate}
		\end{minipage}}
	\end{minipage}
\end{Exercise}
%%%%%%%%%%%%%%%%%%%%%%%%%%%%%%%%%%%%%%%%%
\begin{Answer}[ref=begriffeA1]\\
	\begin{enumerate}[label=\alph*)]
		\item 8)
		\item 6)
		\item 9)
		\item 1)
		\item 7)
		\item 4)
		\item 5)
		\item 3)
		\item 2)
	\end{enumerate}
\end{Answer}