% !TeX root = ../../Skript.tex
\cohead{\Large\textbf{Einfaches Rechnen}}
\fakesubsection{Einfaches Rechnen}
Die vier Grundrechenarten sollten bereits bekannt sein:
\begin{tcolorbox}
	\begin{itemize}
		\item Addition

		\textcolor{loestc}{\(\underbrace{a+b}_{Summe}=c\)\\\(a\) und \(b\) bezeichnet man als Summanden.}

		\item Subtraktion

		\textcolor{loestc}{\(\underbrace{a-b}_{Differenz}=c\)}

		\item Multiplikation

		\textcolor{loestc}{\(\underbrace{a\cdot b}_{Produkt}=c$\\\(a\) und \(b\) bezeichnet man als Faktoren.}

		\item Division

		\textcolor{loestc}{\(\underbrace{a:b}_{Quotient}=c\)\\Wir sollten uns das Geteilt-Zeichen abgewöhnen und stattdessen Brüche verwenden: \(a:b=\frac{a}{b}\).}
	\end{itemize}
\end{tcolorbox}
Reihenfolge von Rechenoperationen:
\begin{enumerate}
	\item \textcolor{loes}{Klammern}
	\item \textcolor{loes}{Potenzen}
	\item \textcolor{loes}{Punktrechnungen (Mal und Geteilt)}
	\item \textcolor{loes}{Strichrechnungen (Plus und Minus)}
\end{enumerate}
Für die Addition und Multiplikation gilt jeweils das Kommutativgesetz, d.h. man kann die Reihenfolge der Summanden bzw. Faktoren vertauschen:
\begin{align*}
	\textcolor{loes}{a+b\ }&\textcolor{loes}{=b+a}\\
	\textcolor{loes}{a\cdot b\ }&\textcolor{loes}{=b\cdot a}
\end{align*}
Für die Addition und Multiplikation gilt jeweils das Assoziativgesetz, d.h. die Reihenfolge, in der drei Summanden bzw. Faktoren addiert bzw. multipliziert werden, spielt keine Rolle:
\begin{align*}
	\textcolor{loes}{\left( a+b\right) +c\ }&\textcolor{loes}{=a+\left( b+c\right) }\\
	\textcolor{loes}{\left( a\cdot b\right) \cdot c\ }&\textcolor{loes}{=a\cdot\left( b\cdot c\right) }
\end{align*}
Das Distributivgesetz verknüpft die Multiplikation und Addition:
\begin{align*}
	\textcolor{loes}{a\cdot\left( b+c\right)\ }&\textcolor{loes}{=a\cdot b+a\cdot c}\\
	\textcolor{loes}{\left( a+b\right) \cdot c\ }&\textcolor{loes}{=a\cdot c+b\cdot c }
\end{align*}
Für das Distributivgesetz lassen sich die Pluszeichen auch durch Minuszeichen sowie die Malzeichen durch Geteiltzeichen ersetzen.
\newpage
%%%%%%%%%%%%%%%%%%%%%%%%%%%%%%%%%%%%%%%%%%%%%%%%%%%%%%%%%%%%%%%%%%%%%
\begin{Exercise}[title={Berechne die folgenden Ausdrücke}, label=einfachesRechnenA1]

	\begin{minipage}{\textwidth}
		\begin{minipage}{0.49\textwidth}
			\begin{enumerate}[label=\alph*)]
				\item \(3\cdot 4-20+2\cdot 5=\)
				\item \(20:\left( 4\cdot 5-16\right) +6=\)
				\item \(\left( 2+5\right) \cdot \left( 6-9\right) =\)
				\item \(\left( 11-23\right) :\left( 2\cdot 5+2\right) =\)
				\item \(\left( 1+2\right) \cdot3\cdot\left( 4+5\right) =\)
				\item \(-2\left( -5-2\right) -14=\)
				\item \(\left( 10:\left( -5\right)\right)  :2=\)
				\item \(1+2+3\left( -3-2-1\right) +2\cdot 5=\)
				\item \(5\cdot 8+4-3\cdot \left( -4\right) =\)
				\item \(-3\left( -2+4\cdot 8-\left( 2+5\right) +8\right) =\)
				\item \(21:\left( 6-\left( 4-5\right) \right) =\)
				\item \(100:\left( 100:\left( 5\cdot 5-3\left( -5\cdot 5\right) \right) \right) =\)
				\item \(\left( 2+3-\left( 12:3-\left( -1\right) \right)\cdot 5 \right) =\)
			\end{enumerate}
		\end{minipage}
		\begin{minipage}{0.49\textwidth}
			\begin{enumerate}[label=\alph*)]
				\setcounter{enumi}{13}
				\item \(2\cdot\left(5-3\right) \left( 15-17\right) \left( 26:13\right) \left( -1\cdot 2\right) = \)
				\item \(-10\cdot 10+5-75+3\cdot\left( -5\right) =\)
				\item \(100:(2\cdot 5(2-12))=\)
				\item \(-2\cdot(-4\cdot (-3\cdot (-1\cdot (-2-1))))=\)
				\item \(-(1-(-2-(-3-8)))\cdot (-2)=\)
				\item \(2\cdot(1+(1+(1+1+(-4))))=\)
				\item \(-2\cdot 3\cdot ((8:6):4):2=\)
				\item \((6-8)\cdot(-4+5)\cdot(4-7)\cdot(8-6)=\)
				\item \(2+3\cdot 4-4\cdot 4-22:2+12:(-4)=\)
				\item \(4\cdot(6\cdot(4\cdot(4:2):8):3):8=\)
				\item \(-(-2-(4-(5-(-5+6)+4)+3)+8)+10=\)
				\item \(-(-2\cdot 4\cdot (8-4):8+10\cdot (4-5))=\)
				\item \(1+2\cdot 3-4:2+5\cdot 3-10:2=\)
			\end{enumerate}
		\end{minipage}
	\end{minipage}
\end{Exercise}
\newpage
%%%%%%%%%%%%%%%%%%%%%%%%%%%%%%%%%%%%%%%%%
\begin{Answer}[ref=einfachesRechnenA1]

	\begin{minipage}{\textwidth}
		\begin{minipage}{0.44\textwidth}
			\begin{enumerate}[label=\alph*)]
				\item \(3\cdot 4-20+2\cdot 5=2\)
				\item \(20:\left( 4\cdot 5-16\right) +6=11\)
				\item \(\left( 2+5\right) \cdot \left( 6-9\right) =-21\)
				\item \(\left( 11-23\right) :\left( 2\cdot 5+1\right) =-1\)
				\item \(\left( 1+2\right) \cdot3\cdot\left( 4+5\right) =81\)
				\item \(-2\left( -5-2\right) -14=0\)
				\item \(\left( 10:\left( -5\right)\right)  :2=-1\)
				\item \(1+2+3\left( -3-2-1\right) +2\cdot 5=-5\)
				\item \(5\cdot 8+4-3\cdot \left( -4\right) =56\)
				\item \(-3\left( -2+4\cdot 8-\left( 2+5\right) +8\right) =-93\)
				\item \(21:\left( 6-\left( 4-5\right) \right) =3\)
				\item \(100:\left( 100:\left( 5\cdot 5-3\left( -5\cdot 5\right) \right) \right) =100\)
				\item \(\left( 2+3-\left( 12:3-\left( -1\right) \right)\cdot 5 \right) =-20\)
			\end{enumerate}
		\end{minipage}
		\begin{minipage}{0.54\textwidth}
			\begin{enumerate}[label=\alph*)]
				\setcounter{enumi}{13}
				\item \(2\cdot\left(5-3\right) \left( 15-17\right) \left( 26:13\right) \left( -1\cdot 2\right) = 32\)
				\item \(-10\cdot 10+5-75+3\cdot\left( -5\right) =-185\)
				\item \(100:(2\cdot 5(2-12))=-1\)
				\item \(-2\cdot(-4\cdot (-3\cdot (-1\cdot (-2-1))))=-72\)
				\item \(-(1-(-2-(-3-8)))\cdot (-2)=-16\)
				\item \(2\cdot(1+(1+(1+1+(-4))))=0\)
				\item \(-2\cdot 3\cdot ((8:6):4):2=-1\)
				\item \((6-8)\cdot(-4+5)\cdot(4-7)\cdot(8-6)=12\)
				\item \(2+3\cdot 4-4\cdot 4-22:2+12:(-4)=-7\)
				\item \(4\cdot(6\cdot(4\cdot(4:2):8):3):8=1\)
				\item \(-(-2-(4-(5-(-5+6)+4)+3)+8)+10=3\)
				\item \(-(-2\cdot 4\cdot (8-4):8+10\cdot (4-5))=14\)
				\item \(1+2\cdot 3-4:2+5\cdot 3-10:2=15\)
			\end{enumerate}
		\end{minipage}
	\end{minipage}
\end{Answer}
\newpage
%%%%%%%%%%%%%%%%%%%%%%%%%%%%%%%%%%%%%%%%%%%%%%%%%%%%%%%%%%%%%%
\cohead{\Large\textbf{Bruchrechnen}}
\begin{tcolorbox}
	Um zwei Brüche zu Addieren/Subtrahieren, müssen zuerst beide Brüche auf den gleichen Nenner gebracht werden (Hauptnenner) und dann die Zähler addiert/subtrahiert werden.
\end{tcolorbox}
\begin{bsp}
	\begin{align*}
		&\frac{1}{2}+\frac{2}{3}=\textcolor{loes}{\frac{3}{6}+\frac{4}{6}=\frac{3+4}{6}=\frac{7}{6}}\\
		&\frac{7}{4}+\frac{3}{10}=\textcolor{loes}{\frac{35}{20}+\frac{6}{20}=\frac{35+6}{20}=\frac{41}{20}}\\
		&\frac{5}{6}-\frac{8}{15}=\textcolor{loes}{\frac{25}{30}-\frac{16}{30}=\frac{\cancel{9}^3}{\cancel{10}^{10}}=\frac{3}{10}}
	\end{align*}
\end{bsp}

\begin{tcolorbox}
	Um zwei Brüche zu Multiplizieren, werden die Zähler miteinander multipliziert und die Nenner miteinander multipliziert. Innerhalb eines Produkts darf direkt gekürzt werden.
\end{tcolorbox}
\begin{bsp}
	\begin{align*}
		&\frac{1}{2}\cdot\frac{1}{3}=\textcolor{loes}{\frac{1\cdot 1}{2\cdot 3}=\frac{1}{6}}\\
		&\frac{1}{2}\cdot\frac{2}{3}=\textcolor{loes}{\frac{1\cdot 2}{2\cdot 3}=\frac{1\cdot \cancel{2}}{\cancel{2}\cdot 3}=\frac{1}{3}}\\
		&\frac{4}{7}\cdot\frac{21}{6}=\textcolor{loes}{\frac{\cancel{4}^2\cdot \cancel{21}^3}{\cancel{7}^1\cdot \cancel{6}^3}=\frac{2\cdot\cancel{3}^1}{1\cdot\cancel{3}^1}=2}
	\end{align*}
\end{bsp}

\begin{tcolorbox}
	Zwei Brüche werden dividiert, indem mit dem Kehrwert multipliziert wird.
\end{tcolorbox}
\begin{bsp}
	\begin{align*}
		&\frac{1}{2}:\frac{5}{3}=\textcolor{loes}{\frac{1}{2}\cdot\frac{3}{5}=\frac{3}{10}}\\
		&\frac{15}{2}:\frac{21}{4}=\textcolor{loes}{\frac{15}{2}\cdot\frac{4}{21}=\frac{\cancel{15}^5\cdot\cancel{4}^2}{\cancel{21}^7\cdot \cancel{3}}=\frac{10}{7}}\\
		&\frac{7}{30}:\frac{21}{10}=\textcolor{loes}{\frac{7}{30}\cdot\frac{10}{21}=\frac{\cancel{7}^1\cdot\cancel{10}^1}{\cancel{30}^3\cdot\cancel{21}^3}=\frac{1}{9}}
	\end{align*}
\end{bsp}
\newpage
%%%%%%%%%%%%%%%%%%%%%%%%%%%%%%%%%%%%%%%%%%%%%%%%%%%%%%%%%%%%%%%%%%%%%
\cohead{\Large\textbf{Kürzen von Brüchen}}
Als Primzahlen bezeichnet man die natürlichen Zahlen größer 1, die nur durch 1 und sich selbst ohne Rest teilbar sind. Die erste Primzahl ist also 2, da 2 nur durch 1 und sich selbst ohne Rest teilbar ist. Die nächste Primzahl ist 3. 4 ist keine Primzahl, da \(4:2=2\) gilt. Die für uns wichtigen Primzahlen sind:
\begin{tcolorbox}\centering
	\(\textcolor{loestc}{\mathbb{P}=\{2,\ 3,\ 5,\ 7,\ 11,\dots\}}\)
\end{tcolorbox}
Haben der Zähler und der Nenner eines Bruches einen gemeinsamen Teiler, so kann man den Bruch kürzen. Dabei muss man nur prüfen, ob die Primzahlen jeweils ein Teiler sind. Ist eine Zahl nicht durch 2 teilbar, so kann sie nicht durch \(4,\ 6,\ 8,\ ,\dots\) teilbar sein. Für die ersten drei Primzahlen gibt es dabei einfach zu prüfende Teilbarkeitsregeln:
\begin{tcolorbox}\centering
	\textcolor{loestc}{Eine ganze Zahl ist genau dann durch 2 teilbar, wenn sie gerade ist, d.h. die letzte Ziffer ist eine \(0,\ 2,\ 4,\ 6,\) oder  \(8\).}
\end{tcolorbox}
\begin{tcolorbox}\centering
	\textcolor{loestc}{Eine ganze Zahl ist genau dann durch 3 teilbar, wenn die Quersumme (Summe aller Ziffern) durch 3 teilbar ist, z.B. ist 123 durch 3 teilbar, da 1+2+3=6 durch 3 teilbar ist. 563 ist nicht durch 3 teilbar, da 5+6+3=14 nicht durch 3 teilbar ist.}
\end{tcolorbox}
\begin{tcolorbox}\centering
	\textcolor{loestc}{Eine ganze Zahl ist genau dann durch 5 teilbar, wenn die letzte Ziffer eine 0 oder 5 ist.}
\end{tcolorbox}
Beim Kürzen prüft man nun einfach, ob Zähler und Nenner durch 2 teilbar sind. Falls ja, teilt man beide durch 2 und prüft nochmals, bis mindestens einer von beiden nicht mehr durch 2 teilbar ist. Dann führt man das gleiche Verfahren für 3, 5, 7, usw. durch. Dabei muss man sich natürlich nicht fest an diese Reihenfolge halten. Enden z.B. Zähler und Nenner jeweils auf eine 0, so kann man beide direkt mit 10 kürzen.
\begin{bsp}
	\begin{align*}
		&\frac{72}{60}=\textcolor{loes}{\frac{36}{30}=\frac{18}{15}=\frac{12}{15}=\frac{4}{5}}\\
		&\frac{280}{700}=\textcolor{loes}{\frac{28}{70}=\frac{14}{35}=\frac{2}{5}}\\
		&\frac{300}{126}=\textcolor{loes}{\frac{150}{63}=\frac{50}{21}}
	\end{align*}
\end{bsp}
Man kann Brüche auch kürzen, bevor man Zähler und Nenner komplett zusammengefasst hat. Dazu muss man jeweils die gleiche Zahl im Zähler und Nenner ausklammern können:
\begin{bsp}
	\begin{align*}
		&\frac{4+8}{14}=\textcolor{loes}{\frac{\cancel{2}(2+4)}{\cancel{2}\cdot 7}=\frac{2+4}{7}=\frac{6}{7}}\\
		&\frac{6-9}{3+15}=\textcolor{loes}{\frac{\cancel{3}(2-3)}{\cancel{3}(1+5)}=\frac{2-3}{1+5}=-\frac{1}{6}}\\
		&\frac{5x^2+10x-25}{30}=\textcolor{loes}{\frac{\cancel{5}(x^2+2x-5)}{\cancel{5}\cdot 6}=\frac{x^2+2x-5}{6}}
	\end{align*}
\end{bsp}
\newpage
%%%%%%%%%%%%%%%%%%%%%%%%%%%%%%%%%%%%%%%%%%%%%%%%%%%%
\begin{Exercise}[title={Berechne die folgenden Ausdrücke und kürze soweit wie möglich}, label=bruecheA1]

	\begin{minipage}{\textwidth}
		\begin{minipage}{0.49\textwidth}
			\begin{enumerate}[label=\alph*)]
				\item \(\frac{2}{3}+\frac{5}{7}=\)
				\item \(\frac{3}{4}-\frac{10}{3}=\)
				\item \(\frac{11}{25}+\frac{3}{5}=\)
				\item \(\frac{14}{15}-\frac{5}{6}=\)
				\item \(\frac{14}{9}+\frac{7}{18}=\)
				\item \(\frac{14}{15}\cdot\frac{5}{28}=\)
				\item \(\frac{30}{77}\cdot\frac{49}{24}=\)
				\item \(\frac{5}{28}\cdot\frac{8}{7}=\)
				\item \(\frac{12}{25}\cdot\frac{15}{16}=\)
				\item \(\frac{13}{42}:\frac{39}{56}=\)
				\item \(\frac{14}{17}:\frac{28}{5}=\)
				\item \(\frac{9}{16}:\frac{27}{4}=\)
				\item \(\frac{14}{30}:\frac{35}{2}=\)
			\end{enumerate}
		\end{minipage}
		\begin{minipage}{0.49\textwidth}
			\begin{enumerate}[label=\alph*)]
				\setcounter{enumi}{13}
				\item \(\frac{15}{16}\cdot\frac{56}{25}\cdot\frac{15}{28}=\)
				\item \(\left( \frac{27}{14}+\frac{9}{14}\right) \cdot \frac{14}{9}=\)
				\item \(\frac{3}{2}-\left( \frac{5}{4}+\frac{1}{8}\right) =\)
				\item \(\frac{10}{7}\cdot\left( \frac{2}{5}-\frac{3}{4}\right) =\)
				\item \(\frac{5}{2}-\frac{4}{3}+\frac{7}{6}=\)
				\item \(\frac{17}{3}-\left( \frac{15}{4}:\frac{5}{8}\right) =\)
				\item \(\frac{34}{27}:\left( \frac{5}{9}+\frac{4}{3}\right) =\)
				\item \(\frac{2}{3}-\frac{12}{25}:\frac{36}{35}=\)
				\item \(\frac{64}{81}\cdot\frac{63}{80}+\frac{5}{9}=\)
				\item \(\frac{1}{5}-\left( \frac{2}{3}\cdot \frac{7}{4}\cdot \frac{6}{35}\right) =\)
				\item \(\left( \frac{42}{33}\cdot\frac{11}{35}\right):\left(\frac{84}{55}\cdot\frac{11}{42}\right)  =\)
				\item \(\frac{9}{70}\cdot\frac{10}{63}+\frac{5}{7}=\)
				\item \(\frac{15}{14}:\frac{45}{28}-\frac{27}{8}:\frac{9}{4}=\)
			\end{enumerate}
		\end{minipage}
	\end{minipage}
\end{Exercise}
\newpage
%%%%%%%%%%%%%%%%%%%%%%%%%%%%%%%%%%%%%%%%%
\begin{Answer}[ref=bruecheA1]

	\begin{minipage}{\textwidth}
		\begin{minipage}{0.44\textwidth}
			\begin{enumerate}[label=\alph*)]
				\item \(\frac{29}{21}\)
				\item \(-\frac{31}{12}\)
				\item \(\frac{26}{25}\)
				\item \(\frac{1}{10}\)
				\item \(\frac{35}{18}\)
				\item \(\frac{1}{6}\)
				\item \(\frac{35}{44}\)
				\item \(\frac{10}{49}\)
				\item \(\frac{9}{20}\)
				\item \(\frac{4}{9}\)
				\item \(\frac{5}{34}\)
				\item \(\frac{1}{12}\)
				\item \(\frac{2}{75}\)
			\end{enumerate}
		\end{minipage}
		\begin{minipage}{0.54\textwidth}
			\begin{enumerate}[label=\alph*)]
				\setcounter{enumi}{13}
				\item \(\frac{9}{8}\)
				\item \(4\)
				\item \(\frac{1}{8}\)
				\item \(-\frac{1}{2}\)
				\item \(\frac{7}{3}\)
				\item \(-\frac{29}{7}\)
				\item \(\frac{17}{21}\)
				\item \(\frac{1}{5}\)
				\item \(\frac{53}{45}\)
				\item \(0\)
				\item \(1\)
				\item \(\frac{36}{49}\)
				\item \(-\frac{5}{6}\)
			\end{enumerate}
		\end{minipage}
	\end{minipage}
\end{Answer}
\newpage
%%%%%%%%%%%%%%%%%%%%%%%%%%%%%%%%%%%%%%%%%%%%%%%%%%%%%%%%%%%%%%

\cohead{\Large\textbf{Rechnen mit Variablen}}
Variablen sind in der Mathematik Platzhalter für Zahlen, deren Wert man nicht kennt. Mit ihrer Hilfe kann man allgemeine Zusammenhänge aufstellen, z.B. lautet der Zusammenhang zwischen der Fläche eines Rechtecks und seinen Seitenlängen:
\begin{tcolorbox}
	Flächeninhalt eines Rechtecks
	\begin{align*}
		\textcolor{loestc}{A}&\textcolor{loestc}{=a\cdot b}\\
		\textcolor{loestc}{A}&\textcolor{loestc}{: \text{Fläche des Rechtecks}}\\
		\textcolor{loestc}{a, b}&\textcolor{loestc}{: \text{Seitenlängen des Rechtecks}}
	\end{align*}
\end{tcolorbox}
Kennt man zwei der drei Größen, kann man die fehlende berechnen.\\
Für uns ist nur eines der Potenzgesetze relevant:
\begin{tcolorbox}
	Zwei Potenzen mit der gleichen Basis werden multipliziert, indem man die Hochzahlen addiert:
	\begin{align*}
		\textcolor{loestc}{x^a\cdot x^b=x^{a+b}}
	\end{align*}
\end{tcolorbox}
\begin{bsp}
	\begin{align*}
		&x\cdot x^2 \cdot x^3=\textcolor{loes}{x^{1+2}\cdot x^3=x^3\cdot x^3=x^{3+3}=x^6}\\
		&x^2(3x^3+4x^2-x)=\textcolor{loes}{3x^{2+3}+4x^{2+2}-x^{1+2}=3x^5+4x^4-x^3}\\
		&x(2x^3-4x^2+2x)-2x^2(x^2+5x+1)=\textcolor{loes}{2x^4-4x^3+2x^2-2x^4-10x^3-2x^2=-16x^3}
	\end{align*}
\end{bsp}
Ausklammern oder Vorklammern kann man Zahlen oder auch Variablen. Beim Ausklammern ändert man den Wert des mathematischen Ausdrucks nicht, sondern lediglich sein Aussehen. Klammert man Variablen aus (im Normalfall \(x\)), so ist es in den meisten Fällen nicht sinnvoll die Variable öfter als die kleinste Hochzahl auszuklammern, da dann die Variable im Nenner des Bruches stehen würde. Beim Ausklammern von Variablen wendet man das obige Potenzgesetz rückwärts an.
\begin{bsp}
	\begin{align*}
		&2x^2-4x=\textcolor{loes}{x(2x-4)=2x(x-2)}\\
		&10x^3-5x^2+25x=\textcolor{loes}{x(10x^2-5x+25)=5x(2x^2-x+5)}\\
		&27x^4-18x^2=\textcolor{loes}{x^2(27x^2-18)=9x^2(3x^2-2)}
	\end{align*}
\end{bsp}
Wir müssen im Normalfall nur so viele \(x\) wie möglich vorklammern ohne eine zusätzliche Zahl.
\newpage
%%%%%%%%%%%%%%%%%%%%%%%%%%%%%%%%%%%%%%%%%%%%%%%%%%%%%%%%%%%%%%%%%%%%%%%%%%%%%
\begin{Exercise}[title={Löse die Klammern auf und fasse soweit wie möglich zusammen}, label=aufloesenA1]

	\begin{minipage}{\textwidth}
		\begin{minipage}{0.44\textwidth}
			\begin{enumerate}[label=\alph*)]
				\item \(x\left( x-2\right) =\)
				\item \(2x\left( x^2-3x+5\right) =\)
				\item \(-4x\left( 2x^2-6\right) =\)
				\item \(x^2\left( -3x+5\right) =\)
				\item \(x^3-7x^2\left( x+1\right) =\)
				\item \(\frac{2}{3}x\left(6x^2-3x+5\right) =\)
				\item \(-\frac{4}{7}x-\frac{3}{2}\left(\frac{4}{5}x^2-\frac{8}{21}x+9\right) =\)
				\item \(\frac{4}{9}x^3\left(x^2-81x+27\right) =\)
				\item \(\left(2x-4\right) \left(-\frac{3}{4}x^2+\frac{7}{8}x\right) =\)
				\item \(\left(x^2-\frac{2}{3}\right) ^2=\)
				\item \(-10x\left(\frac{4}{5}x^2-\frac{8}{15}x\right) =\)
				\item \(\frac{5}{6}x^3\left(-\frac{7}{15}x^2+2x\right) =\)
				\item \(\left(\frac{5}{3}x+\frac{10}{3}\right)\left(-\frac{6}{5}x^2-\frac{9}{10}x \right) =\)
			\end{enumerate}
		\end{minipage}
		\begin{minipage}{0.54\textwidth}
			\begin{enumerate}[label=\alph*)]
				\setcounter{enumi}{13}
				\item \(x\left(-2x^4+3x^3-2x^2+5 \right)+x^5-3x^4+5 =\)
				\item \(\frac{4}{3}x^2\left( -3x^2+6x-2\right) +\left( \frac{1}{2}x\right) ^2=\)
				\item \(\frac{1}{4}x^3\left(-\frac{8}{3}x-6 \right) =\)
				\item \(\frac{2}{5}x\left(-\frac{15}{8}x^2+10x-\frac{15}{4} \right) =\)
				\item \(-\frac{4}{35}x^3\left(-\frac{15}{8}x- \frac{5}{8}\right) =\)
				\item \(-\frac{8}{15}x^4\left(\frac{9}{4}x^2- x\right) =\)
				\item \(-\frac{7}{8}x\left(\frac{64}{49}x^3+ 4x^2-8x\right) =\)
				\item \(\frac{14}{15}x^2\left(-\frac{3}{28}x+ \frac{30}{7}x^2\right) =\)
				\item \(\frac{5}{7}x\left(-\frac{7}{5}x^2-x \right)^2-\frac{7}{5}x^5-\frac{3}{7}x^3 =\)
				\item \(-\frac{22}{9}x^2\left(-\frac{5}{11}x+3 \right)^2 +\frac{3}{11}x^4+20x^2=\)
				\item \(-\frac{20}{21}x^3\left(\frac{3}{4}x^4-3x \right)^2+x^{10} =\)
				\item \(-\frac{7}{3}x^5\left(\frac{18}{35}x+ 6x^2\right) =\)
				\item \(\frac{15}{14}x^3\left(-\frac{42}{35}x^3-7x^2+\frac{28}{5} \right)-x\left(-\frac{15}{2}x^4-6x^3\right)=\)
			\end{enumerate}
		\end{minipage}
	\end{minipage}
\end{Exercise}
%%%%%%%%%%%%%%%%%%%%%%%%%%%%%%%%%%%%%%%%%%%%%%%%%%%%%%%%%%%%%%%%%%%%%%%%%%%
\begin{Exercise}[title={Klammere so viele \(x\) wie möglich vor (ohne, dass \(x\) im Nenner eines Bruches benötigt wird)}, label=ausklammernA1]

	\begin{minipage}{\textwidth}
		\begin{minipage}{0.49\textwidth}
			\begin{enumerate}[label=\alph*)]
				\item \(3x^2-4x=\)
				\item \(-x^2+3x=\)
				\item \(7x^3+3x^2=\)
				\item \(10x^3-5x=\)
				\item \(x^4-x^2=\)
				\item \(8x^4-5x^3=\)
				\item \(3x^4+2x^3-x^2=\)
				\item \(4x^4+x=\)
				\item \(\frac{1}{3}x^4+x^3=\)
				\item \(-\frac{2}{5}x^5-\frac{2}{3}x^4=\)
				\item \(\frac{2}{5}x^6-8x^3=\)
				\item \(x^4-2x^5+x^6=\)
				\item \(9x^2-5x+4x^3=\)
			\end{enumerate}
		\end{minipage}
		\begin{minipage}{0.49\textwidth}
			\begin{enumerate}[label=\alph*)]
				\setcounter{enumi}{13}
				\item \(3x^7-2x^4+x^2=\)
				\item \(8x^3+8x=\)
				\item \(\frac{13}{3}x^3-\frac{3}{2}x^2=\)
				\item \(4x^3-x^2=\)
				\item \(8x^8-3x^4+x^5=\)
				\item \(3x+7x^4-8x^6=\)
				\item \(4x^5-3x^3+x^7=\)
				\item \(\frac{4}{7}x^3+\frac{8}{9}x^4=\)
				\item \(x\left( 2x^2+3\right) -4x^3=\)
				\item \(x^3-\left( 3x+4x^2\right)=\)
				\item \(\left(-2x^4\right)^2 -\left(3x^2-x\right)^2 =\)
				\item \(x^2\left(3x^4+5x^2\right) =\)
				\item \(x\left(4x^2+5x\right) =\)
			\end{enumerate}
		\end{minipage}
	\end{minipage}
\end{Exercise}
\newpage
%%%%%%%%%%%%%%%%%%%%%%%%%%%%%%%%%%%%%%%%%
\begin{Answer}[ref=aufloesenA1]

	\begin{minipage}{\textwidth}
		\begin{minipage}{0.44\textwidth}
			\begin{enumerate}[label=\alph*)]
				\item \(x^2-2x\)
				\item \(2x^3-6x^2+10x\)
				\item \(-8x^3+24x\)
				\item \(-3x^3+5x^2\)
				\item \(-6x^3-7x^2\)
				\item \(4x^3-2x^2+\frac{10}{3}x\)
				\item \(-6\frac{6}{5}x^2-\frac{27}{2}\)
				\item \(\frac{4}{9}x^5-9x^4+12x^3\)
				\item \(-\frac{3}{2}x^3+\frac{19}{4}x^2-\frac{7}{2}x\)
				\item \(x^4-\frac{4}{3}x^2+\frac{4}{9}\)
				\item \(-8x^3+\frac{16}{3}x^2\)
				\item \(-\frac{7}{18}x^5+\frac{5}{3}x^4\)
				\item \(-2x^3-\frac{11}{2}x^2-3x\)
			\end{enumerate}
		\end{minipage}
		\begin{minipage}{0.54\textwidth}
			\begin{enumerate}[label=\alph*)]
				\setcounter{enumi}{13}
				\item \(-x^5-2x^3+5x+5\)
				\item \(-4x^4+8x^3-\frac{29}{12}x^2\)
				\item \(-\frac{2}{3}x^4-\frac{3}{2}x^3\)
				\item \(-\frac{3}{4}x^3+4x^2-\frac{3}{2}x\)
				\item \(\frac{3}{14}x^4+\frac{1}{14}x^3\)
				\item \(-\frac{6}{5}x^6+\frac{8}{15}x^5\)
				\item \(-\frac{8}{7}x^4-\frac{7}{2}x^3+7x^2\)
				\item \(4x^4-\frac{1}{10}x^3\)
				\item \(\frac{56}{5}x^5+2x^4+\frac{2}{7}x^3\)
				\item \(-\frac{23}{99}x^4+\frac{20}{3}x^3-2x^2\)
				\item \(-\frac{15}{28}x^{11}+x^{10}+\frac{30}{7}x^8-\frac{60}{7}x^5\)
				\item \(-14x^7-\frac{6}{5}x^6\)
				\item \(-\frac{9}{7}x^6+\frac{75}{14}x^5+6x^4+6x^3\)
			\end{enumerate}
		\end{minipage}
	\end{minipage}
\end{Answer}
\begin{Answer}[ref=ausklammernA1]

	\begin{minipage}{\textwidth}
		\begin{minipage}{0.44\textwidth}
			\begin{enumerate}[label=\alph*)]
				\item \(x\left( 3x-4\right)\)
				\item \(x\left(-x+3\right)\)
				\item \(x^2\left(7x+3\right)\)
				\item \(x\left(10x^2-5\right)\)
				\item \(x^2\left(x^2-1\right)\)
				\item \(x^3\left(8x-5\right)\)
				\item \(x^2\left(3x^2+2x-1\right)\)
				\item \(x\left(4x^3+1\right)\)
				\item \(x^3\left(\frac{1}{3}x+1\right)\)
				\item \(x^4\left(-\frac{2}{5}x-\frac{2}{3}\right)\)
				\item \(x^3\left(\frac{2}{5}x^3-8\right)\)
				\item \(x^4\left(1-2x+x^2\right)\)
				\item \(x^2\left(9-5x+4x\right)\)
			\end{enumerate}
		\end{minipage}
		\begin{minipage}{0.54\textwidth}
			\begin{enumerate}[label=\alph*)]
				\setcounter{enumi}{13}
				\item \(x^2\left(3x^5-2x^2+1\right)\)
				\item \(x\left(8x^2+8\right)\)
				\item \(x^2\left(\frac{13}{3}x-\frac{3}{2}\right)\)
				\item \(x^2\left(4x-1\right)\)
				\item \(x^4\left(8x^4-3+x\right)\)
				\item \(x\left(3+7x^3-8x^5\right)\)
				\item \(x^3\left(4x^2-3+x^4\right)\)
				\item \(x^3\left(\frac{4}{7}+\frac{8}{9}x\right)\)
				\item \(x\left(-2x^2+3\right)\)
				\item \(x\left(x^2-3-4x\right)\)
				\item \(x^2\left(4x^6-9x^2+6x-1\right)\)
				\item \(x^4\left(3x^2+5\right)\)
				\item \(x^2\left(4x+5\right)\)
			\end{enumerate}
		\end{minipage}
	\end{minipage}
\end{Answer}
\newpage
%%%%%%%%%%%%%%%%%%%%%%%%%%%%%%%%%%%%%%%%%%%%%%%%%%%%%%%%%%%%%%%%%%%%%%%%%%%
\cohead{\Large\textbf{Häufige Fehler}}
\begin{enumerate}[label=\arabic*)]
	\item Brüche im Quadrat: \(\left(\frac{2}{3}\right) ^2=\frac{2^2}{3^2}=\frac{4}{9}\neq\frac{2}{3}^2=\frac{4}{3}\)

	Der Bruch ist eine andere Schreibweise für ein Geteilt-Zeichen. Da zuerst Potenzen, dann Punktrechnungen durchgeführt werden, wird bei einem Bruch ohne Klammer nur der Zähler potenziert.

	\item Quadrat von negativen Zahlen: \(\left( -2\right) ^2=\left( -2\right) \cdot\left( -2\right) =4\neq-2^2=-4\)

	Zuerst werden Potenzen, dann Strichrechnungen durchgeführt, d.h. wenn man die Klammer weglässt, wird die Zahl zuerst potenziert und dann das Minuszeichen hinzugefügt.

	\item Rechnen mit Dezimalzahlen statt Brüchen: In den allermeisten Fällen ist es einfacher mit Brüchen zu rechnen, so lässt sich z.B. folgende Wurzel in Dezimalzahlen nur schwer berechnen, als Bruch dagegen ist die Rechnung simpel:

	\(\sqrt{12,25\vphantom{A^2}}=\sqrt{\frac{49}{4}}=\frac{\sqrt{49}}{\sqrt{4}}=\frac{7}{2}\)
\end{enumerate}