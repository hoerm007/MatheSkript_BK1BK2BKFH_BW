\documentclass[a4paper,12pt, headsepline, ngerman]{scrartcl}

\usepackage{scrlayer-scrpage}
\usepackage[nodisplayskipstretch]{setspace} %vspace before/after math mode
%\setstretch{5}
\raggedbottom
\onehalfspacing
%\renewcommand*{\familydefault}{\sfdefault}



\pagestyle{scrheadings} %KOMA-Script mit Kopf-Fuß-Zeilen
\usepackage{hyperref}			%Hyperlinks setzen
\usepackage{babel}				%Silbentrennung mit ngerman

\usepackage{booktabs} 			% For prettier tables

\usepackage{mathtools}  		%Mathe-Paket
\usepackage{amssymb}			%Symbole
\usepackage{bbm}				%\mathbbm{N} für natürliche Zahlen o.ä.
%\usepackage{graphicx}			%Optionen für \includegraphics{imagefile}
%Solte in mathtools beinhaltet sein
\usepackage{color}				%\textcolor{blue}{text...}
\usepackage[dvipsnames]{xcolor}
%Häufig verwendetet Farben
%ForestGreen 		RGB(  0,155, 85)
%YellowOrange		RGB(250,162, 26)

\usepackage[T1]{fontenc}		%Umlaute




\usepackage{ulem}
%\uline{important} underlined text
%\uuline{urgent} double-underlined text
%\uwave{boat} wavy underline
%\sout{wrong} line struck through word
%\xout{removed} marked over like //////////
%\dashuline{dashing} dashed underline
%\dotuline{dotty} dotted underline
\usepackage{cancel}					%Durchstreichen von Dingen in Formeln
\usepackage{enumitem}				%Aufzählungen [label=\alph*)]
\setlist[enumerate]{nosep, topsep=0pt}	%Kleinere Abstände bei Aufzählungen
\setlist[itemize]{noitemsep, topsep=0pt}

\usepackage{framed}                 %Rahmen machen \begin{framed} ... \end{framed}

\usepackage{tcolorbox} 				%Für Boxen um Text
\usepackage{array}
%Plots
\usepackage{tikz}
\usepackage{pgf}
\usepackage{pgfmath}

\usepackage{bm}						%\bm{xxx} for bold in math mode

\usepackage{geometry}
\geometry{a4paper, portrait, left=1.5cm, right=2cm, top=1cm, bottom=2cm, headsep=0.2cm, includehead, head=27.30193pt}

\usepackage{marvosym} %\Lightning
\usepackage{multirow}
\renewcommand{\mvchr}[1]{\Large{\mbox{\mvs\symbol{#1}}}} %\Lightning in math mode

\usetikzlibrary{intersections}

\setkomafont{headsepline}{\color{black}}

\usepackage{amsthm}					%Definitionsumgebung für \newtheorem{defi}{Definition}[section] usw.
\theoremstyle{definition}
\newtheorem{defi}{Definition}[subsection]
\newtheorem*{bsp}{Beispiel}
\newtheorem{kon}[defi]{Konstruktion}
\newtheorem{nota}[defi]{Notation}
\newtheorem{cha}[defi]{Charakterisierung}
\newtheorem{norm}[defi]{Normierung}
\newtheorem{bem}[defi]{Bemerkung}
\newtheorem{folg}[defi]{Folgerung}
\newtheorem{beob}[defi]{Beobachtung}
\newtheorem{erin}[defi]{Erinnerung}
\newtheorem{sit}[defi]{Situation}
%Einheiten
\newcommand{\ms}{\frac{m}{s}}
\newcommand{\kmh}{\frac{km}{h}}
%Mathebefehle
\newcommand{\beq}{\begin{align}}
\newcommand{\eeq}{\end{align}}
\newcommand{\beqn}{\begin{align*}}
\newcommand{\eeqn}{\end{align*}}
\newcommand{\td}{\text{d}}
\newcommand{\ul}{\underline}
\newcommand{\tTr}{\text{Tr}}
\newcommand{\bra}[1]{\langle #1|}
\newcommand{\braa}{\langle}
\newcommand{\ket}[1]{|#1\rangle}
\newcommand{\kett}{\rangle}
\newcommand{\braket}[2]{\langle #1|#2\rangle}
\newcommand{\mH}{\mathcal{H}}
\newcommand{\R}{\mathbb{R}}
\newcommand{\N}{\mathbb{N}}
\newcommand{\Z}{\mathbb{Z}}
\newcommand{\Q}{\mathbb{Q}}

%Exercise-Paket Umbenennungen
\usepackage[answerdelayed]{exercise}			%Nach hyperref einbinden! answerdelayed
\renewcommand{\listexercisename}{Liste der Aufgaben}%
\renewcommand{\ExerciseName}{Aufgabe}%
\renewcommand{\AnswerName}{L{\"o}sung zu Aufgabe}%
\renewcommand{\ExerciseListName}{Aufg.}%
\renewcommand{\AnswerListName}{L{\"o}sung}%
\renewcommand{\ExePartName}{Teil}%
\renewcommand{\ArticleOf}{von\ }%
%\renewcommand{\ExerciseHeaderTitle}{\ExerciseTitle}
\renewcommand{\ExerciseHeader}{%
	\textbf{\large\ExerciseHeaderDifficulty\ExerciseName\ %
	\ExerciseHeaderNB\normalsize\ExerciseHeaderTitle\ExerciseHeaderOrigin}\medskip}
\renewcommand{\AnswerHeader}{
	\newpage\textbf{L{\"o}sung zu \ExerciseName\ \ExerciseHeaderNB}\smallskip}


\definecolor{tcback}{rgb}{.95,.95,.95}
%Farbe für die Lösungen, die die Schüler selbst ausfüllen sollen
\definecolor{loes}{rgb}{1,1,1}
\definecolor{loestc}{rgb}{.95,.95,.95}
%Arbeitsblatt-Modus
\definecolor{loes}{rgb}{.36,.58,.93}
\definecolor{loestc}{rgb}{1, .4, 1}
\tcbset{
%	frame code={}
%	center title,
%	left=0pt,
%	right=0pt,
%	top=0pt,
%	bottom=0pt,
	colback=tcback,
%	colframe=white,
%	width=\dimexpr\textwidth\relax,
%	enlarge left by=0mm,
%	boxsep=5pt,
%	arc=0pt,outer arc=0pt,
}
%\sqrt[\leftroot{0}\uproot{2}n]{x}
\newcommand{\fakesection}[1]{%
	\par\refstepcounter{section}% Increase section counter
	\sectionmark{#1}% Add section mark (header)
	\addcontentsline{toc}{section}{\protect\numberline{\thesection}#1}% Add section to ToC
	% Add more content here, if needed.
}
\newcommand{\fakesubsection}[1]{%
	\par\refstepcounter{subsection}% Increase section counter
	\subsectionmark{#1}% Add section mark (header)
	\addcontentsline{toc}{subsection}{\protect\numberline{\thesubsection}#1}% Add section to ToC
	% Add more content here, if needed.
}

\begin{document}
	\setlength\parindent{0pt} %keine Einrückungen beim Start eines Paragraphen

	%Header
	\lohead{Mathematik}
	%\cohead{} %im Arbeitsblatt
	\rohead{Datum \phantom{00.00.0000}}
	\lehead{lehead}
	\cehead{cehead}
	\rehead{rehead}
%	\cofoot[plain content]{} %keine Seitenzahlen
	\cofoot[\pagemark]{\pagemark}
	\title{Mathematik\\
		Ein Skript für das Berufskolleg 2}
	\author{Hermann Maier}
%	\maketitle
%	\newpage
%	\tableofcontents
%	\newpage

	%\setlength\extrarowheight{10pt} %Horizontales padding für Tabellen
	\def\grundlagen{./Arbeitsblaetter/Grundlagen}
	\def\linFkt{./Arbeitsblaetter/LineareFunktionen}
	\def\quadFkt{./Arbeitsblaetter/QuadratischeFunktionen}
	\def\ganzFkt{./Arbeitsblaetter/GanzrationaleFunktionen}
	\def\eFkt{./Arbeitsblaetter/EFunktionen}
	\def\ableitung{./Arbeitsblaetter/Ableitung}
	\fakesection{Grundlagen}
	\input{\grundlagen/Mengen.tex}
	\newpage
	\input{\grundlagen/EinfachesRechnen.tex}
	\newpage
	\input{\grundlagen/Mitternachtsformel.tex}
	\newpage
	\cohead{\Large\textbf{Lösungen}}
	\fakesubsection{Lösungen}
	\shipoutAnswer
	\newpage
	\fakesection{Lineare Funktionen}
	\input{\linFkt/Einfuehrung.tex}
	\newpage
	\input{\linFkt/Steigungswinkel.tex}
	\newpage
	\input{\linFkt/Punktprobe.tex}
	\newpage
	\input{\linFkt/Nullstellen.tex}
	\newpage
	\cohead{\Large\textbf{Gegenseitige Lage von Geraden}}
\fakesubsection{Gegenseitige Lage von Geraden}

\begin{minipage}{\textwidth}
	\centering{\Large\textcolor{loes}{Parallele Geraden}}
	\adjustbox{valign=t}{\begin{minipage}{0.5\linewidth}\centering
		\includegraphics[width=0.95\textwidth]{\linFkt/pics/lage1.png}
		
		\(f_1(x)=\tfrac{1}{2}x-2\qquad g_1(x)=0,5x+1\)
	\end{minipage}}%
	\adjustbox{valign=t}{\begin{minipage}{0.5\linewidth}\centering
		\begin{tcolorbox}[width=0.95\textwidth, height=1.2cm, valign=center]\centering
			\textcolor{loestc}{\(m_f=m_g\)}
		\end{tcolorbox}
		\raggedright
		\textcolor{loes}{Die beiden Geraden haben die gleiche Steigung. Solche Paare von Geraden nennt man parallele Geraden. Sie haben keinen Schnittpunkt, d.h. die Gleichung \(f(x)=g(x)\) hat keine Lösungen. Paralle Geraden, die auch den gleichen y-Achsenabschnitt haben, nennt man identische Geraden. In diesem Fall ist jedes \(x\) eine Lösung der Gleichung \(f(x)=g(x)\)}.
	\end{minipage}}%
\end{minipage}

\bigskip

\begin{minipage}{\textwidth}
	\adjustbox{valign=t}{\begin{minipage}{0.5\linewidth}
		\centering{\Large\textcolor{loes}{Senkrechte Geraden}}
	\end{minipage}}%
	\adjustbox{valign=t}{\begin{minipage}{0.5\linewidth}
		\centering{\Large\textcolor{loes}{keine besondere Lage}}
	\end{minipage}}%
\end{minipage}%

\begin{minipage}{\textwidth}
	\adjustbox{valign=t}{\begin{minipage}{0.5\linewidth}
			\centering{\includegraphics[width=0.95\textwidth]{\linFkt/pics/lage2.png}}	
	\end{minipage}}%
	\adjustbox{valign=t}{\begin{minipage}{0.5\linewidth}
			\centering{\includegraphics[width=0.95\textwidth]{\linFkt/pics/lage3.png}}
	\end{minipage}}%
\end{minipage}%

\bigskip

\begin{minipage}{\textwidth}
	\adjustbox{valign=t}{\begin{minipage}{0.5\linewidth}
			\centering\(f_2(x)=-\tfrac{1}{2}x+2\qquad g_2(x)=2x-1\)
	\end{minipage}}%
	\adjustbox{valign=t}{\begin{minipage}{0.5\linewidth}
			\centering\(f_3(x)=\tfrac{1}{3}x+2\qquad g_3(x)=3x-3\)
	\end{minipage}}%
\end{minipage}%

\bigskip

\begin{minipage}{\textwidth}
	\adjustbox{valign=t}{\begin{minipage}{0.5\linewidth}\centering
			\begin{tcolorbox}[width=0.95\textwidth, height=1.2cm, valign=center]\centering
				\textcolor{loestc}{\(m_f\cdot m_g=-1\)}
			\end{tcolorbox}
	\end{minipage}}%
	\adjustbox{valign=t}{\begin{minipage}{0.5\linewidth}\centering
			\begin{tcolorbox}[width=0.95\textwidth, height=1.2cm, valign=center]\centering
				\textcolor{loestc}{\(m_f\neq m_g\textbf{ und }m_f\cdot m_g\neq-1\)}
			\end{tcolorbox}
	\end{minipage}}%
\end{minipage}%

\bigskip

\begin{minipage}{\textwidth}
	\adjustbox{valign=t}{\begin{minipage}{0.5\linewidth}
			\raggedright
			\textcolor{loes}{Die beiden Geraden schneiden sich in einem rechten Winkel. Solche Paare von Geraden stehen orthogonal bzw. normal zueinander.}	
	\end{minipage}}%
	\adjustbox{valign=t}{\begin{minipage}{0.5\linewidth}
			\raggedright
			\textcolor{loes}{Die beiden Geraden sind weder parallel noch orthogonal, d.h. sie haben keine besondere Lage zueinander.}
	\end{minipage}}%
\end{minipage}%
	\newpage
	\cohead{\Large\textbf{Schnittstellen und Schnittpunkte}}
\fakesubsection{Schnittstellen und Schnittpunkte}
Erinnerung: In der Mathematik unterscheidet man grundsätzlich zwischen Stellen und Punkten. Stellen sind x-Werte während Punkte einen x-Wert und einen y-Wert haben. Die Schnittpunkte zweier Funktionen $f(x)$ und $g(x)$ sind alle Punkte, in denen sich die Schaubilder schneiden. Um die Schnittstellen zu erhalten, muss man die Funktionen gleichsetzen:
\begin{tcolorbox}\centering
	$\textcolor{loestc}{f(x)=g(x)}$
\end{tcolorbox}
\begin{minipage}{0.49\textwidth}
	\includegraphics[width=.95\textwidth]{\linFkt/pics/schnittpunkt.png}
\end{minipage}
\begin{minipage}{0.49\textwidth}
	Im nebenstehenden Beispiel sind die Schaubilder der Funktionen $\textcolor{red}{f(x)=x+1}$ und $\textcolor{blue}{g(x)=3x-1}$ gezeichnet. Der Schnittpunkt lässt sich wie folgt berechnen:
	\begin{align*}
		\textcolor{red}{f(x)}&=\textcolor{blue}{g(x)}\\
		\textcolor{red}{x+1}&=\textcolor{blue}{3x-1}\ \rvert -3x-1\\
		-2x&=-2\ \rvert \cdot\left(-\tfrac{1}{2}\right)\\
		x&=1
	\end{align*}
\end{minipage}\smallskip\\
Die Schnittstelle ist also $\textcolor{ForestGreen}{x=1}$. Um die y-Koordinate zu erhalten, setzt man $\textcolor{ForestGreen}{x=1}$ entweder in $\textcolor{red}{f(x)}$ oder $\textcolor{blue}{g(x)}$ ein. Zur Demonstration setzen wir die Schnittstelle in beide Funktionen ein:
\begin{align*}
	\textcolor{red}{f(}\textcolor{ForestGreen}{1}\textcolor{red}{)}&=\textcolor{red}{\textcolor{ForestGreen}{1}+1}=\textcolor{YellowOrange}{2}\\
	\textcolor{blue}{g(}\textcolor{ForestGreen}{1}\textcolor{blue}{)}&=\textcolor{blue}{3\cdot \textcolor{ForestGreen}{1}-1}=\textcolor{YellowOrange}{2}
\end{align*}
Der Schnittpunkt liegt also bei $P\left(\textcolor{ForestGreen}{1}\lvert\textcolor{YellowOrange}{2}\right)$.
\begin{Exercise}[title={Bestimme jeweils den Schnittpunkt}, label=schnittpunktA1]\\
	\begin{minipage}{0.5\textwidth}
		\begin{enumerate}[label=\alph*)]
			\item $f_1(x)=x-1$ und $g_1(x)=-x+3$
			\item $f_2(x)=-2x+4$ und $g_2(x)=0,5x-1$
			\item $f_3(x)=\frac{3}{2}x+\frac{1}{2}$ und $g_3(x)=4x$
		\end{enumerate}
	\end{minipage}
	\begin{minipage}{0.5\textwidth}
		\begin{enumerate}[label=\alph*)]
			\setcounter{enumi}{3}
			\item $f_4(x)=\frac{4}{5}x+\frac{2}{5}$ und $g_4(x)=-\frac{2}{5}x$
			\item $f_5(x)=-\frac{2}{3}x-15$ und $g_5(x)=3x-\frac{5}{4}$
			\item $f_6(x)=-\frac{5}{8}x$ und $g_6(x)=-\frac{3}{2}x+\frac{1}{2}$
		\end{enumerate}
	\end{minipage}
\end{Exercise}\vspace{.5cm}
\begin{Answer}[ref=schnittpunktA1]\\
	\begin{minipage}{0.5\textwidth}
		\begin{enumerate}[label=\alph*)]
			\item $P_1\left(2\vert 1\right)$
			\item $P_2\left(2\vert 0\right)$
			\item $P_3\left(\frac{1}{5}\vert \frac{4}{5}\right)$
		\end{enumerate}
	\end{minipage}
	\begin{minipage}{0.5\textwidth}
		\begin{enumerate}[label=\alph*)]
			\setcounter{enumi}{3}
			\item $P_4\left(-\frac{1}{3}\vert -\frac{2}{15}\right)$
			\item $P_5\left(-\frac{15}{4}\vert -\frac{25}{2}\right)$
			\item $P_6\left(\frac{4}{7}\vert -\frac{5}{14}\right)$
		\end{enumerate}
	\end{minipage}
\end{Answer}
	\newpage
	\cohead{\Large\textbf{Lösungen}}
	\fakesubsection{Lösungen}
	\shipoutAnswer
	\newpage
	\fakesection{Quadratische Funktionen}
	\input{\quadFkt/Scheitelform.tex}
	\newpage
	\input{\quadFkt/Hauptform.tex}
	\newpage
	\input{\quadFkt/Produktform.tex}
	\newpage
	\cohead{\Large\textbf{Lösungen}}
	\fakesubsection{Lösungen}
	\shipoutAnswer
	\newpage
	\fakesection{Ganzrationale Funktionen}
	\input{\ganzFkt/Einfuehrung.tex}
	\newpage
	\cohead{\Large\textbf{Potenzfunktionen}}
\fakesubsection{Potenzfunktionen}
Funktionen vom Typ
\[f(x)=a\cdot x^n, \quad a\neq 0, \quad n\in\N\]
bezeichnen wir als Potenzfunktionen.\\
Der Koeffizient \(a\) ist der Streckfaktor, wie wir ihn bereits von quadratischen Funktionen kennen.\\
Die Hochzahl bzw. der Exponent \(n\) ist eine natürliche Zahl: \(\N=\{1,\,2,\,3,\,4,\,\dots\}\)\\
Die Schaubilder der Potenzfunktionen teilen sich in drei verschiedene Formen auf:\vspace{0.3cm}\\
\begin{tabular}{cc}
	\begin{minipage}{0.6\textwidth}
		\centering\Large\textcolor{loes}{Für \(n=1\) ergibt sich eine Gerade.}
	\end{minipage}
	&
	\begin{minipage}{0.39\textwidth}
		\includegraphics[width=.95\linewidth]{\ganzFkt/pics/potenzGerade.png}
	\end{minipage} \\
	\midrule
	\begin{minipage}{0.6\textwidth}
		\centering\Large\textcolor{loes}{Gerade Hochzahlen: \(x^2,\ x^4,\ x^6,\ \dots\)\\
			Parabelförmig\\
			Achsensymmetrie zur y-Achse\\
			\(f(x)\xrightarrow{\hphantom{\ }x\to-\infty\hphantom{\ }}\infty\)\\
			\(f(x)\xrightarrow{\hphantom{\ }x\to\infty\hphantom{\ }}\infty\)
		}
	\end{minipage}
	&
	\begin{minipage}{0.39\textwidth}
		\includegraphics[width=.95\linewidth]{\ganzFkt/pics/potenzGeradeHZ.png}
	\end{minipage} \\
	\midrule
	\begin{minipage}{0.6\textwidth}
		\centering\Large\textcolor{loes}{Ungerade Hochzahlen (größer 1): \(x^3,\ x^5,\ x^7,\ \dots\)\\
			S-förmig\\
			Punktsymmetrie zum Ursprung\\
			\(f(x)\xrightarrow{\hphantom{\ }x\to-\infty\hphantom{\ }}-\infty\)\\
			\(f(x)\xrightarrow{\hphantom{\ }x\to\infty\hphantom{\ }}\infty\)
		}
	\end{minipage}
	&
	\begin{minipage}{0.39\textwidth}
		\includegraphics[width=.95\linewidth]{\ganzFkt/pics/potenzUngeradeHZ.png}
	\end{minipage} \\
\end{tabular}
\newpage
%%%%%%%%%%%%%%%%%%%%%%%%%%%%%%%%%%%%%%%%%%%%%%%%%%%%%%%%%%%%%%%%%%%%%%%%%%%%%%%%%%%%%%%%%%%%%%%%%%%%%%%%%%%%%%%%%%%%%
\begin{Exercise}[title={Skizziere das Schaubild, gib die Symmetrie sowie das Verhalten für sehr große/kleine \(x\) an.}, label=potenzA1]\\
	\begin{minipage}{\textwidth}
		\begin{minipage}{0.49\textwidth}
			\begin{enumerate}[label=\alph*)]
				\item \(f(x)=-x^2\)
				\item \(g(x)=0,5x^3\)
				\item \(h(x)=2x^6\)
				\item \(i(x)=-\frac{3}{2}x^5\)
			\end{enumerate}
		\end{minipage}
		\begin{minipage}{0.49\textwidth}
			\begin{enumerate}[label=\alph*)]
				\setcounter{enumi}{4}
				\item \(j(x)=0,1x^4\)
				\item \(k(x)=-\frac{3}{5}x^7\)
				\item \(l(x)=-\sqrt{2}x^4\)
				\item \(m(x)=3x^5\)
			\end{enumerate}
		\end{minipage}
	\end{minipage}
\end{Exercise}
\newpage
%%%%%%%%%%%%%%%%%%%%%%%%%%%%%%%%%%%%%%%%%
\begin{Answer}[ref=potenzA1]\\
	Man muss nur das Vorzeichen des Streckfaktors \(a\) beachten sowie ob die Hochzahl gerade oder ungerade ist:\vspace{0.5cm}\\
	\begin{tabular}{c|c}
		\begin{minipage}{0.49\textwidth}\centering
			\(a\) positiv und \(n\) gerade wie \(h(x)\) und \(j(x)\)\\
			Parabelförmig\\
			Achsensymmetrie zur y-Achse\\
			\(f(x)\xrightarrow{\hphantom{\ }x\to-\infty\hphantom{\ }}\infty\)\\
			\(f(x)\xrightarrow{\hphantom{\ }x\to\infty\hphantom{\ }}\infty\)\\
			\includegraphics[width=.95\linewidth]{\ganzFkt/pics/potenzPosGerA1.png}\vspace{0.5cm}
		\end{minipage}&
		\begin{minipage}{0.49\textwidth}\centering
			\(a\) negativ und \(n\) gerade wie \(f(x)\) und \(l(x)\)\\
			Parabelförmig\\
			Achsensymmetrie zur y-Achse\\
			\(f(x)\xrightarrow{\hphantom{\ }x\to-\infty\hphantom{\ }}-\infty\)\\
			\(f(x)\xrightarrow{\hphantom{\ }x\to\infty\hphantom{\ }}-\infty\)\\
			\includegraphics[width=.95\linewidth]{\ganzFkt/pics/potenzNegGerA1.png}\vspace{0.5cm}
		\end{minipage}\\ \hline
		\begin{minipage}{0.49\textwidth}\centering\vspace{0.5cm}
			\(a\) positiv und \(n\) ungerade wie \(g(x)\) und \(m(x)\)\\
			S-förmig\\
			Punktsymmetrie zum Ursprung\\
			\(f(x)\xrightarrow{\hphantom{\ }x\to-\infty\hphantom{\ }}-\infty\)\\
			\(f(x)\xrightarrow{\hphantom{\ }x\to\infty\hphantom{\ }}\infty\)\\
			\includegraphics[width=.95\linewidth]{\ganzFkt/pics/potenzPosUngerA1.png}
		\end{minipage}&
		\begin{minipage}{0.49\textwidth}\centering\vspace{0.5cm}
			\(a\) negativ und \(n\) ungerade wie \(i(x)\) und \(k(x)\)\\
			S-förmig\\
			Punktsymmetrie zum Ursprung\\
			\(f(x)\xrightarrow{\hphantom{\ }x\to-\infty\hphantom{\ }}\infty\)\\
			\(f(x)\xrightarrow{\hphantom{\ }x\to\infty\hphantom{\ }}-\infty\)\\
			\includegraphics[width=.95\linewidth]{\ganzFkt/pics/potenzNegUngerA1.png}
		\end{minipage}
	\end{tabular}
\end{Answer}
	\newpage
	\input{\ganzFkt/Hauptform.tex}
	\newpage
	\input{\ganzFkt/Symmetrie.tex}
	\newpage
	\input{\ganzFkt/Verhalten.tex}
	\newpage
	\input{\ganzFkt/Nullstellen.tex}
	\newpage
	\input{\ganzFkt/Produktform.tex}
	\newpage
	\cohead{\Large\textbf{Lösungen}}
	\fakesubsection{Lösungen}
	\shipoutAnswer
	\newpage
	\fakesection{Exponentialfunktionen}
	\input{\eFkt/Einfuehrung.tex}
	\newpage
 	\input{\eFkt/Begriffe.tex}
	\newpage
 	\input{\eFkt/WaagrechteAsymptoten.tex}
 	\newpage
 	\input{\eFkt/Ln.tex}
 	\newpage
 	\input{\eFkt/FktAuf.tex}
 	\newpage
 	\input{\eFkt/SchiefeAsymptoten.tex}
 	\newpage
 	\input{\eFkt/NaehrungsweiseNST.tex}
 	\newpage
 	\cohead{\Large\textbf{Lösungen}}
 	\fakesubsection{Lösungen}
 	\shipoutAnswer
 	\newpage

% 	\input{\ableitung/MittlereAenderungsrate.tex}
% 	\newpage
% 	\input{\ableitung/GrafischesAbleitung.tex}
% 	\newpage
% 	\input{\ableitung/MomentanPunktweise.tex}
% 	\newpage
% 	\input{\ableitung/MomentanAllg.tex}
% 	\newpage
% 	\input{\ableitung/Ableitungsregeln.tex}
% 	\newpage
% 	\cohead{\Large\textbf{Faktorregel}}
\fakesubsection{Faktorregel}
Wir kennen nun die Ableitung der Normalparabel. Es scheint naheliegend, dass die Ableitungen von \(2x^2,\ 3x^2,\ -x^2\) und ähnlichen Funktionen eine zu \(x^2\) ähnliche Ableitung haben. Wir berechnen die Ableitung von \(f_a(x)=ax^2\):\\
\begin{minipage}[t]{\textwidth}
	\begin{minipage}{0.4\textwidth}
		\begin{align*}
			f_a(x)&=ax^2\\
			f_a'(x)&=\textcolor{loes}{\lim\limits_{h\to 0}\frac{f_a(x+h)-f_a(x)}{h}}\\
			&\textcolor{loes}{=\lim\limits_{h\to 0}\frac{a(x+h)^2-ax^2}{h}}\\
			&\textcolor{loes}{=\lim\limits_{h\to 0}\frac{ax^2+a2hx+ah^2-ax^2}{h}}\\
			&\textcolor{loes}{=\lim\limits_{h\to 0}\frac{a2hx+ah^2}{h}}\\
			&\textcolor{loes}{=\lim\limits_{h\to 0}a2x+ah}\\
			&\textcolor{loes}{=a2x}\\
		\end{align*}
	\end{minipage}
	\begin{minipage}{0.6\textwidth}
		\textcolor{loes}{\(f(x)=x^2\) und \(f_a(x)=ax^2\) unterscheiden sich nur durch den Faktor \(a\). Das gleiche Verhalten zeigen die Ableitungen. \(f'(x)=2x\) und \(f_a'(x)=a2x\) unterscheiden sich ebenfalls nur durch den Faktor \(a\).}
	\end{minipage}
\end{minipage}\\
Tatsächlich lässt sich ganz allgemein zeigen, dass die Ableitung einer Funktion und einem Faktor \(af(x)\) gleich dem gleichen Faktor mal der Ableitung ist: \((af(x))'=af'(x)\):\\
\begin{minipage}[t]{\textwidth}
	\begin{minipage}{0.5\textwidth}
		\begin{align*}
			g(x)&=af(x)\\
			g'(x)&=\textcolor{loes}{\lim\limits_{h\to 0}\frac{af(x+h)-af(x)}{h}}\\
			&\textcolor{loes}{=a\lim\limits_{h\to 0}\frac{f(x+h)-f(x)}{h}}\\
			&\textcolor{loes}{=af'(x)}\\
		\end{align*}
	\end{minipage}
	\begin{minipage}{0.5\textwidth}
		\begin{tcolorbox}
			\phantom{text}\\
			\textcolor{loestc}{Faktorregel:\\
				Die Ableitung von \(af(x)\) ist \(af'(x)\).}\\
		\end{tcolorbox}
	\end{minipage}
\end{minipage}\\
% 	\newpage
% 	\input{\ableitung/Potenzregel.tex}
% 	\newpage
% 	\input{\ableitung/Summenregel.tex}
% 	\newpage

%\cohead{\Large\textbf{Abl. von e-Fkt.}}
%\fakesubsection{Ableitung von e-Funktionen}
%Die Faktor- und Summenregel gelten analog auch für e-Funktionen. Funktionen vom Typ \(f(x)=ae^{kx}\) lassen sich wie folgt ableiten:
%\begin{tcolorbox}
%	\phantom{text}\\
%	\textcolor{loestc}{Ableitung von e-Funktionen:\\
%	Eine Funktion vom Typ \(f(x)=ae^{kx}\) hat die Ableitung \(f'(x)=ake^{kx}\).\\
%	Es wird also nur der Faktor \(k\) nach unten geholt, alles andere bleibt gleich.}\\
%\end{tcolorbox}
%%%%%%%%%%%%%%%%%%%%%%%%%%%%%%%%%%%%%%%%%%%%%%%%%%%%%%%%%%%%%%%%%%%%%%%%%%%%%%%%%%%%%%%%%%%%%%%%%%%%%%%
%\begin{minipage}{\textwidth}
%	\begin{Exercise}[title={\raggedright Berechne jeweils allgemein die Ableitung \(f'(x)\)}, label=summenregelA1]
%	\begin{minipage}{\textwidth}
%	\begin{minipage}{0.49\textwidth}
%		\begin{enumerate}[label=\alph*)]
%			\item \(f_1(x)=e^x\)
%			\item \(f_2(x)=3e^x-e^{3x}\)
%			\item \(f_3(x)=e^{-x}+2\)
%			\item \(f_4(x)=3e^{0,5x}+2e^x\)
%			\item \(f_5(x)=-4e^{\frac{3}{5}x}+2e^{\frac{1}{4}x}\)
%			\item \(f_6(x)=e^{-\frac{7}{8}x}+2e^{-\frac{1}{2}x}\)
%			\item \(f_7(x)=-e^{3x}-2e^{x}+5\)
%			\item \(f_8(x)=0,5e^{4x}-2e^{2x}+e^x\)
%		\end{enumerate}
%	\end{minipage}
%	\begin{minipage}{0.49\textwidth}
%		\begin{enumerate}[label=\alph*)]
%			\setcounter{enumi}{8}
%			\item \(f_9(x)=3e^{-2x}-e^{-x}\)
%			\item \(f_{10}(x)=\frac{1}{4}e^{\frac{2}{5}x}+4x\)
%			\item \(f_{11}(x)=\)
%			\item \(f_{12}(x)=\)
%			\item \(f_{13}(x)=\)
%			\item \(f_{14}(x)=\)
%			\item \(f_{15}(x)=3e^x-4e\)
%			\item \(f_{16}(x)=\)
%		\end{enumerate}
%	\end{minipage}
%	\end{minipage}
%	\end{Exercise}
%\end{minipage}
%%%%%%%%%%%%%%%%%%%%%%%%%%%%%%%%%%%%%%%%%%
%\begin{Answer}[ref=summenregelA1]\\
%	\begin{minipage}{\textwidth}
%	\begin{minipage}{0.5\textwidth}
%		\begin{enumerate}[label=\alph*)]
%			\item \(f_1'(x)=6x^2-4\)
%			\item \(f_2'(x)=8x^3\)
%			\item \(f_3'(x)=6x^2-2x+7\)
%			\item \(f_4'(x)=8x\)
%			\item \(f_5'(x)=12x^3+3x^2-8\)
%			\item \(f_6'(x)=\frac{5}{2}x^2+10x\)
%			\item \(f_7'(x)=-6x^3+3x^2-2\)
%			\item \(f_8'(x)=16x^3+9x^2-4x\)
%		\end{enumerate}
%	\end{minipage}
%	\begin{minipage}{0.5\textwidth}
%		\begin{enumerate}[label=\alph*)]
%			\setcounter{enumi}{8}
%			\item \(f_9'(x)=6x^3-4,6x\)
%			\item \(f_{10}'(x)=-\frac{3}{2}x+2\)
%			\item \(f_{11}'(x)=\frac{8}{3}x^3-\frac{16}{3}x\)
%			\item \(f_{12}'(x)=-\frac{5}{3}x^{10}+\frac{4}{9}x^8\)
%			\item \(f_{13}'(x)=\frac{2}{3}x^3+\frac{3}{4}x\)
%			\item \(f_{14}'(x)=6x^2-8x\)
%			\item \(f_{15}'(x)=2x+2\)
%			\item \(f_{16}'(x)=2x-1\)
%		\end{enumerate}
%	\end{minipage}
%\end{minipage}\\  \\
%Anmerkung: Bei n), o) und p) muss man zuerst ausmultiplizieren bevor man ableiten kann
%\end{Answer}

%\begin{tcolorbox}\centering
%	\textcolor{loestc}{Das Schaubild einer ganzrationalen Funktion ist\dots\\
	%		\dots achsensymmetrisch zur y-Achse, wenn alle Hochzahlen gerade oder Null sind.\\
	%		\dots punksymmetrisch zum Ursprung, wenn alle Hochzahlen ungerade sind.\\
	%		\dots weder achsensymmetrisch zur y-Achse noch punktsymmetrisch zum Ursprung, wenn die Hochzahlen eine Mischung aus geraden Hochzahlen oder Null und ungeraden Hochzahlen sind.}
%\end{tcolorbox}
\newpage
\cohead{\Large\textbf{Lösungen}}
\fakesubsection{Lösungen}
\shipoutAnswer
\end{document}



%\begin{Exercise}[title={xxxx}, label=xxxx]\\
%xxxx
%\end{Exercise}
%\newpage
%\begin{Answer}[ref=xxx]\\
%xxxx
%\end{Answer}
%\begin{Exercise}[title=xxxx}, label=xxxx]\\
%xxxx
%\end{Exercise}
%\newpage
%\begin{Answer}[ref=xxx]\\
%xxxx
%\end{Answer}


%\begin{figure}[h]
%	\centering
%	\includegraphics[width=0.8\textwidth]{\quadFkt/pics/steigungswinkel.png}
%\end{figure}