\section{Grundlagen}
Die folgenden mathematischen Grundkenntnisse sind unabdingbare Voraussetzung zum Verst�ndnis der folgenden Kapitel. 
\subsection{Mengen}
Die Mengenlehre ist ein grundlegendes Teilgebiet der Mathematik. Wir werden uns hier nur mit den n�tigen Grundlagen der Mengenlehre besch�ftigen. Die f�r uns relevanten Mengen bestehen aus Zahlen:\\
\begin{tcolorbox}
\textbf{Zahlenmengen}
\begin{itemize}
	\item Die nat�rlichen Zahlen: ${\N=\{(0), 1; 2; 3; 4; 5; 6; \ldots\}}$\\Die Mathematiker k�nnen sich nicht einigen, ob die 0 mit eingeschlossen sein soll. Daher wird meist $\N^*$ f�r die nat�rlichen Zahlen ohne die 0 verwendet und $\N_0$ f�r die nat�rlichen Zahlen mit der 0.
	\item Die ganzen Zahlen: ${\Z=\{0; 1; -1; 2; -2; 3; -3; 4; -4;\ldots\}}$\\Erg�nzt man die nat�rlichen Zahlen um das Vorzeichen, so erh�lt man die ganzen Zahlen.
	\item Die rationalen Zahlen: ${\Q=\{\frac{n}{m}|n\in\Z, m\in\N\}}$\\Die rationalen Zahlen enthalten alle Zahlen, die sich als Br�che mit einem Z�hler aus den ganzen Zahlen und einem Nenner aus den nat�rlichen Zahlen (nat�rlich ohne der 0) darstellen lassen.
	\item Die reellen Zahlen: $\R$\\In den reellen Zahlen $\R$ liegen "`alle"' Zahlen, zumindest alle uns bekannten Zahlen. $\R$ beinhaltet neben $\Q$ auch Zahlen wie $\sqrt{2}$ oder $\pi$.
\end{itemize}
\end{tcolorbox}
Mengen lassen sich auf verschiedene Arten darstellen. Nehmen wir als Beispiele die Menge aller positiven, geraden Zahlen $G$ und die Menge $H$ aller Zahlen, die gr��er oder gleich $1$ und kleiner $2$ sind:
\begin{itemize}
	\item Aufz�hlung: ${G=\{2; 4; 6; 8; 10;\ldots \}}$
	\item Einschr�nkung einer �bergeordneten Menge ${G=\{x\in\N|x\text{ ist gerade}\}}$ oder\linebreak[4] ${H=\{x\in\R|-1<=x<2\}}$
	\item Darstellung als Intervall: $H=[-1;2)$ Dabei steht die eckige Klammer f�r ein abgeschlossenes Ende, d.h. die Grenze liegt noch im Intervall und die runde Klammer f�r ein offenes Intervall, d.h. die Grenze liegt nicht mehr im Intervall.
\end{itemize}
Liegt eine Zahl in einer Menge, z.B. $-2$ in $\Q$, so schreibt man ${-2\in\Q}$ (Sprich $-2$ ist Element der rationalen Zahlen).
\subsection{Grundrechenarten und Rechengesetze}
\begin{tcolorbox}
\begin{itemize}
	\item Addition\\$\underbrace{a+b}_{Summe}=c$\\$a$ und $b$ bezeichnet man als Summanden.
	\item Subtraktion\\$\underbrace{a-b}_{Differenz}=c$
	\item Multiplikation\\$\underbrace{a\cdot b}_{Produkt}=c$\\$a$ und $b$ bezeichnet man als Faktoren.
	\item Division\\$\underbrace{a:b}_{Quotient}=c$\\Wir sollten uns das Geteilt-Zeichen abgew�hnen und stattdessen Br�che verwenden: $a:b=\frac{a}{b}$.
\end{itemize}
\end{tcolorbox}
Addition und Multiplikation sind kommutativ, d.h. man kann die Reihenfolge der Summanden bzw. Faktoren vertauschen:
\begin{align*}
	a+b&=b+a\\
	a\cdot b&=b\cdot a
\end{align*}

\subsection{Binomische Formeln}
Es gibt 3 binomische Formeln, die f�r beliebige ${a, b\in\R}$ gelten:
\begin{tcolorbox}
\textbf{1. Binomische Formel}
\begin{align*}
	(a+b)^2=a^2+2ab+b^2
\end{align*}
\textbf{2. Binomische Formel}
\begin{align*}
	(a-b)^2=a^2-2ab+b^2
\end{align*}
\textbf{3. Binomische Formel}
\begin{align*}
	(a+b)\cdot(a-b)=a^2-b^2
\end{align*}
\end{tcolorbox}
\subsection{Bruchrechnen}
\begin{tcolorbox}
\begin{defi}
Um zwei Br�che zu Addieren/Subtrahieren, m�ssen zuerst beide Br�che auf den gleichen Nenner gebracht werden (Hauptnenner) und dann die Z�hler addiert/subtrahiert werden.
\end{defi}
\end{tcolorbox}
\begin{bsp}
\begin{align*}
&\frac{1}{2}+\frac{2}{3}=\frac{3}{6}+\frac{4}{6}=\frac{3+4}{6}=\frac{7}{6}\\
&\frac{7}{4}+\frac{2}{x}=\frac{7x}{4x}+\frac{8}{4x}=\frac{7x+8}{4x}\\
&\frac{1}{x+1}-\frac{1}{x-1}=\frac{x-1}{(x+1)(x-1)}-\frac{x+1}{(x+1)(x-1)}=\frac{x-1-(x+1)}{x^2-1}=\frac{-2}{x^2-1}
\end{align*}
\end{bsp}

\begin{tcolorbox}
\begin{defi}
Um zwei Br�che zu Multiplizieren, werden die Z�hler miteinander multipliziert und die Nenner miteinander multipliziert. Innerhalb eines Produkts darf direkt gek�rzt werden.
\end{defi}
\end{tcolorbox}
\begin{bsp}
\begin{align*}
&\frac{1}{2}\cdot\frac{1}{3}=\frac{1\cdot 1}{2\cdot 3}=\frac{1}{6}\\
&\frac{1}{2}\cdot\frac{2}{3}=\frac{1\cdot 2}{2\cdot 3}=\frac{1\cdot \cancel{2}}{\cancel{2}\cdot 3}=\frac{1}{3}\\
&\frac{4}{7}\cdot\frac{x}{2}=\frac{\cancel{4}_2\cdot x}{7\cdot \cancel{2}}=\frac{2x}{7}=\frac{2}{7}x
\end{align*}
\end{bsp}

\begin{tcolorbox}
\begin{defi}
Zwei Br�che werden dividiert, indem mit dem Kehrwert multipliziert wird.
\end{defi}
\end{tcolorbox}
\begin{bsp}
\begin{align*}
&\frac{1}{2}:\frac{5}{3}=\frac{1}{2}\cdot\frac{3}{5}=\frac{3}{10}\\
&\frac{3}{2}:\frac{3}{4}=\frac{3}{2}\cdot\frac{4}{3}=\frac{\cancel{3}\cdot\cancel{4}^2}{\cancel{2}\cdot \cancel{3}}=2\\
&\frac{7}{4x}:\frac{2}{x}=\frac{7}{4x}\cdot\frac{x}{2}=\frac{7\cdot\cancel{x}}{4\cdot\cancel{x}\cdot 2}=\frac{7}{8}
\end{align*}
\end{bsp}
\subsection{Rechnen mit Variablen}
Variablen sind in der Mathematik Platzhalter f�r Zahlen, deren Wert man nicht kennt. Mit ihrer Hilfe kann man allgemeine Zusammenh�nge aufstellen, z.B. lautet der Zusammenhang zwischen der Fl�che eines Rechtecks und seinen Seitenl�ngen:
\begin{tcolorbox}
	Fl�cheninhalt eines Rechtecks
\begin{align*}
	A&=a\cdot b\\
	A&: \text{Fl�che des Rechtecks}\\
	a, b&: \text{Seitenl�ngen des Rechtecks}
\end{align*}
\end{tcolorbox}
Kennt man zwei der drei Gr��en, kann man die fehlende berechnen. 
\subsection{Ausklammern}
Ausklammern oder Vorklammern kann man Zahlen oder auch Variablen. Dabei �ndert man den Wert des mathematischen Ausdrucks nicht, sondern lediglich sein Aussehen. Klammert man Variablen vor (im Normalfall $x$), so ist es in den meisten F�llen nicht sinnvoll die Variable �fter als die kleinste Hochzahl vorzuklammern, da dann die Variable im Nenner des Bruches stehen w�rde.
\begin{bsp}
	\begin{align*}
		&2x^2-4x=x(2x-4)=2x(x-2)\\
		&10x^3-5x^2+25x=5x(2x^2-x+5)\\
		&27x^4-18x^2=9x^2(3x^2-2)
	\end{align*}
\end{bsp}
\subsection{Potenzgesetze}
F�r uns ist nur eines der Potenzgesetze relevant:
\begin{tcolorbox}
	\begin{defi}
		Zwei Potenzen mit der gleichen Basis werden multipliziert, indem man die Hochzahlen addiert:
		\begin{align*}
		x^a\cdot x^b=x^{a+b}
	\end{align*}
	\end{defi}
\end{tcolorbox}